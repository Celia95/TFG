\documentclass[spanish, a4paper, 12pt, final, slideColor, nototal, colorBG, pdf, noaccumulate, darkblue]{beamer}
\usepackage[spanish]{babel}
\usepackage[utf8]{inputenc}
\usepackage{latexsym}
\usepackage{soul}
\usepackage{import}
\usepackage{multicol}
\usepackage{graphicx}
\usepackage{hyperref}
\usepackage{caption}
\usepackage{listings}
\captionsetup{font=scriptsize,labelfont=scriptsize}
\DeclareGraphicsExtensions{.pdf,.png,.jpg}

% Mathematics
\usepackage{amssymb}
\usepackage{amsfonts}
\usepackage{anysize}
\usepackage{float}
\usepackage{amsthm}
\usepackage{amsmath}
\usepackage{verbatim}
\usepackage{mathtools}
\usefonttheme[onlymath]{serif}

\providecommand{\abs}[1]{\left\lvert#1\right\rvert}
\providecommand{\norm}[1]{\left\lVert#1\right\rVert}
\providecommand{\norminf}[1]{\norm{#1}_{\infty}}
\providecommand{\holomorphic}[1]{\mathcal{H}(#1)}
\providecommand{\bholomorphic}[1]{\mathcal{H}^{\infty}(#1)}
\newcommand{\rpm}{\raisebox{.2ex}{$\scriptstyle\pm$}}

\newcommand*\xbar[1]{%
   \hbox{%
     \vbox{%
       \hrule height 0.5pt % The actual bar
       \kern0.2ex%         % Distance between bar and symbol
       \hbox{%
         \kern-0.1em%      % Shortening on the left side
         \ensuremath{#1}%
         \kern-0.1em%      % Shortening on the right side
       }%
     }%
   }%
}

\newcommand{\complex}{\mathbb{C}}
\newcommand{\real}{\mathbb{R}}
\newcommand{\integer}{\mathbb{Z}}
\newcommand{\naturals}{\mathbb{N}}
\newcommand{\disk}{\mathbb{D}}
\newcommand{\closedisk}{\overline{\disk}}
\newcommand{\diam}{\operatorname{diam}}
\renewcommand{\Re}{\operatorname{Re}}
\renewcommand{\Im}{\operatorname{Im}}
\newcommand{\id}{\operatorname{id}}
\newcommand{\fiber}{\mathfrak{M}}

\makeatletter
\newcommand{\leqnomode}{\tagsleft@true}
\newcommand{\reqnomode}{\tagsleft@false}
\makeatother

\newtagform{Alph}[\renewcommand{\theequation}{\Alph{equation}}]()
\newtagform{alph}[\renewcommand{\theequation}{\alph{equation}}]()
\newtagform{Roman}[\renewcommand{\theequation}{\roman{equation}}]()
\newtagform{roman}[\renewcommand{\theequation}{\roman{equation}}]()
\newtagform{scroman}[\renewcommand{\theequation}{\scshape\roman{equation}}][]

\newcounter{align}[equation]
\renewcommand{\thealign}{\roman{align}}
\newcommand{\alignno}{\refstepcounter{align}\tag{\thealign}}
\expandafter\let\expandafter\oldalignstar\csname align*\endcsname
\expandafter\def\csname align*\endcsname{\refstepcounter{equation}\oldalignstar}

\usetheme{Madrid}

\title{Problemas geométricos que arrancan de la teoría clásica de funciones}
\author{Celia de Frutos Palacios \thanks{\url{https://github.com/Celia95/TFG}}}
\subtitle{}
\institute[UCM]{}

\logo{\includegraphics[height=0.7cm]{Imagenes/Vectorial/escudoUCM.png}}

\date{\today}

\begin{document}
\maketitle

\AtBeginSection[] % To put de table of contents at the beggining of each section
{
  \begin{frame}
    \frametitle{Índice}
    \tableofcontents[currentsection]
  \end{frame}
}

%\begin{frame}
%    \frametitle{Índice}
%    \tableofcontents
%\end{frame}

\section{Teorema de Fatou. Teorema de Carathéodory}

\begin{frame}
\frametitle{Sample frame title}
This is a text in second frame.
For the sake of showing an example.

\begin{itemize}
    \item<1-> Text visible on slide 1
    \item<2-> Text visible on slide 2
    \item<3> Text visible on slide 3
    \item<4-> Text visible on slide 4
\end{itemize}

\end{frame}

\begin{frame}
In this slide \pause

the text will be partially visible \pause

And finally everything will be there
\end{frame}


\begin{frame}
    \frametitle{Sample frame title}

    In this slide, some important text will be \alert{highlighted} beause it's important. Please, don't abuse it.

    \begin{block}{Remark}
        Sample text
    \end{block}

    \begin{alertblock}{Important theorem}
        Sample text in red box
    \end{alertblock}

    \begin{examples}
        Sample text in green box. "Examples" is fixed as block title.
    \end{examples}
\end{frame}

\begin{frame}
    \frametitle{Problema de Dirichlet}
    \begin{block}{}

    \end{block}
\end{frame}


\begin{frame}
    \frametitle{Integral de Poisson}
    Integral de Poisson de una función $f \in L^1(\partial \disk)$:
    \begin{equation*}
        F = P[f]: z=re^{i \theta} \in \disk \mapsto F(re^{i \theta}) = \dfrac{1}{2 \pi} \int_{- \pi}^{\pi} P_r (\theta - t) f(e^{it}) dt.
    \end{equation*}

    La integral de Poisson proporciona una solución afirmativa al problema de Dirichlet.

    \begin{equation*}
        u : z = re^{i \theta} \in \closedisk \mapsto u(re^{i \theta}) =
        \begin{cases}
            f(e^{i\theta}) & \text{si } r=1 \\
            F(re^{i\theta}) & \text{si } 0 \leq r<1
        \end{cases}
    \end{equation*}
\end{frame}

\begin{frame}
    \frametitle{Teorema de Fatou}
    \begin{block}{Teorema de Fatou}
         Para toda función $f \in \bholomorphic{\disk}$, existe una función $f^* \in L^{\infty} (\partial \disk)$ definida por
    \begin{equation}
        f^*(e^{it}) = \lim_{r \to 1} f(re^{it})
    \end{equation}
    en casi todo punto.

    Se tiene la igualdad $\norminf{f} = \norminf{f^*}$. Para todo $z \in \disk$, la fórmula integral de Cauchy
    \begin{equation}
        f(z) = \dfrac{1}{2 \pi i} \int_{\gamma} \dfrac{f^*(\xi)}{\xi - z} d\xi
    \end{equation}

    se satisface, donde $\gamma$ es el círculo unidad positivamente orientado: $\gamma(t) = e^{it}, 0 \leq t \leq 2 \pi$.

    Las funciones $f^* \in L^{\infty}(\partial \disk)$ que se obtienen mediante este procedimiento son precisamente aquellas que cumplen la siguiente relación
    \begin{equation}
        \dfrac{1}{2 \pi i} \int_{-\pi}^{\pi} f^*(e^{it})e^{-int} dt = 0, n = -1,-2, \dots
    \end{equation}
    \end{block}
\end{frame}

\begin{frame}
    \frametitle{Teorema de Carathéodory}
    \begin{block}{Teorema de Carathéodory}
        Sea $\varphi$ una aplicación conforme del disco unidad $\disk$ en un dominio de Jordan $\Omega$. Entonces $\varphi$ tiene una extensión continua al disco cerrado $\closedisk$, y la extensión es inyectiva de $\closedisk$ en $\xbar{\Omega}$.
    \end{block}
\end{frame}

\section{Series de potencias. Productos infinitos}

\section{$\bholomorphic{\disk}$ como álgebra de Banach}

\section{Transformaciones en el disco}


\begin{frame}
    \frametitle{Teorema de Julia}
    \begin{block}{Teorema de Julia}
        Si $f$ es una función holomorfa del disco $\disk$ en sí mismo no constante, y existen $w, \mu \in \partial \disk$ y una sucesión $\{p_n\} \in \disk$ que verifican
    {
    \leqnomode
    \setcounter{align}{0}
    \renewcommand{\thealign}{\alph{align}}
    \begin{align}
        & \lim_{n \to \infty} p_n = w;
        \alignno \\
        & \lim_{n \to \infty} f(p_n) = \mu;
        \alignno \\
        & \lim_{n \to \infty} \frac{1-\abs{f(p_n)}}{1-\abs{p_n}} = \delta < \infty.
        \alignno
    \end{align}
    }
    Entonces se cumple que
    {
    \leqnomode
    \setcounter{align}{0}
    \begin{align}
        & \delta > 0;
        \alignno \\
        & f(H(w, \lambda)) \subseteq H(\mu, \lambda \delta), \text{ para todo } \lambda > 0;
        \alignno \\
        & \angle \lim_{z \to w} f(z) = \mu.
        \alignno
    \end{align}
    }

    Además, si se da la igualdad en \eqref{eq:julia2} para algún $\lambda > 0$, entonces $f$ es un automorfismo del disco.
    \end{block}
\end{frame}

\end{document}

