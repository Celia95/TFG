\usepackage[fleqn]{amsmath} %[fleqn] or [leqno] to put numbers of equations on the right or on left side
\usepackage{amssymb}
\usepackage{amsfonts}
\usepackage{anysize}
\usepackage{float}
\usepackage{amsthm}
\usepackage{amsmath}
%\usepackage{etoolbox}
\usepackage{verbatim}
\usepackage{mathtools}
\usepackage{enumerate}
\usepackage{geometry}
\usepackage[table]{xcolor}
\usepackage{color}
\usepackage{setspace}
\usepackage{multirow}
\usepackage{tikz}
\usetikzlibrary{babel, hobby, calc}
% \usepackage[disable]{todonotes} % notes not showed
\usepackage[colorinlistoftodos,prependcaption,textsize=normal,linecolor=cyan,backgroundcolor=lavander,bordercolor=black]{todonotes} % notes showed
% \newcommand{\note}[1]{}  %comment not showed
\newcommand{\note}[1]{\par {\bfseries \color{blue} #1 \par}} %comment showed

\definecolor{lila}{RGB}{153, 134, 244}
\definecolor{lavander}{RGB}{230, 225, 250}
\definecolor{deepblue}{rgb}{0,0,0.5}
\definecolor{deepred}{rgb}{0.6,0,0}
\definecolor{deepgreen}{rgb}{0,0.5,0}
\definecolor{Gray}{gray}{0.9}

% Margins
\geometry
{
	top=1.69in,
	bottom=1.56in,
	left=1.56in,
	right=1.04in
}

%\geometry
%{
%	top=1.3in,
%	bottom=1.2in,
%	left=1.2in,
%	right=0.8in
%}



%Theorem styles

%definition boldface title, romand body. Commonly used in definitions, conditions, problems and examples.
%plain boldface title, italicized body. Commonly used in theorems, lemmas, corollaries, propositions and conjectures.
%remark italicized title, romman body. Commonly used in remarks, notes, annotations, claims, cases, acknowledgments and conclusions.


%Para definir un nuevo estilo:

%\newtheoremstyle{stylename}% name of the style to be used
%  {spaceabove}% measure of space to leave above the theorem. E.g.: 3pt
%  {spacebelow}% measure of space to leave below the theorem. E.g.: 3pt
%  {bodyfont}% name of font to use in the body of the theorem
%  {indent}% measure of space to indent
%  {headfont}% name of head font
%  {headpunctuation}% punctuation between head and body
%  {headspace}% space after theorem head; " " = normal interword space
%  {headspec}% Manually specify head


\theoremstyle{plain}
\newtheorem{theorem}{Teorema}[section] %section restart the theorem counter at every new section.
\newtheorem{lemma}[theorem]{Lema} %it will use the same counter as the theorem environment.
\newtheorem{corollary}{Corolario}[theorem] %the counter of this new environment will be reset every time a new theorem environment is used.
\newtheorem{prop}[theorem]{Proposición}

\theoremstyle{definition}
\newtheorem{definition}[theorem]{Definición}
\newtheorem{example}[theorem]{Ejemplo}
%\AtEndEnvironment{example}{\null\hfill\qedsymbol}
\newtheorem{conj}[theorem]{Conjetura}

\theoremstyle{remark}
\newtheorem*{remark}{Nota}
\newtheorem*{obs}{Observación}

%%%%%%%%%%%%%%%%%%%%%%%%%%%%%%%%%%%%%%%%%%%%%%%%%%%%%%%%%%%%%%%%%%%%%%%%%%%
%
% To change symbol at the end of the proof
\renewcommand\qedsymbol{$\square$}
%\renewcommand\qedsymbol{$\blacksquare$}

%%%%%%%%%%%%%%%%%%%%%%%%%%%%%%%%%%%%%%%%%%%%%%%%%%%%%%%%%%%%%%%%%%%%%%%%%%%
% Syntax: [draw options] (center) (initial angle:final angle:radius)
\def\centerarc[#1](#2)(#3:#4:#5){
\draw[#1] ($(#2)+({#5*cos(#3)},{#5*sin(#3)})$) arc (#3:#4:#5); }

% Absolute value
\providecommand{\abs}[1]{\left\lvert#1\right\rvert}

% Norm
\providecommand{\norm}[1]{\left\lVert#1\right\rVert}

% Infinity norm
\providecommand{\norminf}[1]{\norm{#1}_{\infty}}

% Holomorphic
\providecommand{\holomorphic}[1]{\mathcal{H}(#1)}

% Holomorphic and bounded
\providecommand{\bholomorphic}[1]{\mathcal{H}^{\infty}(#1)}

% Fiber
%\providecommand{\fiber}[1]{\mathfrak{M}(#1)}

% Plus minus centering
\newcommand{\rpm}{\raisebox{.2ex}{$\scriptstyle\pm$}}

% Bar for the closure
\newcommand*\xbar[1]{%
   \hbox{%
     \vbox{%
       \hrule height 0.5pt % The actual bar
       \kern0.2ex%         % Distance between bar and symbol
       \hbox{%
         \kern-0.1em%      % Shortening on the left side
         \ensuremath{#1}%
         \kern-0.1em%      % Shortening on the right side
       }%
     }%
   }%
}

\newcommand{\complex}{\mathbb{C}}
\newcommand{\real}{\mathbb{R}}
\newcommand{\integer}{\mathbb{Z}}
\newcommand{\naturals}{\mathbb{N}}

\newcommand{\disk}{\mathbb{D}}
\newcommand{\closedisk}{\overline{\disk}}

\newcommand{\diam}{\operatorname{diam}}
\renewcommand{\Re}{\operatorname{Re}}
\renewcommand{\Im}{\operatorname{Im}}
\newcommand{\id}{\operatorname{id}}


%Fiber
\newcommand{\fiber}{\mathfrak{M}}

\makeatletter
\newcommand{\leqnomode}{\tagsleft@true} %to put numbers of equations on the left side
\newcommand{\reqnomode}{\tagsleft@false} %to put numbers of equations on the right side
\makeatother

% Change the appearance of the numbering of equations. Use with \usetagform{<tagform>
\newtagform{Alph}[\renewcommand{\theequation}{\Alph{equation}}]()
\newtagform{alph}[\renewcommand{\theequation}{\alph{equation}}]()
\newtagform{Roman}[\renewcommand{\theequation}{\roman{equation}}]()
\newtagform{roman}[\renewcommand{\theequation}{\roman{equation}}]()
\newtagform{scroman}[\renewcommand{\theequation}{\scshape\roman{equation}}][]

% Change line spacing of equations
%\setlength{\abovedisplayskip}{30pt}
%\setlength{\belowdisplayskip}{30pt}
%\setlength{\jot}{30pt}

% Change de alignment of equations
%\setlength{\mathindent}{1cm}

% New counter for aligns
\newcounter{align}[equation]
\renewcommand{\thealign}{\roman{align}} % style of the numbering
\newcommand{\alignno}{\refstepcounter{align}\tag{\thealign}}
\expandafter\let\expandafter\oldalignstar\csname align*\endcsname
\expandafter\def\csname align*\endcsname{\refstepcounter{equation}\oldalignstar}
