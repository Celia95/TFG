\usepackage[fleqn]{amsmath}
\usepackage{amssymb}
\usepackage{amsfonts}
\usepackage{anysize}
\usepackage{float}
\usepackage{amsthm}
\usepackage{verbatim}
\usepackage{pstricks}
\usepackage{pst-plot}
\usepackage{tikz}
\usetikzlibrary{babel, hobby}
%\usetikzlibrary{backgrounds}
\usetikzlibrary{arrows,calc,shapes,decorations.pathreplacing}

%Theorem styles

%definition boldface title, romand body. Commonly used in definitions, conditions, problems and examples.
%plain boldface title, italicized body. Commonly used in theorems, lemmas, corollaries, propositions and conjectures.
%remark italicized title, romman body. Commonly used in remarks, notes, annotations, claims, cases, acknowledgments and conclusions.


%Para definir un nuevo estilo:

%\newtheoremstyle{stylename}% name of the style to be used
%  {spaceabove}% measure of space to leave above the theorem. E.g.: 3pt
%  {spacebelow}% measure of space to leave below the theorem. E.g.: 3pt
%  {bodyfont}% name of font to use in the body of the theorem
%  {indent}% measure of space to indent
%  {headfont}% name of head font
%  {headpunctuation}% punctuation between head and body
%  {headspace}% space after theorem head; " " = normal interword space
%  {headspec}% Manually specify head


\theoremstyle{plain}
\newtheorem{theorem}{Teorema}[section] %section restart the theorem counter at every new section.
\newtheorem{lemma}[theorem]{Lema} %it will use the same counter as the theorem environment.
\newtheorem{corollary}{Corolario}[theorem] %the counter of this new environment will be reset every time a new theorem environment is used.
\newtheorem{prop}{Proposición}[theorem]

\theoremstyle{definition}
\newtheorem{definition}{Definición}[section]
%\newtheorem{example}{Ejemplo}[section]
\newtheorem{conj}{Conjetura}[section]

\theoremstyle{remark}
\newtheorem*{remark}{Nota}
\newtheorem*{obs}{Observación}

%%%%%%%%%%%%%%%%%%%%%%%%%%%%%%%%%%%%%%%%%%%%%%%%%%%%%%%%%%%%%%%%%%%%%%%%%%%
%
%Para cambiar el símbolo del final de la demostración: \square, \blacksquare

\renewcommand\qedsymbol{$\square$}
%\renewcommand\qedsymbol{$\blacksquare$}

%%%%%%%%%%%%%%%%%%%%%%%%%%%%%%%%%%%%%%%%%%%%%%%%%%%%%%%%%%%%%%%%%%%%%%%%%%%
%
%Valor absoluto
\providecommand{\abs}[1]{\left\lvert#1\right\rvert}

%Norma
\providecommand{\norm}[1]{\left\lVert#1\right\rVert}

%Norma infinito
\providecommand{\norminf}[1]{\norm{#1}_{\infty}}

%Holomorfa
\providecommand{\holomorphic}[1]{\mathcal{H}(#1)}

%Holomorfa y acotada
\providecommand{\bholomorphic}[1]{\mathcal{H}^{\infty}(#1)}


%Mas menos centrado
\newcommand{\rpm}{\raisebox{.2ex}{$\scriptstyle\pm$}}

%Barra para la adherencia
\newcommand*\xbar[1]{%
   \hbox{%
     \vbox{%
       \hrule height 0.5pt % The actual bar
       \kern0.2ex%         % Distance between bar and symbol
       \hbox{%
         \kern-0.1em%      % Shortening on the left side
         \ensuremath{#1}%
         \kern-0.1em%      % Shortening on the right side
       }%
     }%
   }%
}

\newcommand{\complex}{\mathbb{C}}
\newcommand{\real}{\mathbb{R}}
\newcommand{\integer}{\mathbb{Z}}
\newcommand{\naturals}{\mathbb{N}}

\newcommand{\disk}{\mathbb{D}}
\newcommand{\closedisk}{\overline{\disk}}



\newcommand{\diam}{\operatorname{diam}}
\renewcommand{\Re}{\operatorname{Re}}
\renewcommand{\Im}{\operatorname{Im}}
