%---------------------------------------------------------------------
%
%                      abstract.tex
%
%---------------------------------------------------------------------
%
% Contiene el capítulo del abstract.
%
% Se crea como un capítulo sin numeración.
%
%---------------------------------------------------------------------

\chapter{Abstract}
\cabeceraEspecial{Abstract}

This paper aims to show the behaviour of some functions of a complex variable on the boundary of the open unit disk, especially the holomorphic and harmonic functions (on the open disk). With this objective, theoretical results and their proofs are detailed and examples of different nature are provided. Currently we have computer tools which allow us to enrich the usual analytical study carried out. Therefore, part of this work consist of designing appropriate applications which allow us to visualize, analyse and illustrate some of these classic problems such as the well-known Dirichlet problem, solved by the Poisson integral. \\

\textbf{Keywords}: Poisson integral, Fatou theorem, Carathéodory theorem, Blaschke product, Banach algebra, spectrum, cluster set, angular derivative, Schwarz-Pick lemma, Julia theorem. \\


\endinput
% Variable local para emacs, para  que encuentre el fichero maestro de
% compilación y funcionen mejor algunas teclas rápidas de AucTeX
%%%
%%% Local Variables:
%%% mode: latex
%%% TeX-master: "../Tesis.tex"
%%% End:
