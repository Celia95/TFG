%---------------------------------------------------------------------
%
%                      abstract.tex
%
%---------------------------------------------------------------------
%
% Contiene el capítulo del abstract.
%
% Se crea como un capítulo sin numeración.
%
%---------------------------------------------------------------------

\chapter{Abstract}
\cabeceraEspecial{Abstract}

This paper aims to show the behavior of some functions of a complex variable on the boundary of the open unit disk, especially holomorphic and harmonic functions (on the open disk). Currently we have the necessary computer tools that allow us to enrich the analytical study carried out. Thus, part of this work will be to develop our own applications, which complement the existing ones, to analyze and illustrate some classic problems such as the well-known Dirichlet problem, solved by the Poisson integral. \\

\textbf{Keywords}: Poisson integral, Fatou theorem, Carathéodory theorem, Blaschke product, Banach algebra, spectrum, cluster set, angular derivative, Schwarz-Pick lemma, Julia theorem. \\


\endinput
% Variable local para emacs, para  que encuentre el fichero maestro de
% compilación y funcionen mejor algunas teclas rápidas de AucTeX
%%%
%%% Local Variables:
%%% mode: latex
%%% TeX-master: "../Tesis.tex"
%%% End:
