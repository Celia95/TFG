%---------------------------------------------------------------------
%
%                      resumen.tex
%
%---------------------------------------------------------------------
%
% Contiene el capítulo del resumen.
%
% Se crea como un capítulo sin numeración.
%
%---------------------------------------------------------------------

\chapter{Resumen}
\cabeceraEspecial{Resumen}

\todo[inline]{Añadir algo más de información sobre el esquema de lo que se hace.}

En este trabajo se muestra el comportamiento de algunas funciones de variable compleja a lo largo del borde del disco unidad, especialmente de las funciones holomorfas y las armónicas (en el disco). Con este objetivo se detallan resultados teóricos y sus demostraciones y se aportan ejemplos de distinta naturaleza. En la actualidad contamos con herramientas informáticas que permiten enriquecer el estudio analítico usual. Así pues, parte de este trabajo consiste en diseñar aplicaciones apropiadas que permiten visualizar, analizar e ilustrar algunos de estos problemas clásicos como el conocido problema de Dirichlet, resuelto mediante la integral de Poisson. \\

\textbf{Palabras clave}: integral de Poisson, teorema de Fatou, teorema de Carathéodory, producto de Blaschke, álgebra de Banach, espectro, conjunto de valores adherentes, derivada angular, lema de Schwarz-Pick, teorema de Julia. \\

\endinput
% Variable local para emacs, para  que encuentre el fichero maestro de
% compilación y funcionen mejor algunas teclas rápidas de AucTeX
%%%
%%% Local Variables:
%%% mode: latex
%%% TeX-master: "../Tesis.tex"
%%% End:
