%---------------------------------------------------------------------
%
%                      resumen.tex
%
%---------------------------------------------------------------------
%
% Contiene el capítulo del resumen.
%
% Se crea como un capítulo sin numeración.
%
%---------------------------------------------------------------------

\chapter{Resumen}
\cabeceraEspecial{Resumen}

En este trabajo se pretende mostrar el comportamiento de algunas funciones de variable compleja a lo largo del borde del disco unidad, especialmente de las funciones holomorfas y las armónicas (en el disco). En la actualidad contamos con las herramientas informáticas necesarias que permiten enriquecer el estudio analítico realizado. Así pues, parte de este trabajo será realizar aplicaciones propias, que complementen las existentes, para analizar e ilustrar algunos problemas clásicos como el conocido problema de Dirichlet, resuelto mediante la integral de Poisson. \\

\textbf{Palabras clave}: integral de Poisson, teorema de Fatou, teorema de Carathéodory, producto de Blaschke, álgebra de Banach, espectro, conjunto de valores adherentes, derivada angular, lema de Schwarz-Pick, teorema de Julia. \\

\endinput
% Variable local para emacs, para  que encuentre el fichero maestro de
% compilación y funcionen mejor algunas teclas rápidas de AucTeX
%%%
%%% Local Variables:
%%% mode: latex
%%% TeX-master: "../Tesis.tex"
%%% End:
