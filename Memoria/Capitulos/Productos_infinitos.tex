\chapter{Productos infinitos. Productos de Blaschke}

\section{Productos infinitos}
\todo[inline, color=yellow]{¿Su producto infinito se define como el límite de los productos parciales $u_1 u_2 \cdots u_N$ o  $u_{n_0} u_{n_0+1} \cdots u_N$?}

\begin{definition}
    Sea $\{u_n\}$ una sucesión de números complejos que puede anularse en una cantidad finita de elementos hasta un cierto $n_0 \in \naturals$, a partir del cual todos los elementos serán no nulos. Su producto infinito se define como el límite de los productos parciales $u_1 u_2 \cdots u_N$ cuando $N$ tiende a infinito:
\begin{equation*}
    \prod_{n=1}^{\infty} u_n = \lim_{N \to \infty} \prod_{n=1}^{N} u_n.
\end{equation*}

Además, decimos que el producto converge cuando el límite existe y no es cero. En otro caso, se dice que el producto diverge. \\
\end{definition}

\todo[inline]{Añadir resultados de convergencia y convergencia absoluta en productos infinitos}

\begin{prop}
    Sea $\{u_n\}$ una sucesión de números complejos no nulos. Si $\lim u_n =1$ y la serie
    \begin{equation*}
        \sum_{n=1}^{\infty} \log u_n
    \end{equation*}
    converge absolutamente, es decir, $ \sum_{n=1}^{\infty} \abs{\log u_n}$ converge, entonces el producto infinito
    \begin{equation*}
        \prod_{n=1}^{\infty} u_n
    \end{equation*}
    converge absolutamente.
\end{prop}

\begin{proof}
    Si $n$ es suficientemente grande, entonces $u_n$ puede escribirse como $u_n = 1 - \alpha_n$, donde $\abs{\alpha_n} < 1$, y entonces podemos definir $\log{u_n}$ como $\log{(1 - \alpha_n)}$. Por hipótesis, se sigue que la serie
    \begin{equation*}
        \sum_{n=1}^{\infty} \log u_n = \sum_{n=1}^{\infty} \log{(1 - \alpha_n)}
    \end{equation*}
    converge. Así que las sumas parciales
    \begin{equation*}
        \sum_{n=1}^{N} \log u_n
    \end{equation*}
    tienen límite. Como la función exponencial es continua, podemos exponenciar las sumas parciales y vemos que
    \begin{equation*}
        \prod_{n=1}^{\infty} u_n = \lim_{N \to \infty} \prod_{n=1}^{N} u_n
    \end{equation*}
    existe.
\end{proof}

\begin{lemma}
    \label{th:convergencia}
    Sea $\{\alpha_n\}$ una sucesión de números complejos tales que $\alpha_n \not = 1$ para todo $n$. Supongamos que
    \begin{equation*}
        \sum_{n=1}^{\infty} \abs{\alpha_n}
    \end{equation*}
    converge. Entonces
    \begin{equation*}
        \prod_{n=1}^{\infty} (1 - \alpha_n)
    \end{equation*}
    converge absolutamente.
\end{lemma}

\begin{proof}
    Para una cantidad finita $n$, tenemos que $\abs{\alpha_n} < \frac{1}{2}$, así que $\log{(1 - \alpha_n)}$ está definido por la serie usual, y para alguna constante $C$, tenemos
    \begin{equation*}
        \abs{\log{(1 - \alpha_n)}} \leq C \abs{\alpha_n}.
    \end{equation*}
    Por tanto, el producto converge absolutamente por definición y utilizando la hipótesis de que $\sum_{n=1}^{\infty} \abs{\alpha_n}$ converge.
\end{proof}

Los dos Lemas anteriores nos permiten relacionar la convergencia de un producto infinito con la convergencia de las series de potencias, lo cual nos va a ser de gran ayuda ya que estas últimas son mucho más fáciles de manejar. \\

\todo[inline]{Añadir comentario sobre la equivalencia de las tres situaciones.}

\section{Productos de Blaschke}

\begin{prop}
    Sea $\{\alpha_n\}$ una sucesión en el disco unidad $\disk$ tal que $\alpha_n \not = 0$ para todo $n$ y $\sum_{n=1}^{\infty} (1 - \abs{\alpha_n})$ converge. Entonces el producto de Blaschke
    \begin{equation*}
        B(z) = \prod_{n=1}^{\infty} \frac{\alpha_n - z}{1 - \xbar{\alpha_n}z} \dfrac{\abs{\alpha_n}}{\alpha_n}
    \end{equation*}
    converge uniformemente en $\disk$. La función $B(z)$ define una función holomorfa en el disco unidad que tiene los mismos ceros que $\alpha_n$. Además $\abs{B(z)} \leq 1$ y $\abs{B(e^{i \theta})} = 1$ en casi todo punto.
\end{prop}

\begin{proof}
    Sea
    \begin{equation*}
        b_n (z) = \frac{\alpha_n - z}{1 - \xbar{\alpha_n}z} \dfrac{\abs{\alpha_n}}{\alpha_n}.
    \end{equation*}

    Por el Lema \ref{th:convergencia}, sabemos que $\prod_{n=1}^{\infty} b_n$ converge uniformemente en $\disk$ a una función holomorfa que tiene los mismos ceros que $\{\alpha_n\}$ si y solo si $\sum_{n=1}^{\infty} \abs{1 - b_n}$ converge uniformemente en todo subconjunto compacto de $\disk$. Para $\abs{\alpha_n} < 1$ y $\abs{z} \leq r < 1$, se tiene

    \begin{equation*}
        \begin{split}
            \abs{1 - b_n(z)} & = \abs{1 + \dfrac{z - \alpha_n}{1 - \xbar{\alpha_n}z} \dfrac{\abs{\alpha_n}}{\alpha_n}} = \abs{\dfrac{(1 - \xbar{\alpha_n}z) \alpha_n + (z - \alpha_n) \abs{\alpha_n}}{(1 - \xbar{\alpha_n}z) \alpha_n}} = \\
                             & = \abs{ \dfrac{\alpha_n + z \abs{\alpha_n}}{\alpha_n (1 - \xbar{\alpha_n}z)}} (1 - \abs{\alpha_n}) \leq \dfrac{1 + r}{1 - r} (1 - \abs{\alpha_n}).
        \end{split}
    \end{equation*}
    pues si $\abs{\alpha_n} < 1$ y $\abs{z} \leq r$ se verifican $\abs{\alpha_n + z \abs{\alpha_n}} \leq 1+r$, y $\abs{1- \xbar{\alpha_n}z} \geq 1-\abs{\xbar{\alpha_n}}\abs{z} \geq 1-r$. \\

    Entonces para $\abs{z} \leq r < 1$, se tiene
    \begin{equation*}
        \sum_{n=1}^{\infty} \abs{1 - b_n(z)} \leq \dfrac{1 + r}{1 - r} \sum_{n=1}^{\infty} (1 - \abs{\alpha_n}),
    \end{equation*}
    y la serie $\sum_{n=1}^{\infty} \abs{1 - b_n(z)}$ converge absoluta y uniformemente en el disco cerrado de radio $r$. Por lo que $B(z) = \prod_{n=1}^{\infty} b_n$ converge uniformemente en $\disk$. \\

    Como $b_n(z)$ son funciones holomorfas en $\disk$ y su producto infinito converge uniformemente en los compactos de $\disk$, se tiene que $B(z)$ define una función holomorfa en el disco unidad. \\ %$f$ tiene los mismos ceros que $\alpha_n$ por la definición de producto infinito.

    Además, se cumple $\abs{B(z)} \leq 1$ por la caracterización de los automorfismos del disco unidad ya que los términos $\dfrac{\alpha_n - z}{1 - \xbar{\alpha_n}z}$ definen un automorfismo del disco unidad que lleva el disco abierto en el disco abierto y el borde en el borde. \\

    Así pues, aplicando el Teorema de Fatou, $B(z)$ tiene límites radiales $\abs{B(e^{i\theta})} \leq 1$ en casi todo punto. Para ver que $\abs{B(e^{i \theta})} = 1$ en casi todo punto, tomemos $B_n(z) = \prod_{k=1}^{n} b_k(z)$ el producto parcial. Entonces, $\frac{B}{B_n}$ es otro producto de Blaschke y
    \begin{equation*}
        \abs{\frac{B(0)}{B_n(0)}} \leq \frac{1}{2 \pi} \int_{0}^{2 \pi} \abs{\frac{B(e^{i \theta})}{B_n(e^{i \theta})}} d \theta = \frac{1}{2 \pi}  \int_{0}^{2 \pi} \abs{ B(e^{i \theta})} d \theta.
    \end{equation*}

    Tomando $n \to \infty$, obtenemos
    \begin{equation*}
         \frac{1}{2 \pi}  \int_{0}^{2 \pi} \abs{ B(e^{i \theta})} d \theta = 1,
    \end{equation*}
    y, por consiguiente, $\abs{ B(e^{i \theta})} = 1$ en casi todo punto
\end{proof}

Podemos enunciar un resultado similar al anterior si permitimos que $\alpha_n$ se anule $m$ veces, lo que significa que $B$ se anulará en los mismos puntos que $\alpha_n$. Haciendo uso de la Proposición anterior, lo que atañe a la convergencia y los límites radiales inmediato y no precisa demostración. \\

\begin{corollary}
    Sea $\{\alpha_n\}$ una sucesión en el disco unidad $\disk$ tal que $\sum_{n=1}^{\infty} (1 - \abs{\alpha_n})$ converge. Sea $m$ el número de $\alpha_n$ iguales a cero. Entonces el producto de Blaschke
\begin{equation*}
    B(z) = z^m \prod_{\abs{\alpha_n} \not = 0} \frac{\alpha_n - z}{1 - \xbar{\alpha_n}z} \dfrac{\abs{\alpha_n}}{\alpha_n}
\end{equation*}
converge uniformemente en $\disk$. La función $B(z)$ define una función holomorfa en el disco unidad que tiene los mismos ceros que $\alpha_n$. Además $\abs{B(z)} \leq 1$ y $\abs{B(e^{i \theta})} = 1$ en casi todo punto.
\end{corollary}
