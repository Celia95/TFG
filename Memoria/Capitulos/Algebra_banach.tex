\chapter{$\bholomorphic{\disk}$ como álgebra de Banach}

En este capítulo vamos a trabajar con $\bholomorphic{\disk}$ como el álgebra de las funciones holomorfas acotadas en el disco unidad. \\

\begin{definition}
    Un espacio vectorial complejo se denomina espacio de Banach si es normado y completo.
\end{definition}

\medskip
$\bholomorphic{\disk}$ es un espacio vectorial complejo, que dotado con la norma infinito
\begin{equation*}
    \norminf{f} = \sup_{\abs{z} < 1} \abs{f(z)},
\end{equation*}
es un espacio vectorial normado y completo sobre $\complex$. Atendiendo a la definición anterior, decimos que  $(\bholomorphic{\disk}, \norminf{\cdot})$ es un espacio de Banach. \\

\begin{definition}
    Decimos que $B$ es un álgebra de Banach si es un espacio de Banach con un álgebra asociada tal que la multiplicación satisface:
    \begin{equation*}
        \forall x, y \in B: \, \norm{x \cdot y} \leq \norm{x} \cdot \norm{y}.
    \end{equation*}
\end{definition}

\medskip
También podemos ver $\bholomorphic{\disk}$ como un álgebra. En efecto, si $f, g \in \bholomorphic{\disk}$ y $\alpha, \beta \in \complex$, entonces
\begin{equation*}
    \begin{split}
        & \alpha f + \beta g \in \bholomorphic{\disk} \\
        & fg \in \bholomorphic{\disk}.
    \end{split}
\end{equation*}

Así, $\bholomorphic{\disk}$ es un álgebra de Banach conmutativa (con la función constante 1 como elemento unidad) puesto que es un álgebra conmutativa y un espacio de Banach cuya norma asociada cumple la siguiente propiedad:
\begin{equation*}
    \forall f, g \in \bholomorphic{\disk}: \, \norminf{f \cdot g} \leq \norminf{f} \cdot \norminf{g}.
\end{equation*} \\

\begin{definition}
    Sea $B$ un espacio de Banach. Consideramos $B^*$ el espacio de las aplicaciones $\varphi: B \rightarrow \complex$ continuas. $B^*$ es un espacio vectorial y tiene una norma natural dada por:
    \begin{equation*}
        \norm{\varphi} = \sup_{\norm{x} \leq 1} \abs{\varphi(x)}.
    \end{equation*}
    Con esta norma, $B^*$ es un espacio de Banach al que llamamos espacio conjugado de $B$.
\end{definition}

\medskip
Además de la topología inducida por la norma en el espacio conjugado $B^*$, vamos a considerar otra topología denominada topología débil-* en $B^*$. Está definida de la siguiente manera. Sea $\varphi_0 \in B^*$ y tomemos una cantidad finita de elementos $x_1, \dots x_n \in B$ y $\varepsilon > 0$. Sea
\begin{equation*}
U = \{ \varphi \in B^*: \abs{\varphi(x_k) - \varphi_0 (x_k)} < \varepsilon, k = 1, \dots, n\}.
\end{equation*}
un entorno $\varphi_0$. Un abierto de esta topología será, por tanto, cualquier unión de tales entornos $U$.

Es la topología más débil de $B^*$ tal que todas las funciones $\varphi \rightarrow \varphi(x)$ son continuas de $B^*$ en $\complex$, con $x \in B$. Esta topología se denota por $\sigma(B^*, B)$. % La topología débil-* es la más débil de $B^*$, es decir, es aquella que tiene el menor número de abiertos.

\begin{obs}
    El disco unidad cerrado de $B^*$ es compacto en la topología débil-*.
\end{obs}
\bigskip

Recordemos que $\phi : \bholomorphic{\disk} \rightarrow \complex$ es un homomorfismo de álgebras si para todos $f, g \in \bholomorphic{\disk}$ y $\alpha, \beta \in \complex$ se cumple:
\begin{equation}
    \begin{split}
        & \phi (\alpha f + \beta g) = \alpha \phi(f) + \beta \phi(g) \\
        & \phi(f \cdot g) = \phi(f) \cdot \phi(g).
    \end{split}
\end{equation}

El espectro de $\bholomorphic{\disk}$, denotado por $\fiber = \fiber (\bholomorphic{\disk})$, es el espacio de los homomorfismos $\phi: \bholomorphic{\disk} \rightarrow \complex$ no nulos. Observamos que tales homomorfismos verifican que $\norm{\phi} = 1$ y son continuos.\\

$\fiber$ es un subconjunto del espacio conjugado $\bholomorphic{\disk}^*$ y, de hecho, está contenido en el disco unidad de $\bholomorphic{\disk}^*$. Además, $\fiber$ es cerrado en la topología débil estrella en $B^*$. \\

Como el disco unidad en $\bholomorphic{\disk}^*$ equipado con la topología débil-* es compacto, se sigue que $\fiber$ (como subconjunto de $\bholomorphic{\disk}^*$) equipado con la topología débil estrella es un espacio Hausdorff compacto. ? \\

En este punto queremos asociar cada elemento de $x$ de $B$ con uno que estará sobre $\fiber(B)$. Para ello vamos a definir la siguiente aplicación:
\begin{equation*}
    \begin{split}
        \widehat x:  \fiber(B) & \rightarrow  \complex \\
                 \varphi \, \, \, & \mapsto  \varphi (x).
    \end{split}
\end{equation*}

Cada $\widehat x$ es una función continua en $\fiber (B)$. De hecho, por definición, la topología débil-* es la topología más débil de $\fiber (B)$ que hace que cada $\widehat x$ sea continua. Así pues, tenemos la siguiente representación a la que se le suele denominar \textbf{transformada de Gelfand}
\begin{equation*}
    x \rightarrow \widehat x.
\end{equation*}
La imagen de $B$ bajo este homomorfismo es el álgebra $\widehat B = \{\widehat x: \fiber(B) \rightarrow  \complex \mid x \in B\}$. \\

Lo que hemos comentado puede aplicarse a $\bholomorphic{\disk}$. Así tenemos la siguiente aplicación
\begin{equation*}
    \begin{split}
        \widehat f:  \fiber & \rightarrow  \complex \\
                \phi \, & \mapsto  \phi (f),
    \end{split}
\end{equation*}
con $x \in \bholomorphic{\disk}$, que da lugar a la representación $f \rightarrow \widehat f$. Vamos a poder interpretar $\bholomorphic{\disk}$ como el álgebra de las funciones continuas en el espacio compacto de los ideales maximales $\fiber$. Hay que hablar más de la relación de $\fiber$ y los ideales maximales. \\

Quizá es mejor hablar primero de la transformada de Gelfand y luego introducir la topología débil-* en $\fiber$. \\

Al espacio $\fiber$ se le suele llamar el espacio de ideales maximales de $\bholomorphic{\disk}$. Para cada $\phi \in \fiber$, el kernel de $\phi$ es un ideal maximal en el álgebra $\bholomorphic{\disk}$. Recíprocamente, todo ideal maximal en $\bholomorphic{\disk}$ se corresponde con el núcleo de un homomorfismo en $fiber$. Vamos a estudiar la estructura de este espacio. \\

Los únicos homomorfismos complejos evidentes de $\bholomorphic{\disk}$ son las evaluaciones
\begin{equation*}
    \delta_z (f) = f(z).
\end{equation*}

Hablar más de las evaluaciones.

Existe una aplicación continua que lleva $\fiber$ en el disco unidad cerrado. Si denotamos por $\id$ la función identidad de $\disk$,
\begin{equation*}
    \id(z) = z, z \in \disk,
\end{equation*}
la proyección que buscamos lleva los homomorfismos $\phi \in \fiber$ en su correspondiente valor en la función $\id$. Así pues, la aplicación que nos interesa es $\widehat \id$. Para evitar confusiones, vamos a introducir una notación alternativa para referirnos a la función $\widehat \id$. Si $\phi \in \fiber$,
\begin{equation}
    \label{proyeccion}
    \begin{split}
        \pi: \fiber & \rightarrow \closedisk \\
            \phi \, & \mapsto  \phi (\id).
    \end{split}
\end{equation}

\begin{theorem}
    La aplicación $\pi: \fiber \rightarrow \closedisk$ definida por \ref{proyeccion} es continua. $\pi$ es inyectiva sobre el disco abierto $\disk$ y $\pi^{-1}$ aplica homeomorficamente $\disk$ sobre un abierto de $\fiber$.
\end{theorem}

En esta prueba llamo $\lambda$ a los puntos del disco y $z$ a la variable de la función $f$.

\begin{proof}
$\pi$ es continua por definición. Veamos que $\pi$ lleva $\fiber$ en el disco cerrado. En efecto, ya hemos observado antes que cada punto del disco abierto $\disk$ está en la imagen de $\pi$ puesto que $\pi (\phi_\lambda) = \lambda$. Como $\fiber$ es un conjunto compacto que contiene a $\disk$, y la imagen de un compacto por una aplicación continua es también un compacto, entonces $\pi(\fiber)$ es compacto. Así pues, como $\pi(\fiber)$ es un conjunto compacto que contiene a $\disk$, contiene todo el disco cerrado $\closedisk$. \\

Veamos ahora que $\pi$ es inyectiva sobre el disco. Para ello supongamos que $\abs{\lambda} < 1$ y $\pi (\phi) = \phi (\id) = \lambda$, con $\phi \in \fiber$. Si $f(\lambda) = 0$, entonces $f(z) = (z - \lambda) g(z)$ y
\begin{equation*}
    \phi(f) = \phi(z - \lambda) \phi(f) = 0 \cdot \phi(f) = 0.
\end{equation*}

Si $f(\lambda) = c$, entonces $f(z) = c + g(z)$, con $g(z) = 0$ y
\begin{equation*}
    \phi(f) = \phi(c) + \phi(g) = c + 0 = c.
\end{equation*}
Por lo tanto, $\phi(f) = f(\lambda)$ para toda $f \in \bholomorphic{\disk}$, es decir, $\phi$ es la evaluación en $\lambda$. Esto prueba que $\pi$ es inyectiva sobre los puntos del disco unidad $\disk$. \\

Si tomamos $\Delta = \pi^{-1} (\disk) = \{\phi_z : z \in \disk\}$, entonces $\pi$ lleva $\Delta$ homeomorficamente en el disco $\disk$ ya que la topología de $\Delta$ es la topología débil definida por las aplicaciones $\widehat f$ y la topología de $\disk$ es la topología débil definida por las aplicaciones $f \in \bholomorphic{\disk}$. ?\\
\end{proof}

Si $\abs{\alpha} = 1$, decimos que $\pi^{-1} (\alpha)$ es la fibra de $\fiber$ sobre $\alpha$ y lo denotamos por $\fiber_\alpha$:
\begin{equation*}
    \fiber_\alpha = \pi^{-1} (\alpha) = \{\phi \in \fiber : \phi (\id) = \alpha\}.
\end{equation*}

La fibra $\fiber_\alpha$ es un conjunto cerrado de $\fiber$. Intuitivamente, los elementos de $\fiber_\alpha$ son los homomorfismos complejos de $\fiber$ que se comportan como la ``evaluación en $\alpha$'', es decir, los homomorfismos $\phi \in \bholomorphic{\disk}$ que llevan cada $f \in \bholomorphic{\disk}$ en algo parecido al valor límite $f(z)$ cuando $z$ se aproxima a $\alpha$. Vamos a ver esto con más detalle a continuación. \\

\begin{comment}
Existe una correspondencia uno a uno entre los homomorfismos $\phi: \bholomorphic{\disk} \rightarrow \complex$ y los ideales maximales $M$ en el álgebra $\bholomorphic{\disk}$. Esta correspondencia está definida por $M = \ker (\phi)$. Cada ideal maximal $M$ es cerrado, así que cada homomorfismo $\phi$ es continuo:
\begin{equation*}
    \abs{\phi (x)} \leq \norm{x}.
\end{equation*}
\end{comment}

%Observemos que la imagen de toda función constante por cualquier elemento del espectro es ella misma. Además, la identidad es una función de $\bholomorphic{\disk}$ de norma 1. \\

\begin{theorem}
    \label{result1}
    Sea $f$ una función en $\bholomorphic{\disk}$ y sea $\alpha$ un punto del círculo unidad. Sea $\{z_n\}$ una sucesión de puntos en el disco unidad $\disk$ que converge a $\alpha$, y supongamos que el límite
    \begin{equation*}
        \zeta = \lim_{n \rightarrow \infty} f(z_n)
    \end{equation*}
    existe. Entonces existe un homomorfismo complejo $\phi$ en la fibra $\fiber_\alpha$ tal que $\phi(f) = \zeta$.
\end{theorem}

\begin{proof}
    Sea $J = \{h\in \bholomorphic{\disk} : \lim_{n \rightarrow \infty} h(z_n) = 0 \}$ un ideal propio en $\bholomorphic{\disk}$. $J$ está contenido en un ideal maximal $M$, esto es, existe un homomorfismo complejo $\phi$ de  $\bholomorphic{\disk}$ del que $M$ es el núcleo. En particular, $\phi(h) = 0$ para todo $h \in J$. Las funciones $(z - \alpha)$ y $(f - \zeta)$ están ambas en $J$. Entonces, $\phi(z) = \alpha$ y $\phi(f) = \zeta$. Por lo tanto $\phi$ es el homomorfismo buscado. \\ %Aquí se usa que \phi(c)=c, siendo c una constante.
\end{proof}

\begin{theorem}
    Sea $f$ una función en $\bholomorphic{\disk}$ y sea $\alpha$ un punto del círculo unidad. La función $\widehat f$ es constante en la fibra $\fiber_\alpha$ si y solo si $f$ se puede extender con continuidad a $\disk \cup \{\alpha\}.$
\end{theorem}

\begin{proof}
    Supongamos primero que $f$ se puede extender con continuidad a $\disk \cup \{ \alpha\}$. Esto significa que existe un número complejo $\zeta$ tal que $\lim_{z_n \rightarrow \alpha} f(z_n) = \zeta$ para toda sucesión $\{z_n\}$ en $\disk$ que converge a $\alpha$. Queremos mostrar que $\widehat f$ vale constantemente $\zeta$ en la fibra $\fiber_\alpha$, es decir, $\phi(f) = \zeta$ para todo $\phi \in \fiber_\alpha$. \\

    Podemos suponer que $\zeta = 0$. Sea $h(z) = \frac{1}{2} (1 + z \alpha^{-1})$, así que $h(\alpha) = 1$ y $\abs{h} < 1$ en cualquier otro lugar dentro del disco unidad cerrado. Como $f$ es continua en $\alpha$ y toma el valor $0$, es fácil ver que $(1 - h^n) f$ converge uniformemente a $f$ cuando $n \rightarrow \infty$. Si $\phi$ es un homomorfismo complejo de $\bholomorphic{\disk}$ que yace en la fibra $\fiber_\alpha$, es decir, $\phi (z) = \alpha$, entonces $\phi (h) = 1$. Por lo tanto, $\phi [(1 - h^n)f] = 0$, y, como $\phi$ es continua, $\phi (f) = 0$. Así, $\widehat f$ es la función idénticamente nula en $\fiber_\alpha$. \\

    %Si $\widehat f$ vale constantemente $\zeta$ en la fibra $\fiber_\alpha$, entonces el Teorema \ref{result1} implica que $f(z) \rightarrow \zeta$ cuando $z_n \rightarrow \alpha$. Si definimos $f (\alpha) = \zeta$, entonces $f$ se puede extender con continuidad a $\disk \cup \{ \alpha \}$.

    Si $\widehat f$ es constante en la fibra $\fiber_\alpha$, entonces el Teorema \ref{result1} muestra directamente que $f$ se puede extender con continuidad a $\disk \cup \{ \alpha \}$. \\
\end{proof}

%\bigskip
%La discusión anterior muestra que $\disk \in \fiber (\bholomorphic{\disk})$. Entonces podemos definir la corona de $\bholomorphic{\disk}$ como $\fiber (\bholomorphic{\disk}) \setminus \disk$.

Podemos ahora hacernos algunas preguntas de carácter topológico sobre el espacio de ideales maximales de $\bholomorphic{\disk}$. Las evaluaciones punto a punto llevan el disco unidad abierto en un conjunto abierto $\Delta$ de $\fiber$. El resto de homomorfismos yacen en las fibras $\fiber_\alpha$ y son límites de los puntos de $\Delta$. La cuestión que nos planteamos es la siguiente: ¿son esos homomorfismos realmente límites de $\phi_z$ en la topología de $\fiber$? En otras palabras, ¿es el disco $\disk$ denso en $\fiber$? A esta pregunta se le ha denominado El Problema de la Corona. \\

\begin{theorem}[Teorema de la Corona]
    El problema de la corona es equivalente a:
     Sean $f_1, \dots, f_n \in \bholomorphic{\disk}$ y $\delta > 0$ tales que para cada $z \in \disk$ se tiene
\begin{equation*}
    \abs{f_1(z)} + \cdots + \abs{f_n(z)} \geq \delta,
\end{equation*}
     entonces existen $g_1, \dots, g_n \in \bholomorphic{\disk}$ tales que $f_1 g_1 + \cdots + f_n g_n = 1$.

    %Si $\phi \in \fiber \exists (z_\alpha) \subset \disk / \forall g \in \bholomorphic{\disk} \lim_\alpha g(z_\alpha) = \widehat g(\phi) = \phi (g)$, siendo $g(z_\alpha) = \delta_{z_\alpha} (g)$.
\end{theorem}

\begin{proof}
Supongamos que $\disk$ es denso. Sean $f_1, \dots, f_n \in \bholomorphic{\disk}$ y $\delta > 0$ tales que para cada $z \in \disk$ se tiene
\begin{equation*}
    \abs{f_1(z)} + \cdots + \abs{f_n(z)} \geq \delta.
\end{equation*}

Si la función constante $1$ no se pudiera escribir de la forma $f_1 g_1 + \cdots + f_n g_n$, con $g_1, \dots, g_n \in \bholomorphic{\disk}$, tomemos $\phi \in \fiber$ no nulo tal que el ideal maximal $\ker \phi$ contiene al ideal propio generado por $f_1, \dots, f_n$.

Como $\disk$ es denso en $\fiber$ para $w^*$, existe una red $\{z_\alpha \} \subset \disk$ que tiende $w^*$ a $\phi$. En particular, para cada $f_j$ se tiene que $\lim_\alpha f_j (z_\alpha) = \widehat{f_j} (\phi) = 0, 1 \leq j \leq n$. Esto contradice la acotación relativa a $\abs{f_1(z)} + \cdots + \abs{f_n(z)}$. \\

Recíprocamente, supongamos que $\disk$ no es denso en $\fiber$, entonces existe un elemento no nulo $\phi_0 \in \fiber$ que no está en la adherencia de $\disk$. Por definición de la topología de $\fiber$, existen funciones $f_1, \dots, f_n \in \bholomorphic{\disk}$ y $\delta > 0$ tales que $\phi_0 (f_j) = 0, j = 1, \dots, n$ y el abierto
\begin{equation*}
    \{ \phi \in \fiber : \abs{\phi (f_j)} < \delta, 1 \leq j \leq n \}
\end{equation*}
no corta a $\disk$. En particular, para cada $z \in \disk$ se cumple que
\begin{equation*}
    \abs{f_1(z)} + \cdots + \abs{f_n(z)} \geq \delta
\end{equation*}
y las funciones $f_1, \dots, f_n$ están en un ideal propio de $J \subset \bholomorphic{\disk}$ ya que $J \subset \ker \phi_0$. %Esto es porque $\phi_0 (f_j) = 0$

La afirmación de que $f_1, \dots, f_n$ están en un ideal propio es equivalente a la afirmación de que la función constante $1$ no se puede escribir de la forma $f_1 g_1 + \cdots + f_n g_n = 1$, con $g_1, \dots, g_n \in \bholomorphic{\disk}$, ya que $\phi (1) = 1$ y $\phi (f_1 g_1 + \cdots + f_n g_n) = \phi (f_1) \phi (g_1) + \cdots + \phi (f_n) \phi (g_n) = 0$. \\
\end{proof}

\begin{prop}
    Para todo $f \in \bholomorphic{\disk}$ y $\alpha$ tal que $\abs{\alpha} = 1$ se cumple que
    \begin{equation*}
        \widehat{f} (\fiber_\alpha) \subset Cl(f, \alpha).
    \end{equation*}
\end{prop}

\begin{proof}
    Sea $\phi \in \fiber_\alpha$. Veamos que existe una sucesión $\{ z_n\} \subset \disk$ tal que
    \usetagform{roman} \leqnomode
    \begin{align}
        & \lim_{n \rightarrow \infty} z_n = \alpha \\
        & \lim_{n \rightarrow \infty} f(z_n)= \widehat{f} (\phi).
    \end{align}

    Como $\disk$ es denso en $\fiber$ para $w^*$, se cumple que existe $\{z_\alpha\} \subset \disk$ tal que $\delta_{z_\alpha} \rightarrow \phi$. Es decir, para toda función $h \in \bholomorphic{\disk}$ se tiene que $h (z_\alpha) \rightarrow \widehat{h} (\phi)$. En particular, para $g(z) = z$ es cierto por lo que, como $\phi \in \fiber_\alpha$, tenemos
    \begin{equation*}
        g(z_\alpha) = z_\alpha \rightarrow \widehat{g} (\phi) = \alpha.
    \end{equation*}

    Si tomamos ahora $\{z_{\alpha_n}\}$ una subsucesión de $\{z_\alpha\}$ cumplirá que $\lim_{n \rightarrow \infty} z_n = \alpha$ y, además, $\lim_{n \rightarrow \infty} f(z_n)= \widehat{f} (\phi)$. Es decir, $\widehat{f} (\phi) \in Cl (f, \alpha)$. \\
\end{proof}
