\chapter{$\bholomorphic{\disk}$ como álgebra de Banach}
\label{cap:banach}

En este capítulo vamos a trabajar con $\bholomorphic{\disk}$ como el álgebra de las funciones holomorfas acotadas en el disco unidad. \\

\section{Álgebra de Banach}

\begin{definition}
    Un espacio vectorial complejo $X$, dotado de una norma $\norm{\cdot}$ se denomina espacio de Banach si es completo.
\end{definition}
\bigskip

En nuestro caso particular, $\bholomorphic{\disk}$ es un espacio vectorial complejo, que dotado con la norma infinito
\begin{equation*}
    \norminf{f} = \sup_{\abs{z} < 1} \abs{f(z)},
\end{equation*}
es normado y completo sobre $\complex$ puesto que el límite uniforme sobre compactos de una sucesión de funciones holomorfas es una función holomorfa. Atendiendo a la definición anterior, decimos que  $(\bholomorphic{\disk}, \norminf{\cdot})$ es un espacio de Banach. \\

\begin{definition}
    Decimos que un álgebra $B$, dotada de una norma $\norm{\cdot}$ es un álgebra de Banach si como espacio normado $(B, \norm{\cdot})$ es un espacio de Banach y, además, para el producto satisface:
    \begin{equation*}
        \forall x, y \in B: \, \norm{x \cdot y} \leq \norm{x} \cdot \norm{y}.
    \end{equation*}
\end{definition}
\bigskip

De nuevo, podemos ver $\bholomorphic{\disk}$ como un álgebra, con las operaciones naturales. En efecto, si $f, g \in \bholomorphic{\disk}$ y $\alpha, \beta \in \complex$, entonces
\begin{equation*}
    \begin{split}
        & \alpha f + \beta g \in \bholomorphic{\disk} \\
        & fg \in \bholomorphic{\disk}.
    \end{split}
\end{equation*} \\

Así, $\bholomorphic{\disk}$ es un álgebra de Banach conmutativa (con la función constante 1 como elemento unidad) puesto que es un álgebra conmutativa y un espacio de Banach cuya norma asociada cumple la siguiente propiedad:
\begin{equation*}
    \forall f, g \in \bholomorphic{\disk}: \, \norminf{f \cdot g} \leq \norminf{f} \cdot \norminf{g}.
\end{equation*}

\section{Espacio dual de un álgebra de Banach}

\begin{definition}
    Sea $B$ un espacio de Banach complejo. Consideramos $B^*$ el espacio de las aplicaciones $\varphi: B \to \complex$ lineales y continuas. $B^*$ es un espacio vectorial y tiene una norma natural dada por:
    \begin{equation*}
        \norm{\varphi} = \sup_{\norm{x} \leq 1} \abs{\varphi(x)}.
    \end{equation*}
    Con esta norma, $B^*$ es un espacio de Banach al que llamamos espacio dual de $B$. \\
\end{definition}

Además de la topología inducida por la norma en el espacio dual $B^*$, vamos a considerar otra topología denominada topología débil-* en $B^*$ que está definida de la siguiente manera. Sea $\varphi_0 \in B^*$, y tomemos una cantidad finita de elementos $x_1, \dots x_n \in B$ y $\varepsilon > 0$. Los entornos de $\varphi_0$ serán los conjuntos que contienen uno de la forma $U$, donde
\begin{equation*}
U = \{ \varphi \in B^*: \abs{\varphi(x_k) - \varphi_0 (x_k)} < \varepsilon, k = 1, \dots, n\}.
\end{equation*}
Un abierto de esta topología será, por tanto, cualquier unión de tales entornos $U$. \\

Esta topología se denota por $\sigma(B^*, B)$. Es la topología más débil de $B^*$ tal que todas las funciones $\varphi \to \varphi(x)$ son continuas de $B^*$ en $\complex$, con $x \in B$. \\ % La topología débil-* es la más débil de $B^*$, es decir, es aquella que tiene el menor número de abiertos.

Un resultado importante de análisis funcional que usaremos en nuestro desarrollo es el Teorema de Alaouglu, que establece la compacidad de la bola unidad cerrada de cualquier espacio dual, cuando se considera dotado de la topología débil-*. Obsérvese que en dimensión infinita, la bola unidad no es compacta en ningún espacio para la topología dada por la norma. \\

\begin{theorem}[de Alaouglu]
    La bola unidad cerrada de $B^*$ es compacto en la topología débil-*.
\end{theorem}

\section{El espectro de un álgebra}

\todo[inline]{¿Debería hacer como antes: introducirlo primero para $B$ y luego seguir para $\bholomorphic{\disk}$?}

Recordemos que $\phi : \bholomorphic{\disk} \to \complex$ es un homomorfismo de álgebras si para todos $f, g \in \bholomorphic{\disk}$ y $\alpha, \beta \in \complex$ se cumple:
\begin{equation}
    \begin{split}
        & \phi (\alpha f + \beta g) = \alpha \phi(f) + \beta \phi(g) \\
        & \phi(f \cdot g) = \phi(f) \cdot \phi(g).
    \end{split}
\end{equation}

El espectro de $\bholomorphic{\disk}$, denotado por $\fiber = \fiber (\bholomorphic{\disk})$, es el espacio de los homomorfismos $\phi: \bholomorphic{\disk} \to \complex$ no nulos. Observamos que tales homomorfismos verifican que son continuos y $\norm{\phi} = 1$ ya que $\phi(1) = 1$. \\

Tal y como hemos construido (/definido) $\fiber$, es un subconjunto del espacio dual $\bholomorphic{\disk}^*$. De hecho, está contenido en la bola unidad de $\bholomorphic{\disk}^*$ que, dotada con la topología débil-*, es un compacto, por lo que $\fiber$ como subconjunto de $\bholomorphic{\disk}^*$ dotado con dicha topología es un espacio Hausdorff compacto. Además, $\fiber$ es cerrado en $\bholomorphic{\disk}^*$ dotado con la topología débil-*. \\

Llegados a este punto queremos asociar cada elemento $x$ de $B$ con una función continua sobre $\fiber(B)$. Para ello vamos a definir la siguiente aplicación
\begin{equation*}
    \begin{split}
        \widehat x:  \fiber(B) & \to  \complex \\
              \varphi \, \, \, & \mapsto  \varphi (x).
    \end{split}
\end{equation*}

Si dotamos a $\fiber(B)$ con la topología débil-*, tenemos que cada $\widehat x$ es una función continua en $\fiber(B)$. Más aún, por definición, la topología débil-* es la topología más débil de $\fiber(B)$ que hace que cada $\widehat x$ sea continua. Así pues, tenemos la siguiente representación a la que se le suele denominar \textbf{transformada de Gelfand}
\begin{equation*}
    x \to \widehat x.
\end{equation*}

La imagen de $B$ bajo este homomorfismo es el álgebra $\widehat B$ de las funciones continuas sobre $\fiber(B)$ que toman valores complejos. Es decir,
\begin{equation*}
    \widehat B = \{\widehat x: \fiber(B) \to  \complex \mid x \in B\}.
\end{equation*}

\medskip

En particular, para el álgebra $\bholomorphic{\disk}$ la construcción se describe de la siguiente manera. Tenemos la aplicación
\begin{equation*}
    \begin{split}
        \widehat f:  \fiber & \to  \complex \\
                    \phi \, & \mapsto  \phi (f),
    \end{split}
\end{equation*}
para cada $f \in \bholomorphic{\disk}$, donde cada $\widehat f$ es continua sobre el $\fiber$ si dotamos a $\fiber$ con la topología débil-*. Esto da lugar a la representación $f \to \widehat f$. De esta manera, vamos a poder interpretar $\bholomorphic{\disk}$ como el álgebra de las funciones continuas en el espacio compacto $\fiber$. \\

Al espacio $\fiber$ se le suele denominar espacio de ideales maximales de $\bholomorphic{\disk}$. Para cada $\phi \in \fiber$, el núcleo de $\phi$ es un ideal maximal del álgebra $\bholomorphic{\disk}$. Recíprocamente, todo ideal maximal en $\bholomorphic{\disk}$ se corresponde con el núcleo de un homomorfismo en $\fiber$. Más adelante estudiaremos la estructura de este espacio. \\

\begin{comment}
Existe una correspondencia uno a uno entre los homomorfismos $\phi: \bholomorphic{\disk} \to \complex$ y los ideales maximales $M$ en el álgebra $\bholomorphic{\disk}$. Esta correspondencia está definida por $M = \ker (\phi)$. Cada ideal maximal $M$ es cerrado, así que cada homomorfismo $\phi$ es continuo:
\begin{equation*}
    \abs{\phi (x)} \leq \norm{x}.
\end{equation*}
\end{comment}

\todo[inline]{Hablar más de la relación de $\fiber$ con los ideales maximales.}

En principio, los únicos homomorfismos complejos que se pueden identificar claramente son las evaluaciones punto a punto del disco abierto $\disk$. Si $z \in \disk$,
\begin{equation*}
    \begin{split}
        \delta_z: \bholomorphic{\disk} & \to  \complex \\
                   f \, \, \, \, \, \, & \mapsto  f (z).
    \end{split}
\end{equation*}

Así pues, las evaluaciones en puntos del disco abierto son elementos del espectro y cumplen $\abs{\delta_z} = 1$ para todo $z \in \disk$. \\

\todo[inline]{Hablar más de las evaluaciones. ¿Qué más hay que decir?}

\section{La proyección del espectro sobre el disco}

Existe una proyección natural continua que lleva $\fiber$ en el disco unidad cerrado. Si denotamos por $\id$ la función identidad de $\disk$,
\begin{equation*}
    \id(z) = z, z \in \disk,
\end{equation*}
la aplicación que buscamos lleva los homomorfismos $\phi \in \fiber$ en su correspondiente valor en la función $\id$. Así pues, la aplicación que nos interesa es $\widehat \id$. Para evitar confusiones, vamos a introducir una notación alternativa para referirnos a la función $\widehat \id$. Si $\phi \in \fiber$,
\begin{equation}
    \label{eq:proyeccion}
    \begin{split}
        \pi: \fiber & \to \closedisk \\
            \phi \, & \mapsto  \phi (\id).
    \end{split}
\end{equation}

Nótese que la función identidad tiene norma 1 y cada $\phi \in \fiber$ también, por lo que $\abs{\pi(\phi)} \leq 1$. Es decir, la imagen de $\pi$ está contenida en el disco unidad cerrado. \\

\begin{theorem}
    La aplicación $\pi: \fiber \to \closedisk$ definida por \eqref{eq:proyeccion} es continua. $\pi$ es inyectiva sobre el disco abierto $\disk$ y $\pi^{-1}$ aplica homeomórficamente $\disk$ sobre un abierto de $\fiber$.
\end{theorem}

\begin{proof}
$\pi$ es continua por definición. Veamos que $\pi$ lleva $\fiber$ en el disco cerrado. En efecto, ya hemos observado antes que cada punto del disco abierto $\disk$ está en la imagen de $\pi$ puesto que $\pi (\delta_\lambda) = \lambda$. Como $\fiber$ es un conjunto compacto que contiene a $\disk$, y la imagen de un compacto por una aplicación continua es también un compacto, entonces $\pi(\fiber)$ es compacto. Así pues, como $\pi(\fiber)$ es un conjunto compacto que contiene a $\disk$, contiene todo el disco cerrado $\closedisk$. \\
% Al ser continua $\pi$, el espectro compacto y el disco abierto está en la imagen de $\pi$, y este es denso en el disco cerrado, la imagen debe ser todo el disco cerrado.

Veamos ahora que $\pi$ es inyectiva sobre el disco. Para ello supongamos que $\abs{\lambda} < 1$ y $\pi (\phi) = \phi (\id) = \lambda$, con $\phi \in \fiber$. Si $f(\lambda) = 0$, entonces $f(z) = (z - \lambda) g(z)$ y
\begin{equation*}
    \phi(f) = \phi(z - \lambda) \phi(f) = 0 \cdot \phi(f) = 0.
\end{equation*}

Si $f(\lambda) = c$, entonces $f(z) = c + g(z)$, con $g(z) = 0$ y
\begin{equation*}
    \phi(f) = \phi(c) + \phi(g) = c + 0 = c.
\end{equation*}
Por lo tanto, $\phi(f) = f(\lambda)$ para toda $f \in \bholomorphic{\disk}$, es decir, $\phi$ es la evaluación en $\lambda$. Esto prueba que $\pi$ es inyectiva sobre los puntos del disco unidad $\disk$. \\

\todo[inline]{Falta ver que $\pi^{-1}$ aplica homeomórficamente $\disk$ sobre un abierto de $\fiber$: ?}

Si tomamos $\Delta = \pi^{-1} (\disk) = \{\phi_z : z \in \disk\}$, entonces $\pi$ lleva $\Delta$ homeomórficamente en el disco $\disk$ ya que la topología de $\Delta$ es la topología débil definida por las aplicaciones $\widehat f$ y la topología de $\disk$ es la topología débil definida por las aplicaciones $f \in \bholomorphic{\disk}$. \\
\end{proof}

\begin{definition}
    \label{def:fibra}
    Si $\alpha \in \closedisk$, decimos que $\pi^{-1} (\alpha)$ es la fibra de $\fiber$ sobre $\alpha$ y lo denotamos por $\fiber_\alpha$,
    \begin{equation*}
        \fiber_\alpha = \pi^{-1} (\alpha) = \{\phi \in \fiber : \phi (\id) = \alpha\}.
    \end{equation*}
\end{definition}

Si $z \in \disk$, la fibra de $\fiber$ sobre $z$ coincide con la evaluación en $z$, es decir,
\begin{equation*}
    \fiber_z = \{ \delta_z\}.
\end{equation*}

\todo[inline]{La definición anterior hace que podamos descomponer el espectro como unión disjunta de sus fibras. ¿Cómo hacemos esto?}

\todo[inline]{Observemos que la imagen de toda función constante por cualquier elemento del espectro es ella misma. Además, la identidad es una función de $\bholomorphic{\disk}$ de norma 1.}

\todo[inline]{En base a la Definición \ref{def:fibra}, es sencillo ver que las fibras sobre los puntos del disco abierto tienen un único elemento; mientras que sobre los números del borde del disco la fibra tiene muchos.}

La fibra $\fiber_\alpha$ es un conjunto cerrado de $\fiber$. Intuitivamente, los elementos de $\fiber_\alpha$ son los homomorfismos complejos de $\fiber$ que se comportan como la ``evaluación en $\alpha$'', es decir, los homomorfismos $\phi \in \bholomorphic{\disk}$ que llevan cada $f \in \bholomorphic{\disk}$ en algo parecido al valor límite $f(z)$ cuando $z$ se aproxima a $\alpha$. Vamos a ver esto con más detalle a continuación. \\

A partir de esto, es evidente que para cualquier función $f$ que pueda extenderse con continuidad al disco cerrado, la función $\widehat f$ es constante en cada fibra $\fiber_\alpha$ puesto que tal $f$ es el límite uniforme de polinomios en $z$. De hecho, la continuidad de $f$ en cualquier punto de la frontera implica que $\widehat f$ es constante en la fibra $\fiber_\alpha$. \\

\begin{theorem}
    \label{th:result1}
    Sea $f$ una función en $\bholomorphic{\disk}$ y sea $\alpha$ un punto del círculo unidad. Sea $\{z_n\}$ una sucesión de puntos en el disco unidad $\disk$ que converge a $\alpha$, y supongamos que el límite
    \begin{equation*}
        \zeta = \lim_{n \to \infty} f(z_n)
    \end{equation*}
    existe. Entonces existe un homomorfismo complejo $\phi$ en la fibra $\fiber_\alpha$ tal que $\phi(f) = \zeta$.
\end{theorem}

\begin{proof}
    Sea $J = \{h\in \bholomorphic{\disk} : \lim_{n \to \infty} h(z_n) = 0 \}$ un ideal propio en $\bholomorphic{\disk}$. $J$ está contenido en un ideal maximal $M$, esto es, existe un homomorfismo complejo $\phi$ de  $\bholomorphic{\disk}$ del que $M$ es el núcleo. En particular, $\phi(h) = 0$ para todo $h \in J$. Las funciones $(z - \alpha)$ y $(f - \zeta)$ están ambas en $J$. Entonces, $\phi(z) = \alpha$ y $\phi(f) = \zeta$. Por lo tanto $\phi$ es el homomorfismo buscado. \\ % Aquí se usa que $\phi(c) = c$, siendo $c$ una constante.
\end{proof}

\begin{theorem}
    Sea $f$ una función en $\bholomorphic{\disk}$ y sea $\alpha$ un punto del círculo unidad. La función $\widehat f$ es constante en la fibra $\fiber_\alpha$ si y solo si $f$ se puede extender con continuidad a $\disk \cup \{\alpha\}.$
\end{theorem}

\begin{proof}
    Supongamos primero que $f$ se puede extender con continuidad a $\disk \cup \{ \alpha\}$. Esto significa que existe un número complejo $\zeta$ tal que $\lim_{z_n \to \alpha} f(z_n) = \zeta$ para toda sucesión $\{z_n\}$ en $\disk$ que converge a $\alpha$. Queremos mostrar que $\widehat f$ vale constantemente $\zeta$ en la fibra $\fiber_\alpha$, es decir, $\phi(f) = \zeta$ para todo $\phi \in \fiber_\alpha$. \\

    Podemos suponer que $\zeta = 0$. Sea $h(z) = \frac{1}{2} (1 + z \alpha^{-1})$, así que $h(\alpha) = 1$ y $\abs{h} < 1$ en cualquier otro lugar dentro del disco unidad cerrado. Como $f$ es continua en $\alpha$ y toma el valor $0$, es fácil ver que $(1 - h^n) f$ converge uniformemente a $f$ cuando $n \to \infty$. Si $\phi$ es un homomorfismo complejo de $\bholomorphic{\disk}$ que yace en la fibra $\fiber_\alpha$, es decir, $\phi (z) = \alpha$, entonces $\phi (h) = 1$. Por lo tanto, $\phi [(1 - h^n)f] = 0$, y, como $\phi$ es continua, $\phi (f) = 0$. Así, $\widehat f$ es la función idénticamente nula en $\fiber_\alpha$. \\

    % Si $\widehat f$ vale constantemente $\zeta$ en la fibra $\fiber_\alpha$, entonces el Teorema \ref{th:result1} implica que $f(z) \to \zeta$ cuando $z_n \to \alpha$. Si definimos $f (\alpha) = \zeta$, entonces $f$ se puede extender con continuidad a $\disk \cup \{ \alpha \}$.

    Si $\widehat f$ es constante en la fibra $\fiber_\alpha$, entonces el Teorema \ref{th:result1} muestra directamente que $f$ se puede extender con continuidad a $\disk \cup \{ \alpha \}$. \\
\end{proof}

% La discusión anterior muestra que $\disk \in \fiber (\bholomorphic{\disk})$. Entonces podemos definir la corona de $\bholomorphic{\disk}$ como $\fiber (\bholomorphic{\disk}) \setminus \disk$.

Podemos ahora hacernos algunas preguntas de carácter topológico sobre el espacio de ideales maximales de $\bholomorphic{\disk}$. Las evaluaciones punto a punto llevan el disco unidad abierto en un conjunto abierto $\Delta$ de $\fiber$. El resto de homomorfismos yacen en las fibras $\fiber_\alpha$. La cuestión que nos planteamos es la siguiente: ¿son esos homomorfismos realmente límites de $\delta_z$ en la topología de $\fiber$? En otras palabras, ¿es el disco $\disk$ denso en $\fiber$? A esta
pregunta se le ha denominado El Problema de la Corona. A continuación vamos a dar una formulación algebraica equivalente.\\

\begin{theorem}[Teorema de la Corona]
    El problema de la corona es equivalente a: \\

    Sean $f_1, \dots, f_n \in \bholomorphic{\disk}$ y $\delta > 0$ tales que para cada $z \in \disk$ se tiene
\begin{equation*}
    \abs{f_1(z)} + \cdots + \abs{f_n(z)} \geq \delta,
\end{equation*}
     entonces existen $g_1, \dots, g_n \in \bholomorphic{\disk}$ tales que $f_1 g_1 + \cdots + f_n g_n = 1$.

    % Si $\phi \in \fiber \exists (z_\alpha) \subset \disk / \forall g \in \bholomorphic{\disk} \lim_\alpha g(z_\alpha) = \widehat g(\phi) = \phi (g)$, siendo $g(z_\alpha) = \delta_{z_\alpha} (g)$.
\end{theorem}

\begin{proof}
Supongamos que $\disk$ es denso. Sean $f_1, \dots, f_n \in \bholomorphic{\disk}$ y $\delta > 0$ tales que para cada $z \in \disk$ se tiene
\begin{equation*}
    \abs{f_1(z)} + \cdots + \abs{f_n(z)} \geq \delta.
\end{equation*}

Si la función constante $1$ no se pudiera escribir de la forma $f_1 g_1 + \cdots + f_n g_n$, con $g_1, \dots, g_n \in \bholomorphic{\disk}$, tomemos $\phi \in \fiber$ no nulo tal que el ideal maximal $\ker \phi$ contiene al ideal propio generado por $f_1, \dots, f_n$. \\

Como $\disk$ es denso en $\fiber$ con la topología débil-*, existe una red $\{z_\alpha \} \subset \disk$ que tiende a $\phi$. En particular, para cada $f_j$ se tiene que $\lim_\alpha f_j (z_\alpha) = \widehat{f_j} (\phi) = 0, 1 \leq j \leq n$. Esto contradice la acotación relativa a $\abs{f_1(z)} + \cdots + \abs{f_n(z)}$. \\

Recíprocamente, supongamos que $\disk$ no es denso en $\fiber$, entonces existe un elemento no nulo $\phi_0 \in \fiber$ que no está en la adherencia de $\disk$. Por definición de la topología de $\fiber$, existen funciones $f_1, \dots, f_n \in \bholomorphic{\disk}$ y $\delta > 0$ tales que $\phi_0 (f_j) = 0, j = 1, \dots, n$ y el abierto
\begin{equation*}
    \{ \phi \in \fiber : \abs{\phi (f_j)} < \delta, 1 \leq j \leq n \}
\end{equation*}
no corta a $\disk$. En particular, para cada $z \in \disk$ se cumple que
\begin{equation*}
    \abs{f_1(z)} + \cdots + \abs{f_n(z)} \geq \delta
\end{equation*}
y las funciones $f_1, \dots, f_n$ están en un ideal propio de $J \subset \bholomorphic{\disk}$ ya que $J \subset \ker \phi_0$. \\ % Esto es porque $\phi_0 (f_j) = 0$

La afirmación de que $f_1, \dots, f_n$ están en un ideal propio es equivalente a la afirmación de que la función constante $1$ no se puede escribir de la forma $f_1 g_1 + \cdots + f_n g_n = 1$, con $g_1, \dots, g_n \in \bholomorphic{\disk}$, ya que $\phi (1) = 1$ y $\phi (f_1 g_1 + \cdots + f_n g_n) = \phi (f_1) \phi (g_1) + \cdots + \phi (f_n) \phi (g_n) = 0$. \\
\end{proof}

\begin{definition}[Cluster set y cluster value]
    Dados $f \in \bholomorphic{\disk}$ y $\alpha \in \closedisk$, definimos el ``cluster set'' de $f$ en $\alpha$ como
    \begin{equation*}
        Cl(f, \alpha) = \{\zeta : \exists \{z_n\} \in \disk, \lim_{n \to \infty} z_n = \alpha, \lim_{n \to \infty} f(z_n) = \zeta \}
    \end{equation*}

    Así, cada número $\zeta \in Cl(f, \alpha)$ se denomina un ``cluster value'' de $f$ en $\alpha$.
\end{definition}

\medskip

Si nos restringimos a $z \in \disk$, entonces $Cl(f, z) = \{\delta_z(f)\}$. \\

Teniendo en cuenta esta definición, otro enunciado alternativo al Teorema \ref{th:result1} sería el siguiente: \\

\begin{theorem}
    Sea $f \in \bholomorphic{\disk}$ y $\alpha \in \partial \disk$. Si $\zeta \in Cl(f, \alpha)$, entonces existe un homomorfismo complejo $\phi$ en la fibra $\fiber_\alpha$ tal que $\phi(f) = \zeta$, es decir, $\zeta \in \widehat f (\fiber_\alpha)$. \\
\end{theorem}

Por lo tanto, si $\abs{\alpha} = 1$, el contenido $Cl(f, \alpha) \subset \widehat{f} (\fiber_\alpha)$ se cumple. Veamos que también tenemos el recíproco. \\

\begin{prop}
    Para todo $f \in \bholomorphic{\disk}$ y $\alpha$ tal que $\abs{\alpha} = 1$ se cumple que
    \begin{equation*}
        \widehat{f} (\fiber_\alpha) \subset Cl(f, \alpha).
    \end{equation*}
\end{prop}

\begin{proof}
    Sea $\phi \in \fiber_\alpha$. Veamos que existe una sucesión $\{ z_n\} \subset \disk$ tal que
    {
    \leqnomode
    \setlength{\jot}{10pt}
    \setlength{\mathindent}{30pt}
    \begin{align}
        & \lim_{n \to \infty} z_n = \alpha;
        \alignno \label{eq:property1} \\
        & \lim_{n \to \infty} f(z_n)= \widehat{f} (\phi).
        \alignno \label{eq:property2}
    \end{align}
    }

    Como $\Delta$ es denso en $\fiber$ con la topología débil-*, se cumple que existe $\{z_\alpha\} \subset \disk$ tal que $\delta_{z_\alpha} \to \phi$. Es decir, para toda función $h \in \bholomorphic{\disk}$ se tiene que $h (z_\alpha) \to \widehat{h} (\phi)$. En particular, para la función identidad es cierto por lo que, como $\phi \in \fiber_\alpha$, tenemos
    \begin{equation*}
        \widehat \id(\phi) = \pi(\id) = \alpha.
    \end{equation*}

    Podemos elegir una sucesión $(i_n)$ tal que se cumplan al mismo tiempo $\lim z_{i_n}  = \alpha$ y $\lim f(z_{i_n}) = \widehat f(\phi)$. Es decir, si tomamos ahora $\{z_{\alpha_n}\}$ una subsucesión de $\{z_\alpha\}$ cumplirá que $\lim_{n \to \infty} z_n = \alpha$ y, además, $\lim_{n \to \infty} f(z_n)= \widehat{f} (\phi)$. Es decir, $\widehat{f} (\phi) \in Cl (f, \alpha)$. \\
\end{proof}

\todo[inline]{Quizá viene bien extender un poco lo de los productos de Blaschke.}

\begin{definition}
    Se dice que una función $f \in \bholomorphic{\disk}$ es una función interna si $\abs{f(e^{i \theta})} = 1$ en casi todo punto.
\end{definition}

Los productos de Blaschke de los que hemos hablado en el Capítulo tal son un clásico ejemplo de función interna. Además de este tipo de funciones, la función $\exp{\left(\frac{z + 1}{z - 1}\right)}$, que introdujimos en el Ejemplo \ref{ex:exp} es también una función interna. \\

\todo[inline]{Esto quizás es mejor ponerlo después de la definición de cluster set. No sé.}

Sea $f \in \bholomorphic{\disk}$ y sea $\alpha \in \partial \disk$. El ``cluster set'' de $f$ en $\alpha$ viene dado por
\begin{equation}
    Cl(f, \alpha) = \bigcap_{r>0} \xbar{f(\disk \cap D(\alpha, r))}.
\end{equation}

Esto es, $\zeta \in Cl(f, \alpha)$ si y solo si existe una sucesión $z_n$ en $\disk$ que tiende a $\alpha$ tal que $f(z_n)$ tiende a $\zeta$. El ``cluster set'' es un conjunto compacto, no vacío y conexo. Además, contiene un único punto si y solo si $f$ es continua en $\disk \cap \{\alpha\}$. Si $f$ es holomorfa en $\alpha$, y no constante, entonces $Cl(f, \alpha) = f(\alpha)$. \\

\begin{theorem}
    Sea $f$ una función interna en el disco abierto $\disk$ y sea $\alpha \in \partial \disk$ una singularidad de $f(z)$. Entonces $Cl(f, \alpha) = \closedisk$.
\end{theorem}

El ``cluster set'' se comporta bien cuando tratamos con funciones holomorfas, como hemos visto. Sin embargo, vamos a ver a continuación que no es así en el caso de las funciones no holomorfas.

\begin{example}
    Sea $h: \disk \to \disk$ dada por
    \begin{equation*}
        h(z) = -z \frac{\xbar{z} -1}{z - 1}.
    \end{equation*}

    %$h$ es una involución biyectiva de $\disk$ en $\disk$. $h$ tiene una extensión continua a $\closedisk \setminus \{1\}$ (con el valor constantemente 1).
    Entonces el ``cluster set'' $Cl(h,1)$ de $h$ en $1$ se corresponde con la circunferencia unidad $\partial \disk$.
\end{example}

\begin{proof}
    El ``cluster set'' de $h$ en $1$ viene dado por
    \begin{equation*}
        Cl(h,1) = \bigcap_{n=1}^{\infty} \xbar{h(D_n)},
    \end{equation*}
    donde $D_n = \{z \in \disk : \abs{z-1} \leq \frac{1}{n}\}$. $Cl(h,1)$ es un conjunto compacto, conexo y no vacío. Ahora bien, $\lim_{x \to 1, 0 < x < 1} h(x) = -1$ y $\lim_{\theta \to 0} h(e^{i \theta}) = 1$. \\

    Como $\mu \in Cl(h,1)$ si y solo si $\xbar{\mu} \in Cl(h,1)$, pues $h(\xbar{z}) = \xbar{h(z)}$, y $\abs{h(z)} = \abs{z} \to 1$ si $z \to 1$, concluimos que $Cl(h,1) = \partial \disk$.
\end{proof}
