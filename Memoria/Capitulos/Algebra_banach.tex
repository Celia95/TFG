\chapter{$\bholomorphic{\disk}$ como álgebra de Banach}

En este capítulo vamos a trabajar con $\bholomorphic{\disk}$ como el álgebra de las funciones holomorfas acotadas en el disco unidad. \\

\begin{definition}
    Un espacio vectorial complejo se denomina espacio de Banach si es normado y completo.
\end{definition}

\medskip
$\bholomorphic{\disk}$ es un espacio vectorial complejo, que dotado con la norma infinito
\begin{equation*}
    \norminf{f} = \sup_{z \in \disk} \abs{f(z)},
\end{equation*}
es un espacio vectorial normado y completo sobre $\complex$. Atendiendo a la definición anterior, decimos que  $(\bholomorphic{\disk}, \norminf{\cdot})$ es un espacio de Banach. \\

\begin{definition}
    Decimos que $B$ es un álgebra de Banach si es un espacio de Banach con un álgebra asociada tal que la multiplicación satisface:
    \begin{equation*}
        \forall x, y \in B: \, \norm{x \cdot y} \leq \norm{x} \cdot \norm{y}.
    \end{equation*}
\end{definition}

\medskip
También podemos ver $\bholomorphic{\disk}$ como un álgebra. En efecto, si $f, g \in \bholomorphic{\disk}$ y $\alpha, \beta \in \complex$, entonces
\begin{equation*}
    \begin{split}
        & \alpha f + \beta g \in \bholomorphic{\disk} \\
        & fg \in \bholomorphic{\disk}.
    \end{split}
\end{equation*}

Así, $\bholomorphic{\disk}$ es un álgebra de Banach conmutativa (con la función constante 1 como elemento unidad) puesto que es un álgebra conmutativa y un espacio de Banach cuya norma asociada cumple la siguiente propiedad:
\begin{equation*}
    \forall f, g \in \bholomorphic{\disk}: \, \norminf{f \cdot g} \leq \norminf{f} \cdot \norminf{g}.
\end{equation*} \\

\begin{definition}
    Sea $B$ un espacio de Banach. Consideramos $B^*$ el espacio de las aplicaciones $\varphi: B \rightarrow \complex$ continuas. $B^*$ es un espacio vectorial y tiene una norma natural dada por:
    \begin{equation*}
        \norm{\varphi} = \sup_{\norm{x} \leq 1} \abs{\varphi(x)}.
    \end{equation*}
    Con esta norma, $B^*$ es un espacio de Banach al que llamamos espacio conjugado de $B$.
\end{definition}

\medskip
Además de la topología inducida por la norma en el espacio conjugado $B^*$, vamos a considerar otra topología denominada topología débil-* en $B^*$. Está definida de la siguiente manera. Sea $\varphi_0 \in B^*$ y tomemos una cantidad finita de elementos $x_1, \dots x_n \in B$ y $\varepsilon > 0$. Sea
\begin{equation*}
U = \{ \varphi \in B^*: \abs{\varphi(x_k) - \varphi_0 (x_k)} < \varepsilon, k = 1, \dots, n\}.
\end{equation*}
un entorno $\varphi_0$. Un abierto de esta topología será, por tanto, cualquier unión de tales entornos $U$.

Es la topología más débil de $B^*$ tal que todas las funciones $\varphi \rightarrow \varphi(x)$ son continuas de $B^*$ en $\complex$, con $x \in B$. Esta topología se denota por $\sigma(B^*, B)$. % La topología débil-* es la más débil de $B^*$, es decir, es aquella que tiene el menor número de abiertos.

\begin{obs}
    El disco unidad cerrado de $B^*$ es compacto en la topología débil-*.
\end{obs}
\bigskip

Recordemos que $\phi : \bholomorphic{\disk} \rightarrow \complex$ es un homomorfismo de álgebras si para todos $f, g \in \bholomorphic{\disk}$ y $\alpha, \beta \in \complex$ se cumple:
\begin{equation}
    \begin{split}
        & \phi (\alpha f + \beta g) = \alpha \phi(f) + \beta \phi(g) \\
        & \phi(f \cdot g) = \phi(f) \cdot \phi(g).
    \end{split}
\end{equation}

El espectro de $\bholomorphic{\disk}$, denotado por $\fiber = \fiber (\bholomorphic{\disk})$, es el espacio de los homomorfismos $\phi: \bholomorphic{\disk} \rightarrow \complex$ no nulos. Observamos que tales homomorfismos verifican que $\norm{\phi} = 1$ y son continuos.\\

$\fiber$ es un subconjunto del espacio conjugado $\bholomorphic{\disk}^*$ y, de hecho, está contenido en el disco unidad de $\bholomorphic{\disk}^*$. Además, $\fiber$ es cerrado en la topología débil estrella en $B^*$. \\

Como el disco unidad en $\bholomorphic{\disk}^*$ equipado con la topología débil-* es compacto, se sigue que $\fiber$ (como subconjunto de $\bholomorphic{\disk}^*$) equipado con la topología débil estrella es un espacio Hausdorff compacto. ? \\

Ahora queremos asociar cada elemento de $x$ de $B$ con uno que estará sobre $\fiber(B)$ al que denotaremos por $\hat x$. Para ello vamos a definir la siguiente aplicación
\begin{equation*}
    \hat x (\varphi) = \varphi (x), \text{ donde } x \in B, \varphi \in \fiber (B).
\end{equation*}

Cada $\hat x$ es una función continua en $\fiber (B)$. De hecho, por definición, la topología débil-* es la topología más débil de $\fiber (B)$ que hace que cada $\hat x$ sea continua. Así pues, tenemos la siguiente representación
\begin{equation*}
    x \rightarrow \hat x
\end{equation*}
que va de $B$ en $\hat B$, el álgebra de las aplicaciones continuas que van de $\fiber(B)$ en $\complex$. A esto se le suele denominar transformada de Gelfand.




\bigskip

Observemos que la imagen de toda función constante por cualquier elemento del espectro es ella misma. Además, la identidad es una función de $\bholomorphic{\disk}$ de norma 1. \\


\begin{comment}
Sabemos que existe una correspondencia uno a uno entre los homomorfismos $\phi$ de $\bholomorphic{\disk}$ en el álgebra de los números complejos y los ideales maximales $M$ en el álgebra $\bholomorphic{\disk}$. Esta correspondencia está definida por $M = \ker (\phi)$. Cada ideal maximal $M$ es cerrado, así que cada homomorfismo $\phi$ es continuo:
\begin{equation*}
    \abs{\phi (x)} \leq \norm{x}.
\end{equation*}
\end{comment}


\begin{theorem}
    \label{result1}
    Sea $f$ una función en $\bholomorphic{\disk}$ y sea $\alpha$ un punto del círculo unidad. Sea $\{\lambda_n\}$ una sucesión de puntos en el disco unidad $\disk$ que converge a $\alpha$, y supongamos que el límite
    \begin{equation*}
        \zeta = \lim_{n \rightarrow \infty} f(\lambda_n)
    \end{equation*}
    existe. Entonces existe un homomorfismo complejo $\phi$ en la fibra $\fiber_\alpha$ tal que $\phi(f) = \zeta$.
\end{theorem}

\begin{proof}
    Sea $J = \{h\in \bholomorphic{\disk} : \lim_{n \rightarrow \infty} h(z_n) = 0 \}$ un ideal propio en $\bholomorphic{\disk}$. $J$ está contenido en un ideal maximal $M$, esto es, existe un homomorfismo complejo $\phi$ de  $\bholomorphic{\disk}$ del que $M$ es el núcleo. En particular, $\phi(h) = 0$ para todo $h \in J$. Las funciones $(z - \alpha)$ y $(f - \zeta)$ están ambas en $J$.
        Entonces, $\phi(z) = \alpha$ y $\phi(f) = \zeta$. Por lo tanto $\phi$ es el homomorfismo buscado. %Aquí se usa que \phi(c)=c, siendo c una constante.
\end{proof}

\begin{theorem}
    Sea $f$ una función en $\bholomorphic{\disk}$ y sea $\alpha$ un punto del círculo unidad. La función $\hat f$ es constante en la fibra $\fiber_\alpha$ si y solo si $f$ se puede extender con continuidad a $\disk \cup \{\alpha\}.$
\end{theorem}

\begin{proof}
    Supongamos primero que $f$ se puede extender con continuidad a $\disk \cup \{ \alpha\}$. Esto significa que existe un número complejo $\zeta$ tal que $\lim_{\lambda_n \rightarrow \alpha} f(\lambda_n) = \zeta$ para toda sucesión $\{\lambda_n\}$ en $\disk$ que converge a $\alpha$. Queremos mostrar que $\hat f$ vale constantemente $\zeta$ en la fibra $\fiber_\alpha$, es decir, $\phi(f) = \zeta$ para todo $\phi \in \fiber_\alpha$.

    Podemos suponer que $\zeta = 0$. Sea $h(\lambda) = \frac{1}{2} (1 + \lambda \alpha^{-1})$, así que $h(\alpha) = 1$ y $\abs{h} < 1$ en cualquier otro lugar dentro del disco unidad cerrado. Como $f$ es continua en $\alpha$ y toma el valor $0$, es fácil ver que $(1 - h^n) f$ converge uniformemente a $f$ cuando $n \rightarrow \infty$. Si $\phi$ es un homomorfismo complejo de $\bholomorphic{\disk}$ que yace en la fibra $\fiber_\alpha$, es decir, $\phi (z) = \alpha$, entonces $\phi (h) = 1$. Por lo tanto, $\phi [(1 - h^n)f] = 0$, y, como $\phi$ es continua, $\phi (f) = 0$. Así, $\hat f$ es la función idénticamente nula en $\fiber_\alpha$. \\

    %Si $\hat f$ vale constantemente $\zeta$ en la fibra $\fiber_\alpha$, entonces el Teorema \ref{result1} implica que $f (\lambda) \rightarrow \zeta$ cuando $\lambda_n \rightarrow \alpha$. Si definimos $f (\alpha) = \zeta$, entonces $f$ se puede extender con continuidad a $\disk \cup \{ \alpha \}$.

    Si $\hat f$ es constante en la fibra $\fiber_\alpha$, entonces el Teorema \ref{result1} muestra directamente que $f$ se puede extender con continuidad a $\disk \cup \{ \alpha \}$.
\end{proof}

\bigskip
%La discusión anterior muestra que $\disk \in \fiber (\bholomorphic{\disk})$. Entonces podemos definir la corona de $\bholomorphic{\disk}$ como $\fiber (\bholomorphic{\disk}) \setminus \disk$.

Podemos ahora hacernos algunas preguntas de carácter topológico sobre el espacio de ideales maximales de $\bholomorphic{\disk}$. Las evaluaciones punto a punto llevan el disco unidad abierto en un conjunto abierto $\Delta$ de $\fiber$. El resto de homomorfismos yacen en las fibras $\fiber_\alpha$ y son límites de los puntos de $\Delta$. La cuestión que nos planteamos es la siguiente: ¿son esos homomorfismos realmente límites de $\phi_\lambda$ en la topología de $\fiber$? En otras palabras, ¿es el disco $\disk$ denso en $\fiber$? A esta pregunta se le ha denominado El Problema de la Corona.

\begin{theorem}[Teorema de la Corona]
    El problema de la corona es equivalente a:
     Sean $f_1, \dots, f_n \in \bholomorphic{\disk}$ y $\delta > 0$ tales que para cada $\lambda \in \disk$ se tiene
\begin{equation*}
    \abs{f_1(\lambda)} + \cdots + \abs{f_n(\lambda)} \geq \delta,
\end{equation*}
     entonces existen $g_1, \dots, g_n \in \bholomorphic{\disk}$ tales que $f_1 g_1 + \cdots + f_n g_n = 1$.

    %Si $\phi \in \fiber \exists (z_\alpha) \subset \disk / \forall g \in \bholomorphic{\disk} \lim_\alpha g(z_\alpha) = \hat g(\phi) = \phi (g)$, siendo $g(z_\alpha) = \delta_{z_\alpha} (g)$.
\end{theorem}

\begin{proof}
Supongamos que $\disk$ es denso. Sean $f_1, \dots, f_n \in \bholomorphic{\disk}$ y $\delta > 0$ tales que para cada $\lambda \in \disk$ se tiene
\begin{equation*}
    \abs{f_1(\lambda)} + \cdots + \abs{f_n(\lambda)} \geq \delta.
\end{equation*}

Si la función constante $1$ no se pudiera escribir de la forma $f_1 g_1 + \cdots + f_n g_n$, con $g_1, \dots, g_n \in \bholomorphic{\disk}$, tomemos $\phi \in \fiber$ no nulo tal que el ideal maximal $\ker \phi$ contiene al ideal propio generado por $f_1, \dots, f_n$.

Como $\disk$ es denso en $\fiber$ para $w^*$, existe una red $\{ \lambda_\alpha \} \subset \disk$ que tiende $w^*$ a $\phi$. En particular, para cada $f_j$ se tiene que $\lim_\alpha f_j (\lambda_\alpha) = \hat{f_j} (\phi) = 0, 1 \leq j \leq n$. Esto contradice la acotación relativa a $\abs{f_1(\lambda)} + \cdots + \abs{f_n(\lambda)}$. \\

Recíprocamente, supongamos que $\disk$ no es denso en $\fiber$, entonces existe un elemento no nulo $\phi_0 \in \fiber$ que no está en la adherencia de $\disk$. Por definición de la topología de $\fiber$, existen funciones $f_1, \dots, f_n \in \bholomorphic{\disk}$ y $\delta > 0$ tales que $\phi_0 (f_j) = 0, j = 1, \dots, n$ y el abierto
\begin{equation*}
    \{ \phi \in \fiber : \abs{\phi (f_j)} < \delta, 1 \leq j \leq n \}
\end{equation*}
no corta a $\disk$. En particular, para cada $\lambda \in \disk$ se cumple que
\begin{equation*}
    \abs{f_1(\lambda)} + \cdots + \abs{f_n(\lambda)} \geq \delta
\end{equation*}
y las funciones $f_1, \dots, f_n$ están en un ideal propio de $J \subset \bholomorphic{\disk}$ ya que $J \subset \ker \phi_0$. %Esto es porque $\phi_0 (f_j) = 0$

La afirmación de que $f_1, \dots, f_n$ están en un ideal propio es equivalente a la afirmación de que la función constante $1$ no se puede escribir de la forma $f_1 g_1 + \cdots + f_n g_n = 1$, con $g_1, \dots, g_n \in \bholomorphic{\disk}$, ya que $\phi (1) = 1$ y $\phi (f_1 g_1 + \cdots + f_n g_n) = \phi (f_1) \phi (g_1) + \cdots + \phi (f_n) \phi (g_n) = 0$.
\end{proof}

\begin{prop}
    Para todo $f \in \bholomorphic{\disk}$ y $\alpha$ tal que $\abs{\alpha} = 1$ se cumple que
    \begin{equation*}
        \hat{f} (\fiber_\alpha) \subset Cl(f, \alpha).
    \end{equation*}
\end{prop}

\begin{proof}
    Sea $\phi \in \fiber_\alpha$. Veamos que existe una sucesión $\{ z_n\} \subset \disk$ tal que
    \usetagform{roman} \leqnomode
    \begin{align}
        & \lim_{n \rightarrow \infty} z_n = \alpha \\
        & \lim_{n \rightarrow \infty} f(z_n)= \hat{f} (\phi).
    \end{align}

    Como $\disk$ es denso en $\fiber$ para $w^*$, se cumple que existe $\{z_\alpha\} \subset \disk$ tal que $\delta_{z_\alpha} \rightarrow \phi$. Es decir, para toda función $h \in \bholomorphic{\disk}$ se tiene que $h (z_\alpha) \rightarrow \hat{h} (\phi)$. En particular, para $g(z) = z$ es cierto por lo que, como $\phi \in \fiber_\alpha$, tenemos
    \begin{equation*}
        g(z_\alpha) = z_\alpha \rightarrow \hat{g} (\phi) = \alpha.
    \end{equation*}

    Si tomamos ahora $\{z_{\alpha_n}\}$ una subsucesión de $\{z_\alpha\}$ cumplirá que $\lim_{n \rightarrow \infty} z_n = \alpha$ y, además, $\lim_{n \rightarrow \infty} f(z_n)= \hat{f} (\phi)$. Es decir, $\hat{f} (\phi) \in Cl (f, \alpha)$.
\end{proof}
