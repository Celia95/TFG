\chapter{$\bholomorphic{\disk}$ como álgebra de Banach}

En este capítulo vamos a trabajar con $\bholomorphic{\disk}$ como el álgebra de las funciones holomorfas acotadas en el disco unidad.

$\bholomorphic{\disk}$ es un espacio vectorial complejo, que equipado con la norma infinito
\begin{equation*}
    \norminf{f} = \sup_{z \in \disk} \abs{f(z)},
\end{equation*}
es un espacio vectorial normado y completo sobre $\complex$. Es decir, $(\bholomorphic{\disk}, \norminf{\cdot})$ es un espacio de Banach. 

También podemos ver $\bholomorphic{\disk}$ como un álgebra. Si $f, g \in \bholomorphic{\disk}$ y $\alpha, \beta \in \complex$, entonces
\begin{equation*}
    \begin{split}
        & \alpha f + \beta g \in \bholomorphic{\disk} \\
        & fg \in \bholomorphic{\disk}.
    \end{split}
\end{equation*}

Además, es un álgebra de Banach conmutativa (con identidad) ya que es un álgebra compleja lineal conmutativa que está equipada con una norma bajo la que hay un espacio de Banach y la norma cumple la siguiente propiedad:
\begin{equation*}
    \norminf{f \cdot g} \leq \norminf{f} \cdot \norminf{g}.
\end{equation*}

Una aplicación continua $\phi : \bholomorphic{\disk} \rightarrow \complex$ es un homomorfismo de álgebra si para todo $f, g \in \bholomorphic{\disk}$ y $\alpha, \beta \in \complex$ se cumple:
\begin{equation}
    \begin{split}
        & \phi (\alpha f + \beta g) = \alpha \phi(f) + \beta \phi(g) \\
        & \phi(f \cdot g) = \phi(f) \cdot \phi(g).
    \end{split}
\end{equation}


Denotemos $\fiber = \fiber (\bholomorphic{\disk})$ el espacio de los homomorfismos complejos del álgebra $\bholomorphic{\disk}$. Los elementos de $\fiber$ son los homomorfismos $\phi$ de $\bholomorphic{\disk}$ en el álgebra de los números complejos, es decir, las aplicaciones lineales multiplicativas de $\bholomorphic{\disk}$. \\


\begin{comment}
Sabemos que existe una correspondencia uno a uno entre los homomorfismos $\phi$ de $\bholomorphic{\disk}$ en el álgebra de los números complejos y los ideales maximales $M$ en el álgebra $\bholomorphic{\disk}$. Esta correspondencia está definida por $M = \ker (\phi)$. Cada ideal maximal $M$ es cerrado, así que cada homomorfismo $\phi$ es continuo:
\begin{equation*}
    \abs{\phi (x)} \leq \norm{x}.
\end{equation*}

$\fiber$ es un subconjunto del espacio conjugado $B^*$, y de hecho está contenido en la esfera unidad de $B^*$. Además, $\fiber (B)$ es cerrado en la topología débil estrella en $B^*$.
\end{comment}


\begin{theorem}
    \label{result1}
    Sea $f$ una función en $\bholomorphic{\disk}$ y sea $\alpha$ un punto del círculo unidad. Sea $\{\lambda_n\}$ una sucesión de puntos en el disco unidad $\disk$ que converge a $\alpha$, y supongamos que el límite
    \begin{equation*}
        \zeta = \lim_{n \rightarrow \infty} f(\lambda_n)
    \end{equation*}
    existe. Entonces existe un homomorfismo complejo $\phi$ en la fibra $\fiber_\alpha$ tal que $\phi(f) = \zeta$.
\end{theorem}

\begin{proof}
    Sea $J$ el conjunto de todas las funciones $g$ de $\bholomorphic{\disk}$ tales que $\lim_{n \rightarrow \infty} g(\lambda_n) = 0.$ Entonces $J$ es un ideal propio cerrado en $\bholomorphic{\disk}$ y está contenido en un ideal maximal $M$, esto es, existe un homomorfismo complejo $\phi$ de  $\bholomorphic{\disk}$ del que $M$ es el núcleo ($\phi(g) = 0$ para todo $g \in J$). Las funciones $(z - \alpha)$ y $(f - \zeta)$ están ambas en $J$. Entonces, $\phi(z) = \alpha$ y $\phi(f) = \zeta$. Por lo tanto $\phi$ es el homomorfismo buscado.
\end{proof}

\begin{theorem}
    Sea $f$ una función en $\bholomorphic{\disk}$ y sea $\alpha$ un punto del círculo unidad. La función $\hat f$ es constante en la fibra $\fiber_\alpha$ si y solo si $f$ se puede extender con continuidad a $\disk \cup \{\alpha\}.$
\end{theorem}

\begin{proof}
    Supongamos primero que $f$ se puede extender con continuidad a $\disk \cup \{ \alpha\}$. Esto significa que existe un número complejo $\zeta$ tal que $\lim_{\lambda_n \rightarrow \alpha} f(\lambda_n) = \zeta$ para toda sucesión $\{\lambda_n\}$ en $\disk$ que converge a $\alpha$. Queremos mostrar que $\hat f$ vale constantemente $\zeta$ en la fibra $\fiber_\alpha$, es decir, $\phi(f) = \zeta$ para todo $\phi \in \fiber_\alpha$.

    Podemos asumir que $\zeta = 0$. Sea $g(\lambda) = \frac{1}{2} (1 + \lambda \alpha^{-1})$, así que $g(\alpha) = 1$ y $\abs{g} < 1$ en cualquier otro lugar dentro del disco unidad cerrado. Como $f$ es continua en $\alpha$ y toma el valor $0$, es fácil ver que $(1 - g^n) f$ converge uniformemente a $f$ cuando $n \rightarrow \infty$. Si $\phi$ es un homomorfismo complejo de $\bholomorphic{\disk}$ que yace en la fibra $\fiber_\alpha$, es decir, $\phi (z) = \alpha$, entonces $\phi (g) = 1$. Por lo tanto, $\phi [(1 - g^n)f] = 0$, y, como $\phi$ es continua, $\phi (f) = 0$. Así, $\hat f$ es la función idénticamente nula en $\fiber_\alpha$. \\

    %Si $\hat f$ vale constantemente $\zeta$ en la fibra $\fiber_\alpha$, entonces el Teorema \ref{result1} implica que $f (\lambda) \rightarrow \zeta$ cuando $\lambda_n \rightarrow \alpha$. Si definimos $f (\alpha) = \zeta$, entonces $f$ se puede extender con continuidad a $\disk \cup \{ \alpha \}$.

    Si $\hat f$ es constante en la fibra $\fiber_\alpha$, entonces el Teorema \ref{result1} muestra directamente que $f$ se puede extender con continuidad a $\disk \cup \{ \alpha \}$.

\end{proof}

\bigskip
%La discusión anterior muestra que $\disk \in \fiber (\bholomorphic{\disk})$. Entonces podemos definir la corona de $\bholomorphic{\disk}$ como $\fiber (\bholomorphic{\disk}) \setminus \disk$.

Podemos ahora hacernos algunas preguntas de carácter topológico sobre el espacio de ideales maximales de $\bholomorphic{\disk}$. Las evaluaciones punto a punto  $\phi_\lambda$ llevan el disco unidad abierto en un conjunto abierto $\Delta$ de $\fiber$. El resto de homomorfismos yacen en las fibras $\fiber_\alpha$ y son límites de los puntos de $\Delta$. La cuestión que nos planteamos es la siguiente: ¿son esos homomorfismos realmente límites de $\phi_\lambda$ en la
topología de $\fiber$? En otras palabras, ¿es el disco $\disk$ denso en $\fiber$? A esta pregunta se le ha denominado El Problema de la Corona y todavía continúa sin respuesta.


\begin{theorem}[Teorema de la Corona]
    El problema de la corona es equivalente a:
     Sean $f_1, \dots, f_n \in \bholomorphic{\disk}$ y $\delta > 0$ tales que para cada $\lambda \in \disk$ se tiene
\begin{equation*}
    \abs{f_1(\lambda)} + \cdots + \abs{f_n(\lambda)} \geq \delta,
\end{equation*}
     entonces existen $g_1, \dots, g_n \in \bholomorphic{\disk}$ tales que $f_1 g_1 + \cdots + f_n g_n = 1$.

    %Si $\phi \in \fiber \exists (z_\alpha) \subset \disk / \forall g \in \bholomorphic{\disk} \lim_\alpha g(z_\alpha) = \hat g(\phi) = \phi (g)$, siendo $g(z_\alpha) = \delta_{z_\alpha} (g)$.
\end{theorem}

\begin{proof}
Supongamos que $\disk$ es denso. Sean $f_1, \dots, f_n \in \bholomorphic{\disk}$ y $\delta > 0$ tales que para cada $\lambda \in \disk$ se tiene
\begin{equation*}
    \abs{f_1(\lambda)} + \cdots + \abs{f_n(\lambda)} \geq \delta.
\end{equation*}

Si la función constante $1$ no se pudiera escribir de la forma $f_1 g_1 + \cdots + f_n g_n$, con $g_1, \dots, g_n \in \bholomorphic{\disk}$, tomemos $\phi \in \fiber$ no nulo tal que el ideal maximal $\ker \phi$ contiene al ideal propio generado por $f_1, \dots, f_n$.

Como $\disk$ es denso en $\fiber$ para $w^*$, existe una sucesión $\{ \lambda_\alpha \} \subset \disk$ que tiende $w^*$ a $\phi$. En particular, para cada $f_j$ se tiene que $\lim_\alpha f_j (\lambda_\alpha) = \hat{f_j} (\phi) = 0, 1 \leq j \leq n$. Esto contradice la acotación relativa a $\abs{f_1(\lambda)} + \cdots + \abs{f_n(\lambda)}$. \\

Recíprocamente, supongamos que $\disk$ no es denso en $\fiber$, entonces existe un elemento no nulo $\phi_0 \in \fiber$ que no está en la adherencia de $\disk$. Por definición de la topología de $\fiber$, existen funciones $f_1, \dots, f_n \in \bholomorphic{\disk}$ y $\delta > 0$ tales que $\phi_0 (f_j) = 0, j = 1, \dots, n$ y el abierto
\begin{equation*}
    \{ \phi \in \fiber : \abs{\phi (f_j)} < \delta, 1 \leq j \leq n \}
\end{equation*}
no corta a $\disk$. En particular, para cada $\lambda \in \disk$ se cumple que
\begin{equation*}
    \abs{f_1(\lambda)} + \cdots + \abs{f_n(\lambda)} \geq \delta
\end{equation*}
y las funciones $f_1, \dots, f_n$ están en un ideal propio de $J \subset \bholomorphic{\disk}$ ya que $J \subset \ker \phi_0$. %Esto es porque $\phi_0 (f_j) = 0$

La afirmación de que $f_1, \dots, f_n$ están en un ideal propio es equivalente a la afirmación de que la función constante $1$ no se puede escribir de la forma $f_1 g_1 + \cdots + f_n g_n = 1$, con $g_1, \dots, g_n \in \bholomorphic{\disk}$, ya que $\phi (1) = 1$ y $\phi (f_1 g_1 + \cdots + f_n g_n) = \phi (f_1) \phi (g_1) + \cdots + \phi (f_n) \phi (g_n) = 0$.

%Supongamos que $\disk$ no es denso en $\fiber$, entonces existe un elemento no nulo $\phi \in \fiber$ que no está en la adherencia de $\disk$. Por definición de la topología de $\fiber$, existen funciones $h_1, \dots, h_n \in \bholomorphic{\disk}$ y $\delta > 0$ tales que el entorno de $\phi$ dado por $\{ \psi \in \fiber : \abs{\phi (h_j) - \psi(h_j)} < \delta, 1 \leq j \leq n \}$ no corta a $\disk$. Definimos $f_j = h_j + \phi(h_j)$, para $1 \leq j \leq n$. Entonces, $\phi (f_j) = 0$ (lo que significa que las funciones $f_j$ están en un ideal propio, $J \subset \bholomorphic{\disk}$, ya que $J \subset \ker \phi$) y $\{ \psi \in \fiber : \abs{\psi (f_j)} < \delta, 1 \leq j \leq n \}$ no corta a $\disk$. En particular, para cada $x \in \disk$ se cumple que $\abs{f_1(\lambda)} + \cdots + \abs{f_n(\lambda)} \geq \delta$, pero la función constante $1$ no se puede escribir de la forma $f_1 g_1 + \cdots + f_n g_n$, con $g_1, \dots, g_n \in \bholomorphic{\disk}$, ya que $\phi (1) = 1$ y $\phi (f_1 g_1 + \cdots + f_n g_n) = \phi (f_1) \phi (g_1) + \cdots + \phi (f_n) \phi (g_n) = 0$.
\end{proof}

\begin{prop}
    Para todo $f \in \bholomorphic{\disk}$ y $\alpha$ tal que $\abs{\alpha} = 1$ se cumple que
    \begin{equation*}
        \hat{f} (\fiber_\alpha) \subset Cl(f, \alpha).
    \end{equation*}
\end{prop}

\begin{proof}
    Sea $\phi \in \fiber_\alpha$. Veamos que existe una sucesión $\{ z_n\} \subset \disk$ tal que
    \usetagform{roman} \leqnomode
    \begin{align}
        & \lim_{n \rightarrow \infty} z_n = \alpha \\
        & \lim_{n \rightarrow \infty} f(z_n)= \hat{f} (\phi).
    \end{align}

    Como $\disk$ es denso en $\fiber$ para $w^*$, se cumple que existe $\{z_n\} \subset \disk$ tal que $\delta_{z_\alpha} \rightarrow \phi$. Es decir, para toda función $h \in \bholomorphic{\disk}$ se tiene que $h (z_\alpha) \rightarrow \hat{h} (\phi)$. En particular, para $g(z) = z$ es cierto por lo que, como $\phi \in \fiber_\alpha$, tenemos
    \begin{equation*}
        g(z_\alpha) = z_\alpha \rightarrow \hat{g} (\phi) = \alpha.
    \end{equation*}

    Si tomamos ahora $\{z_{\alpha_n}\}$ una subsucesión de $\{z_n\}$ cumplirá que $\lim_{n \rightarrow \infty} z_n = \alpha$ y, además, $\lim_{n \rightarrow \infty} f(z_n)= \hat{f} (\phi)$. Es decir, $\hat{f} (\phi) \in Cl (f, \alpha)$.
\end{proof}
