\chapter{Ejemplos}
\label{cap:ejemplos}

\todo[inline]{Empezar comentando cómo calcular el radio de convergencia, que los ejemplos son de series con radio de convergencia 1 y que vas a mostrar cómo difieren el comportamiento en la frontera de unos a otros.}

En esta sección vamos a estudiar el comportamiento de algunas series de potencias en el borde de su disco de convergencia.

%%%%%%%%%%%%%%%%%%%%%%%%%%%%%%%%%%%%%%%%%%%%%%%%%%%%%%%%%%%%%%%%%%%%%%%%%%%%%%%%
%EJEMPLO 1
%%%%%%%%%%%%%%%%%%%%%%%%%%%%%%%%%%%%%%%%%%%%%%%%%%%%%%%%%%%%%%%%%%%%%%%%%%%%%%%%

\begin{example}
    Mostrar que
    \begin{equation*}
        \sum_{n=0}^{\infty} z^n, \, \abs{z} < 1
    \end{equation*}
    diverge en todo punto tal que $\abs{z} = 1$.
\end{example}

\begin{proof}
Es fácil ver que $1 - z^{n+1} = (1 - z) (1+ z + z^2 + \cdots + z^n)$. Por lo tanto, si $z \neq 1$, se tiene que
\begin{equation}
    1 + z + \cdots + z^n = \frac{1 - z^{n+1}}{1-z}.
\end{equation}

Por un lado, si $\abs{z} < 1$ entonces $\lim_{n \to \infty} z^n = 0$ y la serie converge a
\begin{equation*}
    \sum_{n=0}^{\infty} z^n = \frac{1}{1 - z}.
\end{equation*}

Ahora bien, si $\abs{z} > 1$ entonces $\lim_{n \to} z^n = \infty$ y la serie diverge. Pero, ¿qué pasa cuando $\abs{z} = 1$? La serie de potencias $\sum_{n=0}^{\infty} z^n$ diverge en todos los puntos del radio de convergencia pues $\abs{z^n}$ no tiende a 0 cuando $n \to \infty$. \\

Sin embargo, $\sum_{n=0}^{\infty} z^n$ puede ser extendida a la función globalmente analítica $\frac{1}{1-z}$ en $\complex \setminus \{1\}$ gracias a una cantidad finita de prolongaciones analíticas. \\

Tomemos $a$ un punto cualquiera de $\complex \setminus \{1\}$ y conectémoslo al origen $0$ mediante la curva de Jordan $\gamma \subset \complex \setminus \{1\}$. Fijemos un punto $z_1$ en $\gamma$ que cumpla $\abs{z} < 1$. $\sum_{n=0}^{\infty} z^n$ puede ser extendida analíticamente en $z_1$ de la siguiente forma:
\begin{equation*}
    \begin{split}
        \frac{1}{1-z} & = \frac{1}{1 - z_1 - (z - z_1)} = \frac{1}{1 - z_1} \frac{1}{1 - \frac{z - z_1}{1 - z_1}} = \frac{1}{1 - z_1} \sum_{n=0}^{\infty} \left(\frac{z - z_1}{1 - z_1} \right)^n = \\
                      & = \sum_{n=0}^{\infty}  \frac{1}{(1 - z_1)^{n+1}} (z - z_1)^n, \abs{z - z_1} < \abs{1 - z_1}.
    \end{split}
\end{equation*}

De nuevo, tomemos $z_2$ en $\gamma$ tal que $\abs{z_2 - z_1} < \abs{1 - z_1}$ y $\abs{z_2} \geq 1$. Podemos extender la serie de potencias a $z_2$ de la misma forma:
\begin{equation*}
    \frac{1}{1-z} = \sum_{n=0}^{\infty} \frac{1}{(1 - z_2)^{n+1}} (z - z_2)^n , \abs{z - z_2} < \abs{1 - z_2}.
\end{equation*}

Después de un número finito de iteraciones, dado que la curva es un conjunto compacto, alcanzaremos el punto $a$ y tendremos
\begin{equation*}
    \frac{1}{1-z} = \sum_{n=0}^{\infty} \frac{1}{(1 - a)^{n+1}} (z - a)^n , \abs{z - a} < \abs{1 - a}.
\end{equation*}

Así, decimos que hemos obtenido la prolongación analítica de $\sum_{n=0}^{\infty} z^n$ que pasa por la curva $\gamma$. \\

\begin{figure}[h]{}
    \centering
    \begin{tikzpicture}[use Hobby shortcut]
        \draw (4, 1) circle (3.1622776601683795cm);
        \draw (0, 0) circle (1cm);
        \draw (0.8377223398316204 ,1) circle (1.01308145723319cm);
        \draw (0.19918404494511877, 0.5989105159617241) circle (1cm);

        \filldraw[black] (0, 0) circle (1pt) node[below, text=black] {\small $z_0 = 0$};
        \filldraw[black] (0.19918404494511877, 0.5989105159617241) circle (1pt) node[below right=-0.1, text=black] {\small $z_1$};
        \filldraw[black] (0.8377223398316204, 1) circle (1pt) node[below right=-0.15, text=black] {\small $z_2$};
        \filldraw[black] (4, 1) circle (1pt) node[below, text=black] {\small $z_m = a$};
        \filldraw[black] (1, 0) circle (1pt) node[below, text=black] {\small $1$};

        \draw (0, 0) .. (0.19918404494511877, 0.5989105159617241) .. (0.8377223398316204, 1) .. (4, 1);
        \node[above] at (2.5, 1.2) {$\gamma$};
    \end{tikzpicture}
    \caption{Prolongaciones analíticas.}
    \label{fig:prolongacion}
\end{figure}
\end{proof}

%%%%%%%%%%%%%%%%%%%%%%%%%%%%%%%%%%%%%%%%%%%%%%%%%%%%%%%%%%%%%%%%%%%%%%%%%%%%%%%%
%EJEMPLO 2
%%%%%%%%%%%%%%%%%%%%%%%%%%%%%%%%%%%%%%%%%%%%%%%%%%%%%%%%%%%%%%%%%%%%%%%%%%%%%%%%

\begin{example}
    Mostrar que
    \begin{equation*}
        g(z) = \sum_{n=1}^{\infty} \frac{z^n}{n}, \, \abs{z} < 1
    \end{equation*}
    diverge en $z = 1$  y converge en el resto de punto tales que $\abs{z} = 1$;

\end{example}

\begin{proof}
    En primer lugar, cabe destacar que la serie armónica $\sum_{n=1}^{\infty} \frac{1}{n}$ diverge. Para demostrar que la serie diverge en $z = 1$  y converge en el resto de punto tales que $\abs{z} = 1$ vamos a aplicar el criterio de Dirichlet, que recordamos a continuación. \\

    Sean $\{a_n\} \subset \real$ y $\{b_n\} \subset \complex$ sucesiones tales que:
    \begin{enumerate}
        \item $\{a_n\}$ es monótona con límite $0$
        \item Las sumas parciales de la serie $\sum_{n=1}^{\infty} b_n$ están acotadas
    \end{enumerate}
    %\begin{enumerate}
    %    \item $a_1 \geq a_2 \geq \dots$
    %    \item $\lim_{n \to \infty} a_n = 0$
    %    \item Existe $M > 0$ tal que $\sum_{n=1}^{N} b_n \leq M$ para todo $N \in \naturals$
    %\end{enumerate}
    entonces $\sum_{n=1}^{N} a_nb_n$ converge. \\

    En nuestro caso vamos a tomar $a_n = \frac{1}{n}$ y $b_n = z^n$. La primera condición se cumple, veamos la que resta:
    \begin{equation*}
        \abs{\sum_{n=1}^{N} z^n} = \abs{\frac{z - z^{N+1}}{1 - z}} \leq \frac{2}{\abs{1 - z}},
    \end{equation*}
    si $z \neq 1$, para todo $N \in \naturals$. \\

    Esto muestra que la condición se satisface para todo $z \not = 1$ en el disco unidad. Por lo tanto, la serie converge para todo $z$ tal que $\abs{z} \leq 1, z \not = 1$ y diverge para $\abs{z} > 1$. \\

    Vamos a ver que la suma de la serie es $\log{\frac{1}{1 - z}}$. En efecto, derivando tenemos que
    \begin{equation*}
        g'(z) = \sum_{n=1}^{\infty} z^{n-1} \Rightarrow z g'(z) = \sum_{n=1}^{\infty} z^n = \frac{z}{1 - z}.
     \end{equation*}
     Si integramos ahora la expresión de la derecha tenemos que la suma es $\log{\frac{1}{1 - z}}$ puesto que $g(0) = 0$. \\
     % Puedes  añadir que un argumento análogo permite presentar ejemplos de series con disco de convergencia de radio 1, pero que divergen en una cantidad arbitraria de puntos del borde. Por ejemplo, la serie de término general z^{pn}/n, diverge en las p raíces p-ésimas de la unidad.
\end{proof}

%%%%%%%%%%%%%%%%%%%%%%%%%%%%%%%%%%%%%%%%%%%%%%%%%%%%%%%%%%%%%%%%%%%%%%%%%%%%%%%%
%EJEMPLO 3
%%%%%%%%%%%%%%%%%%%%%%%%%%%%%%%%%%%%%%%%%%%%%%%%%%%%%%%%%%%%%%%%%%%%%%%%%%%%%%%%

\begin{example}
    Mostrar que
    \begin{equation*}
        f(z) = \sum_{n=1}^{\infty} \frac{z^n}{n^2}, \, \abs{z} < 1
    \end{equation*}
    converge absoluta y uniformemente en $\abs{z} = 1$.
\end{example}

\begin{proof}
    Por el criterio mayorante de Weierstrass, es fácil ver que converge absoluta y uniformemente si $\abs{z} \leq 1$ dado que
    \begin{equation*}
        \sum_{n=1}^{\infty} \abs{\frac{z^n}{n^2}} \leq \sum_{n=1}^{\infty} \abs{\frac{1}{n^2}} < \infty.
    \end{equation*}
    \\
    Esta función $f$ define una función holomorfa y acotada en el disco abierto $\disk$, que además es continua en el disco cerrado $\closedisk$. Sin embargo, no puede extenderse a una función que sea derivable en $z = 1$. \\
    \begin{equation*}
        f'(z) =  \sum_{n=1}^{\infty} \frac{z^{n-1}}{n} \Rightarrow zf'(z) = \sum_{n=1}^{\infty} \frac{z^{n}}{n} = g(z) = \log{\frac{1}{1 - z}}.
    \end{equation*}

    \begin{comment}
    \begin{equation*}
        f''(z) =  \sum_{n=2}^{\infty} \frac{(n - 1)}{n} z^{n-2} =  \sum_{n=2}^{\infty} z^{n-2} - \sum_{n=2}^{\infty} \frac{1}{n} z^{n-2}
    \end{equation*}
     \begin{equation*}
         \sum_{n=2}^{\infty} z^{n-2} = \sum_{n=0}^{\infty} z^{n} = \frac{1}{1 - z}.
    \end{equation*}
    \end{comment}
 \end{proof}

%%%%%%%%%%%%%%%%%%%%%%%%%%%%%%%%%%%%%%%%%%%%%%%%%%%%%%%%%%%%%%%%%%%%%%%%%%%%%%%%
%EJEMPLO 4
%%%%%%%%%%%%%%%%%%%%%%%%%%%%%%%%%%%%%%%%%%%%%%%%%%%%%%%%%%%%%%%%%%%%%%%%%%%%%%%%

\begin{example}
    Mostrar que la serie lagunar,
    \begin{equation*}
        h(z) = \sum_{n=0}^{\infty}  z^{2^n}, \, \abs{z} < 1
    \end{equation*}
 tiene una singularidad en cada punto tal que $\abs{z} = 1$.
\end{example}

\begin{proof}
     Sea $h(z) = \sum_{n=0}^{\infty} z^{2^n} = z + z^2 + z^4 + z^8 + \cdots$. Podemos escribir lo siguiente:
    \begin{equation*}
         h(z^2) = h(z) - z, \,
         h(z^4) = h(z^2) - z^2,
    \end{equation*}
    y aplicando inducción tenemos que
    \begin{equation*}
        h(z^{2^k}) = h(z^{2^{k-1}}) - z^{2^{k-1}}
    \end{equation*}

    Así,
    \begin{equation*}
        h(z) = z + h(z^2) = z + z^2 + h(z^4) = \cdots = z + z^2 + \cdots + z^{2^{k-1}} + h(z^{2^k}).
    \end{equation*}

    Si $m, n \in \naturals$ y $r \in (0,1)$ y llamamos $r$ a $e^{2 \pi i \frac{m}{2^n}}$, tenemos que
    \begin{equation*}
        h(r^{2^n}) = \sum_{k=0}^{\infty} (r^{2^n})^{2^k} = \sum_{k=0}^{\infty} r^{2^n \cdot 2^k} = \sum_{k=0}^{\infty} r^{2^{(n+k)}} =  \sum_{k=n}^{\infty} r^{2^k}.
    \end{equation*}

    Como
    \begin{equation*}
        \sum_{k=n}^{\infty} r^{2^k} \geq \sum_{k=n}^{N} r^{2^k} > (N + 1) r^{2^k} \to N + 1,
    \end{equation*}

    entonces $\lim_{r \to 1} \abs{h(re^{2 \pi i \frac{m}{2^n}})} = \infty \, \, \forall m, n$. \\

    Puesto que $\{e^{2 \pi i \frac{m}{2^n}} : m, n \in \naturals\}$ es denso en $\partial \disk$, todos los puntos del borde del disco unidad son singulares. \\
\end{proof}


\begin{example}
    \label{ex:exp}
    Mostrar que la función
    \begin{equation*}
        f(z) = \exp{\left(\frac{z + 1}{z - 1}\right)}, \, z \in \disk
    \end{equation*}
    es holomorfa, $\abs{f(z)} < 1$ para todo $z \in \disk$, y $f(t) \to 0$ cuando $t \to 1^-$. % Lo que ocurre es que se puede extender al disco cerrado menos el 1 y |f(w)|=1 si |w|=1 y w\neq 1.
\end{example}

\begin{proof}
    La función $f$ es holomorfa ya que es la composición de funciones holomorfas. Obsérvese que el único punto singular es $z = 1$ y $f$ y todas sus derivadas tienen límite radial 0 en $e^{i \theta} = 1$ . \\

    La función $g(z) = \frac{z + 1}{z - 1}$ lleva el disco en el semiplano izquierdo $H = \{w: \Re (w) < 0\}$, así que $z = 1$ se corresponde con $\infty$ y $\partial \disk \setminus \{1\}$ se corresponde con el eje imaginario. Así pues, la exponencial $e^{g(z)}$ es holomorfa en $\partial \disk \setminus \{1\}$ y lleva $H$ en $\disk$:
    \begin{equation*}
        \abs{e^z} = \abs{e^{x+iy}} = \abs{e^{x}(\cos y + i\sen y)} = e^{x} < 1.
    \end{equation*}

    La aplicación $g$ es una transformación de Möbius, y dichas transformaciones tienen la propiedad de que llevan circunferencias y rectas en circunferencias y rectas. Como la función lleva $-1$ a $0$, $i$ a $-i$ y $-i$ a $i$, la imagen del círculo $\abs{z}=1$, ha de ser una recta. \\

    Si tomamos una sucesión $\{t_n\}$ en el intervalo $(-1,1)$ que converge a $1$ cuando $n$ tiende a $\infty$, se tiene que $g(t_n)  = \frac{t + 1}{t - 1}$ tiende a $- \infty$ cuando $t$ tiende a $1^-$. Por lo tanto,
    \begin{equation*}
        \frac{t + 1}{t - 1} \xrightarrow[t \to 1^-]{}  - \infty \Rightarrow \exp \left(  \frac{t + 1}{t - 1} \right) \xrightarrow[t \to 1^-]{} 0.
    \end{equation*}

    Sin embargo, la función $f$ no tiene límite en $1$. Por ejemplo, si tomamos la sucesión $\{z_n\}$ definida por $z_n = g(w_n)$, siendo $\{w_n\}$ la sucesión de término general $-1 + 2n \pi i$. Entonces,
     \begin{equation*}
         z_n = \frac{2n \pi i}{-2 + 2n \pi i} = \frac{n \pi i}{n \pi i - 1} =  \frac{(n \pi i + 1) n \pi i}{- n^2 \pi^2 - 1} = \frac{-n^2 \pi^2 + i n \pi}{-n^2 \pi^2 - 1}.
     \end{equation*}

     Como $g = g^{-1}$ tenemos
     \begin{equation*}
         e^{g(z_n)} = e^{w_n} \to e^{-1} \not = 0.
     \end{equation*}

     Estudiaremos este ejemplo con más detalle en el Capítulo \ref{cap:banach}, centrándonos de nuevo en el comportamiento de la función en el punto singular $z = 1$.
 \end{proof}
