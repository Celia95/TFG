\chapter{Ejemplos}

%Empezar comentando cómo calcular el radio de convergencia, que los ejemplos son de series con radio de convergencia 1 y que vas a mostrar cómo difieren el comportamiento en la frontera de unos a otros.

En esta sección vamos a estudiar el comportamiento de algunas series de potencias en el borde de su disco de convergencia.

1. $\sum_{n=0}^{\infty} z^n, \abs{z} < 1$, diverge en todo punto tal que $\abs{z} = 1$;

2. $\sum_{n=1}^{\infty} \frac{z^n}{n}, \abs{z} < 1$, diverge en $z = 1$  y converge en el resto de punto tales que $\abs{z} = 1$;

3. $\sum_{n=1}^{\infty} \frac{z^n}{n^2}, \abs{z} < 1$, converge absoluta y uniformemente en $\abs{z} = 1$.

4. 4. La serie lagunar: $\sum_{n=1}^{\infty}  z^{2^n}, \abs{z} < 1$,  \\ \par


Para el primer ejemplo, es fácil ver que $1 - z^{n+1} = (1 - z) (1+ z + z^2 + \cdots + z^n)$ así que
\begin{equation}
    1 + z + \cdots + z^n = \frac{1 - z^{n+1}}{1-z}.
\end{equation}

Si $\abs{z} < 1$ entonces $\lim z^n = 0$ y la serie converge a
\begin{equation*}
    \sum_{n=0}^{\infty} z^n = \frac{1}{1 - z}
\end{equation*}

Si $\abs{z} > 1$ entonces $\lim z^n = \infty$ y la serie diverge. Pero, ¿qué pasa cuando $\abs{z} = 1$? La serie de potencias $\sum_{n=0}^{\infty} z^n$ diverge en todos los puntos del radio de convergencia pues $\abs{z^n}$ no tiende a 0 cuando $n \rightarrow \infty$.

Sin embargo, $\sum_{n=0}^{\infty} z^n$ puede ser extendida a la función globalmente analítica $\frac{1}{1-z}$ en $\complex \setminus \{1\}$ gracias a una cantidad finita de prolongaciones analíticas.

Tomemos $a$ un punto cualquiera de $\complex \setminus \{1\}$ y conectémoslo al origen $0$ mediante la curva de Jordan $\gamma \subset \complex \setminus \{1\}$. Fijemos un punto $z_1$ en $\gamma$ que cumpla $\abs{z} < 1$. $\sum_{n=0}^{\infty} z^n$ puede ser extendida analíticamente en $z_1$ de la siguiente forma:
\begin{equation*}
    \frac{1}{1-z} = \sum_{n=0}^{\infty} \frac{1}{(1 - z_1)^{n+1}} (z - z_1)^n , \abs{z - z_1} < \abs{1 - z_1}.
\end{equation*}

De nuevo, tomemos $z_2$ en $\gamma$ tal que $\abs{z - z_1} < \abs{1 - z_1}$ y $\abs{z} \geq 1$. Podemos extender la serie de potencias a $z_2$
\begin{equation*}
    \frac{1}{1-z} = \sum_{n=0}^{\infty} \frac{1}{(1 - z_2)^{n+1}} (z - z_2)^n , \abs{z - z_2} < \abs{1 - z_2}.
\end{equation*}

%Aquí vendría bien un dibujo, como en el libro de Lin. Y quizá hacer una vez el cálculo como te he anotado en la hoja que te has llevado. Por ejemplo, escribir que si b \neq 1, entonces 1/(1-z)=1/(1-b-(z-b)) y que converge a la serie apropiada si |z-b|<|1-b|. Poner ese cálculo antes de empezar a prolongar y luego omitir los cálculos, como hace Lin, para explicar el proceso.  

Después de un número finito de iteraciones, alcanzaremos el punto $a$ y tendremos
\begin{equation*}
    \frac{1}{1-z} = \sum_{n=0}^{\infty} \frac{1}{(1 - a)^{n+1}} (z - a)^n , \abs{z - a} < \abs{1 - a}.
\end{equation*}

Así, decimos que hemos obtenido la prolongación analítica de $\sum_{n=0}^{\infty} z^n$ que pasa por la curva $\gamma$. De hecho, está extensión no depende la curva de Jordan que tomemos. En efecto, sea $\alpha \subset \complex \setminus \{1\}$ otra curva de Jordan que conecta $a$ con $0$ y extendamos analíticamente $\sum_{n=0}^{\infty} z^n$ a través de $\alpha$ hasta el punto $a$. Por la propiedad de unicidad, obtendremos la misma serie que en el caso anterior. Por lo tanto, $\frac{1}{1 - z}$ está bien definida en $\complex \setminus \{1\}$ \\ \par


Segundo ejemplo. $\sum_{n=1}^{\infty} \frac{z^n}{n}, \abs{z} < 1$, diverge en $z = 1$  y converge en el resto de punto tales que $\abs{z} = 1$;
Vamos a aplicar el criterio de Dirichlet que dice lo siguiente: si $\{a_n\}$ son números reales y $\{b_n\}$ son números complejos tales que:

1. $a_1 \geq a_2 \geq \dots$

2. $\lim_{n \rightarrow \infty} a_n = 0$

3. Existe $M > 0$ tal que $\sum_{n=1}^{N} b_n \leq M$ para todo $N \in \naturals$

entonces $\sum_{n=1}^{N} a_nb_n$ converge.

En nuestro caso vamos a tomar $a_n = \frac{1}{n}, b_n = z^n$. Las dos primeras condiciones se cumplen, veamos la tercera:
\begin{equation*}
\abs{\sum_{n=1}^{N} z^n} = \abs{\frac{z - z^{N+1}}{1 - z}} \leq \frac{2}{\abs{1 - z}}, \text{ si } z \neq 1 \text{, para todo } N \in \naturals.
\end{equation*}

Esto muestra que la tercera condición se satisface para todo $z \not = 1$ en el disco unidad. Por lo tanto, la serie converge para todo $z$ tal que $\abs{z} \leq 1, z \not = 1$ y diverge para $\abs{z} > 1$. \\ \par

% Si derivas, te sale la serie geométrica de arriba e, integrando, la suma es, puesto que en 0 vale 0, -Log(1-z).

%Puedes  añadir que un argumento análogo permite presentar ejemplos de series con disco de convergencia de radio 1, pero que divergen en una cantidad arbitraria de puntos del borde. Por ejemplo, la serie de término general z^{pn}/n, diverge en las p raíces p-ésimas de la unidad.

El tercer ejemplo es fácil ver que converge absoluta y uniformemente en $\abs{z} = 1$ dado que $\sum_{n=1}^{\infty} \abs{\frac{z^n}{n^2}} \leq \sum_{n=1}^{\infty} \abs{\frac{1}{n^2}} < \infty$.

%Sobre este ejemplo diría que define una función holomorfa y acotada en el disco abierto, continua en el disco cerrado. De nuevo se puede sumar, usando una argumento como el de antes.

%Faltaría el ejemplo 4 de la serie lagunar, con mucho coeficientes nulos y otros 1's, pero muy separados. Lo comentamos un día, pero también están en el libro de Lin, páginas 38 y 39.

