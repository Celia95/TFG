\chapter{Teorema de Fatou y Teorema de Carathéodory}

\begin{comment}
    \begin{theorem}[Lema de Schwart]
        Sea $f: \disk \rightarrow \closedisk$ una función $\in \holomorphic{\disk}$ tal que $f(0) = 0$. Entonces:
        \begin{itemize}
            \item $\abs{f(z)} \leq \abs{z}$ para todo $z \in \disk$.
            \item Si para algún $z_0 \not = 0$ tenemos que $\abs{f(z_0)} = \abs{z_0}$, entonces existe $\alpha \in \complex, \abs{\alpha} = 1$ tal que $f(z)=\alpha z$.
        \end{itemize}
    \end{theorem}

    \begin{proof}
        Sea $f(z) = a_1z + \cdots$ la serie de potencias de $f$. El término constante es $0$ puesto que suponemos que $f(0) = 0$. Entonces $f(z)/z$ es una función holomorfa y
        \begin{equation*}
            \abs{\dfrac{f(z)}{z}} < 1/r \text{ para } \abs{z} = r < 1
        \end{equation*}
    \end{proof}
\end{comment}

\section{Integral de Poisson y Teorema de Fatou}

\subsection{La Integral de Poisson}

\begin{definition}
    Se llama núcleo de Poisson a la función $P$ definida por
    \begin{equation}
        \label{poisson1}
        P:(r,t) \in [0,1)\times \real \mapsto P_r(t) = \sum_{n=-\infty}^{\infty} r^{\abs{n}}e^{int}.
    \end{equation}

    Podemos considerar el núcleo de Poisson como una función de dos variables $r$ y $t$ o como una familia de funciones de $t$ que dependen de $r$.

    Dados $z=re^{i \theta}$, con $r \in [0,1)$ y $\theta \in \real$ se tiene que
    \begin{equation}
        \label{poisson2}
        P_r(\theta - t) = \Re \left[ \dfrac{e^{it} + z}{e^{it} - z} \right] = \dfrac{1 - r^2}{1 - 2r \cos (\theta - t) + r^2}
    \end{equation}
    para todo $t \in \real$. En efecto:

    \begin{equation*}
        \begin{split}
            P_r(t) & = \sum_{n=-\infty}^{\infty} r^{\abs{n}}e^{int} = 1 + \sum_{n=1}^{\infty} r^n e^{int} + \sum_{n=1}^{\infty} r^n e^{-int} = 1 + \sum_{n=1}^{\infty} r^n (e^{int} + e^{-int}) = \\
                   & =  1 + \sum_{n=1}^{\infty} r^n 2 \Re(e^{int}) = \Re \left[ 1 + 2 \sum_{n=1}^{\infty} (r e^{it})^n  \right] = \Re \left[ 1 + 2 \dfrac{re^{it}}{1-re^{it}} \right] = \Re \left[\dfrac{1 + re^{it}}{1-re^{it}} \right].
        \end{split}
    \end{equation*}
    \\ \par
    Por otra parte
    \begin{equation*}
        \Re \left[ \dfrac{1 + re^{it}}{1-re^{it}} \right] = \Re \left[ \dfrac{(1 + re^{it})(1 - re^{it})}{\abs{1-re^{it}}^2} \right] = \dfrac{1 - r^2}{1 - 2r \cos t + r^2}.
    \end{equation*}
\end{definition}

\newpage

\textbf{Propiedades del núcleo de Poisson:}% \\ \par

%De \ref{poisson1} tenemos que
\begin{equation}
    \dfrac{1}{2 \pi} \int_{- \pi}^{\pi} P_r (t) dt = 1, \forall r \in [0,1).
\end{equation}

%De \ref{poisson2} se sigue que
\begin{equation}
    P_r(t) > 0, \forall r \in [0,1), t \in \real
\end{equation}

\begin{equation}
    P_r(t) = P_r(-t), \forall r \in [0,1), t \in \real
\end{equation}

\begin{equation}
    P_r(t) < P_r(\delta), 0 < \delta < \abs{t} \leq \pi
\end{equation}

\begin{equation}
    \lim_{r \rightarrow 1} P_r(\delta) = 0, \forall \delta \in (0,\pi]
\end{equation}

\bigskip \par

\begin{definition}
    Se llama integral de Poisson de una función $f \in L^1(\partial \disk)$ a la función $F$ dada por
    \begin{equation*}
        F: z=re^{i \theta} \in \disk \mapsto F(re^{i \theta}) = \dfrac{1}{2 \pi} \int_{- \pi}^{\pi} P_r (\theta - t) f(e^{it}) dt.
    \end{equation*}

    Algunas veces nos convendrá referirnos a ella como $F=P[f]$.
\end{definition}

\bigskip

Además si $f$ lleva $\partial \disk$ en los reales, \ref{poisson2} nos muestra que

\begin{equation*}
     P[f] = \Re \left[ \dfrac{1}{2} \int_{-\pi}^{\pi} \dfrac{e^{it} + z}{e^{it} - z} f(t) dt \right].
\end{equation*}

\subsection{Teorema de Fatou}

Para demostrar el Teorema de Fatou nos vamos a basar en unos resultados clásicos del libro \citet[chap. 11]{rudin}.

\begin{theorem}
    \label{fatouaux1}
    Si $f \in L^1(\partial \disk)$ y $F = P[f]$, entonces
    \begin{equation*}
        \lim_{r \rightarrow 1} F(re^{i \theta}) = f(e^{i \theta})
    \end{equation*}
\end{theorem}

\begin{theorem}
    \label{fatouaux2}
    Sean $f \in C (\partial \disk), F=P[f]$ y
    \begin{equation*}
        u(re^{i \theta}) =
        \begin{cases}
            f(re^{i\theta}) & \text{si } r=1\\ F(re^{i\theta}) & \text{si } 0 \leq r<1
        \end{cases}
    \end{equation*}
    Entonces $u$ es una función continua en el disco cerrado $\closedisk$ que es armónica en $\disk$ .
\end{theorem}

\begin{theorem}[Teorema de Fatou]
    Para toda función $f \in \bholomorphic{\disk}$, existe una función $f^* \in L^{\infty} (\partial \disk)$ definida por
    \begin{equation}
        \label{fatou1}
        f^*(e^{it}) = \lim_{r \rightarrow 1} f(re^{it})
    \end{equation}
    en casi todo punto.

    Se tiene la igualdad $\norminf{f} = \norminf{f^*}$. Para todo $z \in U$, la fórmula integral de Cauchy
    \begin{equation}
        \label{fatou2}
        f(z) = \dfrac{1}{2 \pi i} \int_{\gamma} \dfrac{f^*(\xi)}{\xi - z} d\xi
    \end{equation}

    se satisface, donde $\gamma$ es el círculo unidad positivamente orientado: $\gamma(t) = e^{it}, 0 \leq t \leq 2 \pi$.

    Las funciones $f^* \in L^{\infty}(\partial \disk)$ que se obtienen mediante este procedimiento son precisamente aquellas que cumplen la siguiente relación

    \begin{equation}
        \label{fatou3}
        \dfrac{1}{2 \pi i} \int_{-\pi}^{\pi} f^*(e^{it})e^{-int} dt = 0, n = -1,-2, \dots
    \end{equation}
\end{theorem}

\begin{proof}
    La existencia de $f^*$ se sigue de los teoremas \ref{fatouaux1} y \ref{fatouaux2}.

    Por \ref{fatou1}, tenemos que $\norminf{f^*} \leq \norminf{f}$.

    Si $z \in U$ y $\abs{z} < r < 1$, tomemos $\gamma_r(t) = r e^{it}, 0 \leq t \leq 2\pi$. Entonces,
    \begin{equation*}
        f(z) = \dfrac{1}{2 \pi i} \int_{\gamma_r} \dfrac{f(\xi)}{\xi - z} d\xi =
        \dfrac{r}{2 \pi} \int_{-\pi}^{\pi} \dfrac{f(re^{it})}{re^{it} - z}e^{it} dt
    \end{equation*}

    Sea $\{r_n\}$ una sucesión tal que $r_n \rightarrow 1$. Por el teorema de la convergencia dominada de Lebesgue tenemos %quizá puedes aclarar a qué sucesión de funciones f_n lo aplicas, para que se vea mejor y no haga falta más abajo repetir el argumento.
    \begin{equation}
        \label{fatou_proof}
        f(z) = \dfrac{1}{2 \pi} \int_{-\pi}^{\pi} \dfrac{f^* (e^{it})}{1 - ze^{-it}} dt
    \end{equation}
    Por lo que ya hemos probado \ref{fatou2}. Por el teorema de Cauchy, se sigue que
    \begin{equation*}
        \int_{\gamma_r} f(\xi)\xi^n d\xi = 0, n = 0, 1, \dots
    \end{equation*}

    Tomando de nuevo una sucesión $\{r_n\}$ que tienda a 1, el teorema de la convergencia dominada garantiza que $f^*$ cumple \ref{fatou3}. Además, podemos convertir \ref{fatou_proof} en una integral de Poisson, si $z = re^{i \theta}$,
    \begin{equation*}
         \begin{split}
             f(z) & = \dfrac{1}{2 \pi} \int_{- \pi}^{\pi} f^*(e^{it}) \sum_{n=0}^{\infty} r^n e^{in(\theta - t)} dt =  \dfrac{1}{2 \pi} \int_{- \pi}^{\pi} f^*(e^{it}) \sum_{n=-\infty}^{\infty} r^{\abs{n}} e^{in(\theta - t)} dt = \\
                  & =  \dfrac{1}{2 \pi} \int_{- \pi}^{\pi} P_r(\theta - t) f^*(e^{it}) dt
         \end{split}
    \end{equation*}

    De esto concluimos que $\norminf{f} \leq \norminf{f^*}$, así que ambas normas coinciden.
\end{proof}

\section{Teorema de Carathéodory}

\begin{definition}{Aplicación conforme}
    Sean $U$ y $V \subset \complex^n$. Una aplicación $f: U \rightarrow V$ se llama conforme en un punto $u \in U$ si preserva la orientación y los ángulos entre curvas que pasan por $u$.
\end{definition}

\begin{prop}
    Sea $U \subset \complex$. Una aplicación $f: U \rightarrow \complex$ es conforme en $U$ si $f \in \holomorphic{U}$ y $f'(z) \not = 0 \, \forall z \in U$.
\end{prop}

\begin{proof}
    Supongamos que $f(z)$ es una función holomorfa en $U$ tal que $f'(z) \not = 0$ para $z \in U$ y consideremos $f:z \rightarrow w=f(z)$. Sea $\gamma: [a,b] \rightarrow U$ una curva suave. Consideremos $\lambda = (f \circ  \gamma)(t)$. Por la regla de la cadena, $\lambda$ es continuamente diferenciable y como $f'(\gamma(t)) \not = 0$, tenemos
    \begin{equation}
        \label{cadena}
        \lambda'(t) = f'(\gamma(t))\gamma'(t).
    \end{equation}
    Por lo tanto, $\lambda$ es una curva suave en el plano $w$.

    Sean $\gamma_1, \gamma_2: [a,b] \rightarrow U$ curvas suaves tales que $c=\gamma_1(a) = \gamma_2(a)$. Definimos el ángulo $\theta$ entre $\gamma_1$ y $\gamma_2$ en $c$ como el argumento de $\frac{\gamma_2'(a)}{\gamma_1'(a)}$. Como el argumento es aditivo para la multiplicación de funciones, tenemos que
    \begin{equation*}
    \begin{split}
        \arg \lambda_1'(a) = \arg f'(c) + \arg \gamma_1'(a)\\
        \arg \lambda_2'(a) = \arg f'(c) + \arg \gamma_2'(a)
    \end{split}
    \end{equation*}

    y entonces
    \begin{equation*}
        \arg  \frac{\lambda_2'(a)}{\lambda_1'(a)} = \arg \lambda_2'(a) - \arg \lambda_1'(a) = \arg \gamma_2'(a) - \arg \gamma_1'(a) = \arg  \frac{\gamma_2'(a)}{\gamma_1'(a)}.
    \end{equation*}

    Así, el ángulo entre las curvas $\lambda_1$ y $\lambda_2$ en $d = \lambda_1(a) = \lambda_2(a)$ es igual al ángulo $\theta$ entre las curvas $\gamma_1$ y $\gamma_2$ en $c$.

    \begin{comment}
    Supongamos que $f(z)$ es una función holomorfa en $U$ tal que $f'(z) \not = 0$ para $z \in U$ y consideremos $f:z \rightarrow w=f(z)$. Sea $\gamma: [a,b] \rightarrow U$ una curva suave. Consideremos $\lambda = (f \circ  \gamma)(t)$. Por la regla de la cadena, $\lambda$ es continuamente diferenciable y como $f'(\gamma(t)) \not = 0$, tenemos
    \begin{equation}
        \label{cadena}
        \lambda'(t) = f'(\gamma(t))\gamma'(t).
    \end{equation}

    Por lo tanto, $\lambda$ es una curva suave en el plano $w$.

    Sean $\gamma_1, \gamma_2: [a,b] \rightarrow U$ curvas suaves tales que $c=\gamma_1(a) = \gamma_2(a)$. Definimos el ángulo $\theta$ entre $\gamma_1$ y $\gamma_2$ en $c$ como el argumento de $\frac{\gamma_2'(a)}{\gamma_1'(a)}$, es decir,
    \begin{equation*}
        \dfrac{\gamma_2'(a)}{\gamma_1'(a)} = \abs{\dfrac{\gamma_2'(a)}{\gamma_1'(a)}} e^{i\theta}.
    \end{equation*}

    La aplicación $f$ lleva las curvas $\gamma_1$ y $\gamma_2$ en curvas suaves $\lambda_1=f(\gamma_1)$ y $\lambda_2=f(\gamma_2)$ que tienen como punto inicial $d=f(c)$. Por \ref{cadena} tenemos
    \begin{equation*}
        \dfrac{\lambda_2'(a)}{\lambda_1'(a)} = \dfrac{\gamma_2'(a)}{\gamma_1'(a)}
    \end{equation*}
    entonces el ángulo entre las curvas $\lambda_1$ y $\lambda_2$ en $d = \lambda_1(a) = \lambda_2(a)$ es igual al ángulo $\theta$ entre las curvas $\gamma_1$ y $\gamma_2$ en $c$.
    \end{comment}
\end{proof}

Vamos a probar un resultado recíproco a éste que incluye algunas restricciones adicionales sobre $f$.

\begin{prop}
    Sean $U \subset \complex$ y $f: U \rightarrow \complex$ una aplicación conforme en $U$ que admite derivadas parciales continuas con respecto a $x$ e $y$. Entonces $f \in \holomorphic{U}$ y $f'(z) \not = 0 \, \forall z \in U$.
\end{prop}

\begin{proof}
    Fijemos $z$ un punto arbitrario de $U$, y elijamos $\varepsilon > 0$ tal que $D(z, \varepsilon) \subset U$. Consideremos la familia de curvas suaves $\gamma_{\theta}(t) = z + te^{i \theta}, 0 \leq t \leq \varepsilon, \theta \in \real$. Nótese que el ángulo entre $\gamma_0$ y $\gamma_{\theta}$ en $z$ es $\theta$.

     Tomemos la familia de curvas $\lambda_\theta = (f \circ \gamma_\theta$. Como $f$ es conforme, el ángulo entre $\lambda_0$ y $\lambda_{\theta}$ en $f(z)$ es $\theta$. Como $f$ es conforme, el ángulo entre $\lambda_0$ y $\lambda_{\theta}$, es decir, el argumento de $\frac{\lambda_{\theta}'(0)}{\lambda_0' (0)}$ es igual a $\theta$. Si escribimos el argumento de $\lambda_0'(0)$ como $\alpha$, el argumento de $\lambda_{\theta}'(0)$ será $\alpha+\theta$ y, por tanto,
    \begin{equation}
    \label{conformal}
        e^{-i(\theta + \alpha)} \lambda_{\theta}'(0)= \abs{\lambda_{\theta}'(0)} > 0.
    \end{equation}
    \ref{cauchy-riemann}, nos dice que
    \begin{equation*}
    \begin{split}
        \lambda_{\theta}'(0) & = u_x \cos \theta + u_y \sin \theta + i(v_x \cos \theta + v_y \sin \theta) = \\
                             & = (u_x + iv_x) \cos \theta + (u_y + i v_y) \sin \theta = f_x \cos \theta + f_y \sin \theta,
    \end{split}
    \end{equation*}
    por la identidad de Euler,
    \begin{equation*}
        2 \lambda_{\theta}'(0) = (f_x - i f_y) e^{i \theta} + (f_x + i f_y) e^{-i \theta}.
    \end{equation*}

    Entonces por \ref{conformal},
    \begin{equation*}
         (f_x - i f_y) e^{-i \alpha} + (f_x + i f_y) e^{-2i \theta -i  \alpha} = 2 \abs{\lambda_{\theta}'(0)}.
    \end{equation*}

    Derivando en ambos lados con respecto a $\theta$, obtenemos
    \begin{equation*}
        -2i (f_x + i f_y) e^{-2i \theta - i \alpha} = \frac{2d}{d \theta} \abs{\lambda_{\theta}'(0)}.
    \end{equation*}
    Resulta que esa cantidad cumple que al multiplicarla por $-2ie^{-2i\alpha-i\theta}$ tiene siempre parte imaginaria nula. Como el ángulo $\theta$ es arbitrario (pero $\alpha$ es fijo), tiene que ser nulo $f_x+if_y$, ya que el producto por $e^{-2i\alpha-i\theta}$ resulta ser entonces un giro de ángulo arbitrario. Solo puede tener parte imaginaria nula siempre si es nulo, claro.

    Como $\theta$ es una variable real y la parte de la derecha de la igualdad solo toma valores reales, concluimos que
    \begin{equation*}
        f_x + i f_y = 0
    \end{equation*}
    por lo que
    \begin{equation*}
        u_x + v_y + i(v_x + u_y) = 0.
    \end{equation*}

    Como vemos, $u(x,y)$ y $v(x,y)$ satisfacen las ecuaciones de Cauchy-Riemann en $U$. Luego $f(z) = u(x,y) + i v(x,y)$ es holomorfa en $z = x + iy \in U$. Falta ver que $f(z) \not = 0, z \in U$.
    % Errata del libro: hay que usar que lo que el libro llama (3.5) ??
\end{proof}

\begin{comment}
\begin{prop}
    Sea $U \subset \complex$. Una aplicación $f: U \rightarrow \complex$ es conforme en $U$ si satisface las condiciones de Cauchy-Riemann y $f'(z) \not = 0 \, \forall z \in U$.
\end{prop}

\begin{proof}
    Sea $f$ una función continua tal que $f(z) = u(x,y) + i v(x,y), z = x+iv$. Sabemos, por hipótesis, que $u(x,y)$ y $v(x,y)$ son funciones continuamente diferenciables.

    Consideremos la curva suave $\gamma : [a,b] \rightarrow U$, que escribimos como $\gamma (t) = \rho (t) + i \sigma (t)$. Entonces,
    \begin{equation*}
        f(\gamma (t)) = u(\rho (t), \sigma (t)) + i v(\rho (t), \sigma (t)).
    \end{equation*}

    Como $f(\gamma (t))$ es continuamente diferenciable,
    \begin{equation}
        \label{cauchy-riemann}
        \frac{d}{dt}f(\gamma (t)) = u_x \rho' (t) + u_y \sigma' (t) + i (v_x \rho'(t) + v_y \sigma'(t)).
    \end{equation}

    Por hipótesis tenemos que
    \begin{equation*}
        \frac{\partial (u,v)}{\partial (x,y)} =
        \left|
        \begin{matrix}
            u_x(x,y) & u_y(x,y) \\ v_x(x,y) & v_y(x,y)
        \end{matrix}
        \right| \not = 0,
    \end{equation*}
    por lo que $\frac{d}{dt} f(\gamma (t)) \not = 0$ en $t = 0$ pues $\rho' (t) + i \sigma' (t) = \gamma' (t) \not = 0$. Es decir, la curva $f (\gamma)$ es suave en un entorno de su origen. Por lo tanto, si $\gamma_1$ y $\gamma_2$ son curvas suaves con origen $c$, el ángulo entre $f(\gamma_1)$ y $f(\gamma_2)$ en $f(c)$ está bien definido.
\end{proof}
\end{comment}
\bigskip

\begin{theorem}[Teorema de Carathéodory]
    Sea $\varphi$ una aplicación conforme del disco unidad $\disk$ en un dominio de Jordan $\Omega$. Entonces $\varphi$ tiene una extensión continua al disco cerrado $\closedisk$, y la extensión es inyectiva de $\closedisk$ en $\xbar{\Omega}$.
\end{theorem}

\begin{proof}
    Vamos a suponer que $\Omega$ está acotado. Fijemos $\zeta \in \partial \disk$. Primero vamos a probar que $\varphi$ tiene una extensión continua en $\zeta$. Sea $0 < \delta < 1$,
    \begin{equation*}
        D(\zeta, \delta) = \{z: \abs{z - \zeta} < \delta \}
    \end{equation*}

    y tomemos $\gamma_{\delta} = \disk \cap \partial D(\zeta, \delta)$. Entonces $\varphi (\gamma_{\delta})$ es una curva de Jordan de longitud
    \begin{equation*}
        L(\delta) = \int_{\gamma_{\delta}} \abs{\varphi ' (z)} ds
    \end{equation*}
    \\
    Por la desigualdad de Cauchy-Schwarz, tenemos
    \begin{equation*}
        L^2(\delta) \leq \pi \delta \int_{\gamma_{\delta}} \abs{\varphi ' (z)}^2 ds
    \end{equation*}

    entonces para $\rho < 1$

    \begin{equation*}
        \int_{0}^{\rho} \frac{L^2(\delta)}{\delta} d\delta \leq \pi \int \int_{\disk \cap D(\zeta, \rho)} \abs{\varphi ' (z)}^2 dxdy = \pi \text{Área}(\varphi(\disk \cap D(\zeta, \rho))) < \infty
    \end{equation*}
    \\
    %Figura
    Entonces, existe una sucesión $\{ \delta_n\} \downarrow 0$ tal que $L(\delta_n) \rightarrow 0$. Cuando $L(\delta_n) < \infty$, la curva $\varphi(\gamma_{\delta_n})$ tiene extremos $\alpha_n, \beta_n \in \xbar{\Omega}$ y ambos puntos deben estar en $\Gamma = \partial \Omega$. De hecho, si $\alpha_n \in \Omega$, entonces algún punto cerca de $\alpha_n$ tiene dos preimágenes distintas en $\disk$ y esto es imposible pues $\varphi$ es inyectiva. Además,
    \begin{equation}\label{res}
        \abs{\alpha_n - \beta_n} \leq L(\delta_n) \rightarrow 0
    \end{equation}
    \\
    Sea $\sigma_n$ el subarco cerrado de $\Gamma$ que tiene extremos $\alpha_n$ y $\beta_n$ y con un diámetro menor. Entonces \ref{res} implica que $\diam(\sigma_n) \rightarrow 0$ porque $\Gamma$ es homeomorfa al círculo. Por el teorema de la curva de Jordan, $\sigma_n \cup \varphi(\gamma_{\delta_n})$ divide al plano en dos regiones, y una de ellas, llamémosla $U_n$ es acotada. Entonces $U_n \subset \Omega$ ya que $\complex^* \setminus \xbar{\Omega}$ es conexo por arcos. Como
    \begin{equation}
        \label{res2}
        \diam(\partial U_n) = \diam(\sigma_n \cup \varphi(\gamma_{\delta_n})) \rightarrow 0,
        \text{ concluimos que }
        \diam(U_n) \rightarrow 0.
    \end{equation}

    Tomamos $D_n = \disk \cup \{ z: \abs{z - \zeta} < \delta_n \}$. Sabemos que para $n$ suficientemente grande, $\varphi(D_n) = U_n$. Si no, por conexión tendríamos que $\varphi(\disk \setminus \xbar{D_n}) = U_n$ y
    \begin{equation*}
        \diam (U_n) \geq \diam (\varphi(B(0, 1/2))) > 0
    \end{equation*}

    que contradice con \ref{res2}. Entonces $\diam(\varphi(D_n)) \rightarrow 0$ y $\bigcap \xbar{\varphi(D_n)}$ es un solo punto pues $\varphi(D_{n+1}) \subset \varphi(D_n)$. Esto significa que $\varphi$ tiene una extensión continua en $\disk \cap \{ \zeta \}$. La extensión a todos estos puntos define una aplicación continua en $\closedisk$.

    Denotemos ahora por $\varphi$ a la extensión $\varphi : \closedisk \rightarrow \xbar{\Omega}$. Como $\varphi(\disk) = \Omega$, $\varphi$ lleva  $\closedisk$ en $\xbar{\Omega}$. Para probar que $\varphi$ es inyectiva, supongamos que $\varphi(\zeta_1) = \varphi(\zeta_2), \zeta_1 \not = \zeta_2$. El argumento utilizado para mostrar que $\alpha_n \in \Gamma$, también prueba que $\varphi (\partial \disk) = \Gamma$, así que podemos suponer que $\zeta_j \in \partial \disk, j=1,2$. La curva de Jordan
    \begin{equation*}
        \{\varphi (r \zeta_1) : 0 \leq r \leq 1\} \cup \{\varphi (r \zeta_2) : 0 \leq r \leq 1\}
    \end{equation*}

    acota al dominio $W \subset \Omega$, luego $\varphi ^{-1} (W)$ es una de las dos componentes de
    \begin{equation*}
        \disk \setminus ( \{ r \zeta_1 : 0 \leq r \leq 1\} \cup \{ r \zeta_2 : 0 \leq r \leq 1\})
    \end{equation*}
    \\
    Pero como $\varphi(\partial \disk) \subset \Gamma$,
    \begin{equation*}
        \varphi(\partial \disk \cap \partial \varphi ^{-1} (W)) \subset \partial W \cap \partial \Omega = \{ \varphi (\zeta_1)\}
    \end{equation*}

    y $\varphi$ es constante en un arco de $\partial \disk$. Se tiene que $\varphi$ es constante, por el principio de reflexión de Schwarz, y esta contradicción prueba que $\varphi(\zeta_1) \not = \varphi(\zeta_2)$.
\end{proof}

\bigskip

El resultado que presentamos a continuación es un recíproco parcial del teorema de Carathéodory. Muestra que la inyectividad en el borde del dominio se traslada al interior, en condiciones adecuadas.

\begin{theorem}
    Sea $\Gamma$ una curva simple, cerrada y suave con interior $\Omega$. Sea $f \in \holomorphic{\Gamma \cup \Omega}$ una aplicación inyectiva en $\Gamma$. Entonces $f$ es holomorfa e inyectiva en $\Omega$.
\end{theorem}

\begin{proof}
    La aplicación $w = f(z)$ lleva $\Gamma$ en un camino simple, cerrado y suave $\Gamma'$. Sea $w_0$ un punto arbitrario que no esté en $\Gamma'$. Entonces, si llamamos $\Gamma_+$ al camino positivamente orientado,
    \begin{equation*}
        n = \dfrac{1}{2 \pi i} \int_{\Gamma_+} \dfrac{f'(z)}{f(z) - w_0} dz =  \dfrac{1}{2 \pi i} \int_{\Gamma'} \dfrac{dw}{w - w_0}.
    \end{equation*}

    Ahora la última integral es cero si $w_0$ está fuera de $\Gamma'$ y es $\pm 1$ si $w_0$ está dentro de $\Gamma'$. Sin embargo, $n$ no puede ser negativo pues la primera integral nos da el número de ceros de $f(z) - w_0$ dentro de $\Gamma$. Entonces, $n=1$ si $w_0$ está dentro de $\Gamma'$.

    Esto prueba que $f(z) = w_0$ tiene una sola solución si $w_0$ está dentro de $\Gamma'$, que $f(z)$ es holomorfa e inyectiva en $\Omega$ y lleva $\Omega$ en $\Omega'$ (el interior de $\Gamma'$) y que la dirección positiva de $\Gamma'$ se corresponde con la dirección positiva de $\Gamma$.
\end{proof}

\begin{comment}
:Title: Homotopy of paths
:Tags: Coordinate calculations, Decorations, Diagrams, Geometry, Mathematics
:Author: Alain Matthes
:Slug: homotopy
http://texblog.net/latex-archive/maths/jpgfdraw-example/ rewritten in TikZ

Following an illustration in Singer/Thorpe: Lecture Notes in Elementary Topology
and Geometry, the example has been drawn by Stefan Kottwitz using jpgfdraw,
and programmed by Alain Matthes on http://tex.stackexchange.com/q/1238/ .

\begin{tikzpicture}
  \node at (0,0) {$F : I \times I \rightarrow X$};
  \node[label=below:$x_1$]  (x1) at (6,0)  {$\bullet$};
  \node[label=above:$x_0$]  (x0) at (9,4)  {$\bullet$};
  \node  at (9.5,2)  {$\subset X$};
  \draw (x1.center) to [out=5,in=-90]++(2.8,1.8) to[out=90,in=-95](x0.center);
  \draw (x1.center) to [out=10,in=-110]++(2.6,2) to[out=70,in=-103](x0.center);
  \draw (x1.center) to [out=15,in=-105](x0.center);
  \draw (x1.center) to [out=30,in=-150](x0.center);
  \draw (x1.center) to [out=45,in=-170](x0.center);
  \draw (x1.center) to [out=50,in=-105]++(1.2,3)to [out=75,in=-172](x0.center);
  \draw (x1.center) to [out=55,in=-100]++(1.0,3) to[out=80,in=-175](x0.center);
  \draw (x1.center) to [out=60,in=-90]++(0.8,3) to[out=90,in=-180] (x0.center);
  \begin{scope}[every node/.style={draw, anchor=text, rectangle split,
    rectangle split parts=7,minimum width=2cm}]
    \node (R) at (2,4){ \nodepart{two} \nodepart{three}
    \nodepart{four}$I\times I$\nodepart{five}\nodepart{six}\nodepart{seven}};
  \end{scope}
  \draw[decorate,decoration={brace,mirror,raise=6pt,amplitude=10pt}, thick]
    (R.north west)--(R.south west) ;
  \draw[decorate,decoration={brace,raise=6pt,amplitude=10pt}, thick]
    (R.north east)--(R.south east);
  \draw[->] ($(R.west)+(-20pt,0)$) to[out=-180,in=240] ++(0,2)
    to [out=60,in=120]node[above,midway]{$F(0,t_2)$}(x0) ;
  \draw[->] ($(R.north)+(0,10pt)$) to [out=60,in=120]
    node[above,midway]{$\beta \simeq \alpha$} ++(4.5,-1) ;
  \draw[->] ($(R.east)+(20pt,0)$)  to [out=0,in=140]
    node[right,midway]{$F(1,t_2)$}(x1) ;
  \draw[->] ($(R.south)+(0,-20pt)$)  to [out=-85,in=-30]
    node[below,midway]{$\alpha$}++(7,0) ;
\end{tikzpicture}

\begin{tikzpicture}
\draw (2,2) circle (3cm);

\draw (3,0) arc (0:75:3cm);

\draw (3,0) .. controls (0,-0.3) and (1,-1) .. (0,1);
\end{tikzpicture}

\begin{tikzpicture}[use Hobby shortcut,closed=true]
      \filldraw [gray] (-3.5,0.5) circle (2pt)
                   (-3,2.5) circle (2pt)
                   (-1,3.5) circle (2pt)
                   (1.5,3) circle (2pt)
                   (4,3.5) circle (2pt)
                   (5,2.5) circle (2pt)
                   (5,0.5) circle (2pt)
                   (2.5,-2) circle (2pt)
                   (0,-0.5) circle (2pt)
                   (-3,-2) circle (2pt)
                   (-3.5,0.5) circle (2pt);

    \draw (-3.5,0.5) .. (-3,2.5) .. (-1,3.5).. (1.5,3).. (4,3.5).. (5,2.5).. (5,0.5) .. (2.5,-2).. (0,-0.5).. (-3,-2).. (-3.5,0.5);
\end{tikzpicture}

\begin{tikzpicture}[use Hobby shortcut,closed=true]
         \filldraw [gray] (1,0) circle (2pt)
                   (0.2,1) circle (2pt)
                   (1.3,2) circle (2pt)
                   (1.7,3) circle (2pt)
                   (2,4.2) circle (2pt)
                   (3,5.2) circle (2pt)
                   (4.2,4) circle (2pt)
                   (4,2.2) circle (2pt)
                   (3.2,1) circle (2pt)
                   (2.2,-0.2) circle (2pt);

    \draw (1,0) .. (0.2,1) .. (1.3,2) .. (1.7,3) .. (2,4.2) .. (3,5.2) ..
            (4.2,4) .. (4,2.2) .. (3.2,1) .. (2.2, -0.2);
\end{tikzpicture}
\end{comment}
