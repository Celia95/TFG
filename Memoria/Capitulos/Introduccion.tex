%---------------------------------------------------------------------
%
%                          Capítulo 1
%
%---------------------------------------------------------------------

\chapter{Introducción}
\label{cap:introduccion}

En $1545$, Gerolamo Cardano introduce los números complejos en \textit{Ars Magna}. En $1572$, Rafael Bombelli establece la reglas de cálculo correspondientes en su libro \textit{Álgebra}. Sin embargo, la aceptación general dentro de la comunidad matemática de la teoría de funciones complejas precisa logros tan destacados como los de Euler dentro del campo (por ejemplo, el teorema de adición para integrales elípticas, o la identidad que hoy conocemos como de Euler, que relaciona las funciones circulares con la exponencial), más el apoyo decidido de Gauss. Este defiende en una carta de $1811$ la importancia de considerar las funciones de argumentos complejos para reconocer propiedades de las correspondientes funciones de argumentos reales, apuntando que pueden quedar ocultas si se evita el uso de magnitudes complejas. Por otro lado, Gauss populariza la imagen actual del conjunto de números complejos como el plano, permitiendo un modelo geométrico sobre el que razonar y desarrollar ideas e intuiciones. \\

Con este marco consolidado, la teoría de funciones se desarrolla en el siglo XIX, de mano de los pioneros A.L. Cauchy ($1789-1857$), B. Riemann ($1826-1866$) y K. Weierstrass ($1815-1897$). Cada uno de ellos adopta un punto de vista particular. Por un lado, Cauchy fundamenta su trabajo, desarrollado entre $1814$ y $1825$, en la noción de función holomorfa: aquella que es derivable en sentido complejo, con derivada continua. Cauchy basa sus resultados en la técnica de integración y el concepto de residuo. Riemann, dos décadas más tarde, adopta un punto de vista geométrico, entendiendo que las funciones holomorfas son aquellas que transforman dominios del plano en conjuntos similares a pequeña escala (funciones conformes). Su trabajo se apoya en su intuición y experiencia de la física matemática. Finalmente, el proyecto de Weierstrass arranca de la noción de serie de potencias, dado que localmente toda función holomorfa lo es, en una línea de trabajo, iniciada por Lagrange, que pretende algebrizar el análisis. \\

A lo largo de esta memoria recogemos ideas de cada uno de estos matemáticos usando técnicas de integración, analizando propiedades de las funciones conformes o el comportamiento en la frontera de las series de potencias. Pero también nos adentramos en problemas más modernos, usando resultados y puntos de vista más actuales. Además, hacemos un tratamiento informático que nos va a permitir realizar una interpretación geométrica de los temas que abordamos. \\

En el Capítulo \ref{cap:fatou} estudiamos el núcleo y la integral de Poisson y algunas de sus propiedades más relevantes. Gracias a la integral de Poisson vamos a poder dar solución al problema de Dirichlet en el disco que pretende encontrar una función armónica en el disco que coincide en el borde del disco con una función continua dada $f$ definida en la frontera. Los problemas relacionados con esta tarea se estudiaron ya en $1840$ por Gauss, y luego por Dirichlet.
A partir de este resultado, analizamos la existencia de límites radiales mediante el Teorema de Fatou. Además, se introduce la noción de aplicación conforme y se muestra en este contexto la existencia de extensión al borde del disco de funciones holomorfas conformes entre el disco y un dominio de Jordan, resultado conocido como Teorema de Carathéodory. Ambos teoremas, tanto el de Fatou como el de Carathéodory pueden ser considerados una continuación natural en la línea de los trabajos de los pioneros. \\

Con el fin de complementar el trabajo teórico realizado a lo largo del trabajo, introducimos una técnica diseñada para representar funciones complejas en el plano mediante colores conocida como técnica de coloreado del dominio. Gracias a ella hemos podido realizar una aplicación en Python que permite representar funciones complejas así como resolver el problema de Dirichlet para el disco y representar la función resultante. Estas representaciones nos serán de gran utilidad a la hora de visualizar propiedades tanto geométricas como analíticas de algunas funciones y por ello se incluirán figuras para ayudar al lector a examinar su comportamiento. \\

Seguidamente se detallan y analizan ejemplos concretos de funciones definidas por series y productos infinitos o por fórmulas explícitas, que ilustran la gran variedad de comportamiento en la frontera de las series en su disco de convergencia. Algunos de estos ejemplos volverán a surgir más adelante para ser estudiados desde una perspectiva distinta, como será el caso de los productos de Blaschke. \\

A continuación nos adentramos en la estructura como álgebra de Banach del espacio de funciones holomorfas y acotadas en el disco. Esto nos permite hacer una introducción a la teoría desarrollada por Gelfand a partir del trabajo fundamental de su tesis doctoral, presentada en $1936$ y conectar con trabajos posteriores, como el estudio en puntos singulares del borde de los valores adherentes (problema con origen en la década de $1960$). Con este propósito analizamos el espectro de dicho álgebra y su relación con el comportamiento en la frontera de una función determinada dada a través de su transformada de Gelfand. Seremos capaces de identificar el comportamiento cerca de puntos $\alpha$ de la frontera del disco a través de las fibras $\fiber_\alpha$. Definiremos el conjunto de valores adherentes de una función en un punto y trataremos de desentrañar la estructura de dicho conjunto según el comportamiento de la función en dicho punto. Para concluir, probaremos que el conjunto de puntos adherentes de una función $f$ en un punto del borde $\alpha$ coincide con la imagen de la fibra de $\alpha$ a través de la transformada de Gelfand de $f$. \\

Por último, se da una lectura geométrica del lema de Schwarz y su generalización, el lema de Schwarz-Pick. Se presenta la noción de límite angular en la frontera, para comprender la relación entre el equivalente de los lemas citados para discos que son tangentes en un punto de la frontera. Con la intención de analizar bajo distintos puntos de vista el comportamiento en la frontera de funciones holomorfas, incluimos resultados ya considerados clásicos, como los Teoremas de Julia y Julia-Carathéodory, desde una perspectiva más moderna. Esta línea posee numerosas aplicaciones en problemas actuales, como el análisis de operadores de composición entre distintas álgebras de funciones holomorfas en términos de las propiedades geométricas de la función destacada. \\

