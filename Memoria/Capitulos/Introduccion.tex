%---------------------------------------------------------------------
%
%                          Capítulo 1
%
%---------------------------------------------------------------------

\chapter{Introducción}
\label{cap:introduccion}

En $1545$, Cardano introduce los números complejos en  \textit{Ars Magna}. En $1572$, Bombelli establece la reglas de cálculo correspondientes en su libro \textit{Algebra}. Sin embargo, la aceptación general dentro de la comunidad matemática de la teoría de funciones complejas precisa logros tan destacados como los de Euler dentro del campo (por ejemplo, el teorema de adición para integrales elípticas, o la identidad que hoy conocemos como de Euler, que relaciona las funciones circulares con la exponencial), más el apoyo decidido de Gauss. Este defiende en una carta de $1811$ la importancia de considerar las funciones de argumentos complejos para reconocer propiedades de las correspondientes funciones de argumentos reales, apuntando que pueden quedar ocultas si se evita el uso de magnitudes complejas. Por otro lado, Gauss populariza la imagen actual del conjunto de números complejos como el plano, permitiendo un modelo geométrico sobre el que razonar y desarrollar ideas e intuiciones. \\

Con este marco consolidado, la teoría de funciones se desarrolla en el siglo XIX, de mano de los pioneros A.L. Cauchy (1789-1857), B. Riemann (1826-1866) y K. Weierstrass (1815-1897). Cada uno de ellos adopta un punto de vista particular. Por un lado, Cauchy fundamenta su trabajo, desarrollado entre $1814$ y $1825$, en la noción de función holomorfa: aquella que es derivable en sentido complejo, con derivada continua. Cauchy basa sus resultados en la técnica de integración y el concepto de residuo. Riemann, dos décadas más tarde, adopta un punto de vista geométrico, entendiendo que las funciones holomorfas son aquellas que transforman dominios del plano en conjuntos similares a pequeña escala (funciones conformes). Su trabajo se apoya en su intuición y experiencia de la física matemática. Finalmente, el proyecto de Weierstrass arranca de la noción de serie de potencias, dado que localmente toda función holomorfa lo es, en una línea de trabajo, iniciada por Lagrange, que pretende algebrizar el análisis. \\

A lo largo de esta memoria recogemos ideas de cada uno de estos matemáticos, usando técnicas de integración, analizando propiedades de las funciones conformes o el comportamiento en la frontera de las series de potencias. Pero también nos adentramos en problemas más modernos, usamos resultados o puntos de vista más actuales y hacemos un tratamiento informático (???) de los temas que abordamos. En concreto, presentamos los teoremas de Fatou y Carathéodory, que pueden ser considerados una continuación natural en la línea de los trabajos de los pioneros. Nos adentramos en la estructura de álgebra de Banach del espacio de funciones holomorfas y acotadas en el disco. Esto nos permite hacer una introducción a la teoría desarrollada por Gelfand a partir del trabajo fundamental de su tesis doctoral, presentada en $1936$, y conectar con trabajos posteriores, como el estudio en puntos singulares del borde de los valores adherentes (problema con origen en la década de $1960$). Con la perspectiva de analizar bajo distintos puntos de vista el comportamiento en la frontera de funciones holomorfas, incluimos resultados ya considerados clásicos, como los de Julia (Julia-Carathéodory) desde un punto de vista más moderno. Esta línea posee numerosas aplicaciones en problemas actuales, como el análisis de operadores de composición entre distintas álgebras de funciones holomorfas en términos de las propiedades geométricas de la función destacada. \\
