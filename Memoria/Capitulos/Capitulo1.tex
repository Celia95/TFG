\chapter{Teorema de Fatou y Teorema de Carathéodory}

\begin{comment}
    \begin{theorem}[Lema de Schwart]
        Sea $f: \mathbb{D} \rightarrow \overline{\mathbb{D}}$ una función $\in \mathcal{H}(\mathbb{D})$ tal que $f(0) = 0$. Entonces:
        \begin{itemize}
            \item $\abs{f(z)} \leq \abs{z}$ para todo $z \in \mathbb{D}$.
            \item Si para algún $z_0 \not = 0$ tenemos que $\abs{f(z_0)} = \abs{z_0}$, entonces existe $\alpha \in \mathbb{C}, \abs{\alpha} = 1$ tal que $f(z)=\alpha z$.
        \end{itemize}
    \end{theorem}

    \begin{proof}
        Sea $f(z) = a_1z + \cdots$ la serie de potencias de $f$. El término constante es $0$ puesto que suponemos que $f(0) = 0$. Entonces $f(z)/z$ es una función holomorfa y
        \begin{equation*}
            \abs{\dfrac{f(z)}{z}} < 1/r \text{ para } \abs{z} = r < 1
        \end{equation*}
    \end{proof}
\end{comment}

\section{Integral de Poisson y Teorema de Fatou}

\subsection{La Integral de Poisson}

\begin{definition}
    Se llama núcleo de Poisson a la función $P$ definida por
    \begin{equation}
        \label{poisson1}
        P:(r,t) \in [0,1)\times \mathbb{R} \mapsto P_r(t) = \sum_{n=-\infty}^{\infty} r^{\abs{n}}e^{int}.
    \end{equation}

    Podemos considerar el núcleo de Poisson como una función de dos variables $r$ y $t$ o como una familia de funciones de $t$ que dependen de $r$.

    Dados $z=re^{i \theta}$, con $r \in [0,1)$ y $\theta \in \mathbb{R}$ se tiene que
    \begin{equation}
    \label{poisson2}
        P_r(\theta - t) = \Re \left[ \dfrac{e^{it} + z}{e^{it} - z} \right] = \dfrac{1 - r^2}{1 - 2r \cos (\theta - t) + r^2}
    \end{equation}
    para todo $t \in \mathbb{R}$. En efecto:

    \begin{multline*}
        P_r(t) = \sum_{n=-\infty}^{\infty} r^{\abs{n}}e^{int} = 1 + \sum_{n=1}^{\infty} r^n e^{int} + \sum_{n=1}^{\infty} r^n e^{-int} = 1 + \sum_{n=1}^{\infty} r^n (e^{int} + e^{-int}) = \\
        1 + \sum_{n=1}^{\infty} r^n 2 \Re(e^{int}) = \Re \left[ 1 + 2 \sum_{n=1}^{\infty} (r e^{it})^n  \right] = \Re \left[ 1 + 2 \dfrac{re^{it}}{1-re^{it}} \right] = \Re \left[\dfrac{1 + re^{it}}{1-re^{it}} \right].
    \end{multline*}
    \\ \par
    Por otra parte
    \begin{equation*}
        \Re \left[ \dfrac{1 + re^{it}}{1-re^{it}} \right] = \Re \left[ \dfrac{(1 + re^{it})(1 - re^{it})}{\abs{1-re^{it}}^2} \right] = \dfrac{1 - r^2}{1 - 2r \cos (\theta - t) + r^2}.
    \end{equation*}
\end{definition}

\newpage

\textbf{Propiedades del núcleo de Poisson:}% \\ \par

%De \ref{poisson1} tenemos que
\begin{equation}
    \dfrac{1}{2 \pi} \int_{- \pi}^{\pi} P_r (t) dt = 1, \forall r \in [0,1).
\end{equation}

%De \ref{poisson2} se sigue que
\begin{equation}
    P_r(t) > 0, \forall r \in [0,1), t \in \mathbb{R}
\end{equation}

\begin{equation}
    P_r(t) = P_r(-t), \forall r \in [0,1), t \in \mathbb{R}
\end{equation}

\begin{equation}
    P_r(t) < P_r(\delta), 0 < \delta < \abs{t} \leq \pi
\end{equation}

\begin{equation}
    \lim_{r \rightarrow 1} P_r(\delta) = 0, \forall \delta \in (0,\pi]
\end{equation}

\bigskip \par

\begin{definition}
    Se llama integral de Poisson de una función $f \in L^1(\partial \mathbb{D})$ a la función $F$ dada por
    \begin{equation*}
        F: z=re^{i \theta} \in \mathbb{D} \mapsto F(re^{i \theta}) = \dfrac{1}{2 \pi} \int_{- \pi}^{\pi} P_r (\theta - t) f(t) dt.
    \end{equation*}

    Algunas veces nos convendrá referirnos a ella como $F=P[f]$.
\end{definition}

\bigskip

Además si $f$ lleva $\partial \mathbb{D}$ en los reales, \ref{poisson2} nos muestra que

\begin{equation*}
     P[f] = \Re \left[ \dfrac{1}{2} \int_{-\pi}^{\pi} \dfrac{e^{it} + z}{e^{it} - z} f(t) dt \right].
\end{equation*}

\subsection{Teorema de Fatou}

Para demostrar el Teorema de Fatou nos vamos a basar en unos resultados clásicos del libro \citet[chap. 11]{rudin} que no vamos a probar.

\begin{theorem}
\label{fatouaux1}
    Si $f \in L^1(\partial \mathbb{D})$ y $F = P[f]$, entonces
    \begin{equation*}
        \lim_{r \rightarrow 1} F(re^{i \theta}) = f(e^{i \theta})
    \end{equation*}
\end{theorem}

\begin{theorem}
\label{fatouaux2}
    Sean $f \in C (\partial \mathbb{D}), F=P[f]$ y
    \begin{equation*}
        u(re^{i \theta}) =
        \begin{cases}
            f(re^{i\theta}) & \text{si } r=1\\ F(re^{i\theta}) & \text{si } 0 \leq r<1
        \end{cases}
    \end{equation*}
    Entonces $u$ es una función continua en el disco cerrado $\xbar{\mathbb{D}}$.
\end{theorem}

\begin{theorem}[Teorema de Fatou]
    Para toda función $f \in \mathcal{H}^\infty(\mathbb{D})$, existe una función $f^* \in L^{\infty} (\partial \mathbb{D})$ definida en casi todo punto tal que
    \begin{equation}
        \label{fatou1}
        f^*(e^{it}) = \lim_{r \rightarrow 1} f(re^{it})
    \end{equation}

    Se tiene la igualdad $\norminf{f} = \norminf{f^*}$. Para todo $z \in U$, la fórmula integral de Cauchy
    \begin{equation}
        \label{fatou2}
        f(z) = \dfrac{1}{2 \pi i} \int_{\gamma} \dfrac{f^*(\xi)}{\xi - z} d\xi
    \end{equation}

    se satisface, donde $\gamma$ es el círculo unidad positivamente orientado: $\gamma(t) = e^{it}, 0 \leq t \leq 2 \pi$.

    Las funciones $f^* \in L^{\infty}(\partial \mathbb{D})$ que se obtienen mediante este procedimiento son precisamente aquellas que cumplen la siguiente relación

    \begin{equation}
        \label{fatou3}
        \dfrac{1}{2 \pi i} \int_{-\pi}^{\pi} f^*(e^{it})e^{-int} dt = 0, n = -1,-2, \dots
    \end{equation}
\end{theorem}

\begin{proof}
    La existencia de $f^*$ se sigue de los teoremas \ref{fatouaux1} y \ref{fatouaux2}.

    Por \ref{fatou1}, tenemos que $\norminf{f^*} \leq \norminf{f}$.

    Si $z \in U$ y $\abs{z} < r < 1$, tomemos $\gamma_r(t) = r e^{it}, 0 \leq t \leq 2\pi$. Entonces,
    \begin{equation*}
        f(z) = \dfrac{1}{2 \pi i} \int_{\gamma_r} \dfrac{f(\xi)}{\xi - z} d\xi =
        \dfrac{r}{2 \pi} \int_{-\pi}^{\pi} \dfrac{f(re^{it})}{re^{it} - z} dt
    \end{equation*}

    Sea $\{r_n\}$ una sucesión tal que $r_n \rightarrow 1$. Por el teorema de la convergencia dominada de Lebesgue tenemos
    \begin{equation}
        \label{fatou_proof}
        f(z) = \dfrac{1}{2 \pi} \int_{-\pi}^{\pi} \dfrac{f^* (e^{it})}{1 - ze^{it}} dt
    \end{equation}
    Por lo que ya hemos probado \ref{fatou2}. Por el teorema de Cauchy, se sigue que
    \begin{equation*}
        \int_{\gamma_r} f(\xi)\xi^n d\xi = 0, n = 0, 1, \dots
    \end{equation*}

    Pasando al límite tenemos que $f^*$ cumple \ref{fatou3}. Además, podemos convertir \ref{fatou_proof} en una integral de Poisson, si $z = re^{i \theta}$,
    \begin{equation*}
         \begin{split}
             f(z) &  = \dfrac{1}{2 \pi} \int_{- \pi}^{\pi} f^*(e^{it}) \sum_{n=0}^{\infty} r^n e^{in(\theta - t)} dt =  \dfrac{1}{2 \pi} \int_{- \pi}^{\pi} f^*(e^{it}) \sum_{n=-\infty}^{\infty} r^{\abs{n}} e^{in(\theta - t)} dt = \\
                  & =  \dfrac{1}{2 \pi} \int_{- \pi}^{\pi} P_r(\theta - t) f^*(e^{it}) dt
         \end{split}
    \end{equation*}

    De esto concluimos que $\norminf{f} \leq \norminf{f^*}$, así que ambas normas coinciden.
\end{proof}

\section{Teorema de Carathéodory}

\begin{definition}{Aplicación conforme}
    Sean $U$ y $V \subset \mathbb{C}^n$. Una aplicación $f: U \rightarrow V$ se llama conforme en un punto $u \in U$ si preserva la orientación y los ángulos entre curvas que pasan por $u$.
\end{definition}

\begin{prop}
    Sea $U \subset \mathbb{C}$. Una aplicación $f: U \rightarrow \mathbb{C}$ es conforme en $U$ si (y solo si) $f \in \mathcal{H}(U)$ y $f'(z) \not = 0 \, \forall z \in U$.
\end{prop}

\begin{proof}
    $(\Leftarrow)$ Supongamos que $f: U \rightarrow \mathbb{C}$ es holomorfa en $U$ y $f'(z) \not = 0 \, \forall z \in U$. Sea $\gamma: [a,b] \rightarrow U$ una curva simple y sea $z = z(t) = x(t) + iy(t) = \gamma (t)$ la ecuación paramétrica de $\gamma$. Consideremos $w = (f \circ  z)(t)$ holomorfa en $U$.

    Ahora por la regla de la cadena $w'(t) = f'(z(t))z'(t)$. Como el argumento es aditivo para la multiplicación de funciones, tenemos
    \begin{equation*}
        \arg{w'(t_0)} = \arg{f'(z(t_0))} + \arg{z'(t_0)}
    \end{equation*}

    $f(z)$ rota el vector tangente a $\gamma(t_0) = z_0$ por el mismo ángulo ($\arg{f'(z_0)}$), independientemente del camino $\gamma$.

    Si nos fijamos en el módulo, obtenemos un resultado similar,
    \begin{equation*}
        \lim_{z \rightarrow z_0} \dfrac{\abs{f(z) - f(z_0)}}{\abs{z - z_0}} = \abs{f'(z_0)}
    \end{equation*}

    Así, $f(z)$ cambia el módulo de la distancia entre los puntos por el mismo factor, independientemente de la dirección en la que nos aproximemos a $z_0$. \\

    \noindent\rule{8cm}{0.4pt}

    Podemos escribir
    \begin{equation*}
        w - w_0 = f(z) - f(z_0) = \dfrac{f(z)-f(z_0)}{z - z_0} (z - z_0)
    \end{equation*}Entonces

    Entonces
    \begin{equation*}
        \arg{(w - w_0)} =  \arg{ \left( \dfrac{f(z)-f(z_0)}{z - z_0} \right)} + \arg{(z - z_0)}
    \end{equation*}

    Como $f'(z) \not = 0 \, \forall z \in U$, tomando límites tenemos
    \begin{equation*}
        \phi = \lim_{z \rightarrow z_0} \arg({w-w_0)}= \arg{f'(z_0)} + \theta
    \end{equation*}

    La aplicación $f$ ha rotado el vector tangente a $z_0$ un ángulo dado por $\text{arg } f'(z_0)$. Sean dos curvas que pasan por $z_0$ con ángulos $\theta_1$ y $\theta_2$ con respecto al eje $x$. Entonces el ángulo entre sus imágenes en $w_0$ es 
    \begin{equation*}
        \phi_2 - \phi_1 = \theta_2 + \text{arg } f'(z_0) - \theta_1 - \text{arg } f'(z_0) = \theta_2 - \theta_1
    \end{equation*}

    Por lo tanto, $f$ preserva la orientación y los ángulos entre curvas que pasan por $z, \, \forall z \in U$.

    \begin{comment}
        $(\Rightarrow)$ Supongamos que $f$ es conforme y que $\frac{\partial f}{\partial x}$ y $\frac{\partial f}{\partial y}$ son continuas. En este caso tenemos
        \begin{equation*}
            w'(t_0) = \dfrac{\partial f}{\partial x} x'(t_0) +  \dfrac{\partial f}{\partial y} y'(t_0)
        \end{equation*}

        Podemos reescribir esto de la siguiente manera
        \begin{equation*}
            w'(t_0) = \dfrac{1}{2} \left( \dfrac{\partial f}{\partial x} - i\dfrac{\partial f}{\partial y} \right) z'(t_0) + \dfrac{1}{2} \left( \dfrac{\partial f}{\partial x} + i\dfrac{\partial f}{\partial y} \right) \xbar{z'(t_0)}
        \end{equation*}

        Si los ángulos se preservan entonces $\arg{(w'(t_0/z'(t_0)))}$ es independiente de $z'(t_0)$. Así,
        \begin{equation*}
            \dfrac{1}{2} \left( \dfrac{\partial f}{\partial x} - i\dfrac{\partial f}{\partial y} \right) + \dfrac{1}{2} \left( \dfrac{\partial f}{\partial x} + i\dfrac{\partial f}{\partial y} \right) \left( \dfrac{\xbar{z'(t_0)}}{z'(t_0)} \right)
        \end{equation*}
        tiene argumento constante.

        Supongamos que $z'(t_0) = re^{i \theta}$. Entonces
        \begin{equation*}
            \dfrac{\xbar{z'(t_0)}}{z'(t_0)} = \dfrac{re^{-i \theta}}{re^{i \theta}} = e^{-2i \theta}
        \end{equation*}

        Mientras $z'(t_0)$ varía, la curva describe una circunferencia de radio $\frac{1}{2} \abs{\frac{\partial f}{\partial x} - i\frac{\partial f}{\partial y}}$. El argumento solo puede ser constante si $\frac{\partial f}{\partial x} - i\frac{\partial f}{\partial y}$ es cero. Como hemos visto, estas ecuaciones implican que es holomorfa.
    \end{comment}
\end{proof}

\bigskip

\begin{theorem}[Teorema de Carathéodory]
    Sea $\varphi$ una aplicación conforme del disco unidad $\mathbb{D}$ en un dominio de Jordan $\Omega$. Entonces $\varphi$ tiene una extensión continua al disco cerrado $\overline{\mathbb{D}}$, y la extensión es inyectiva de $\overline{\mathbb{D}}$ en $\Omega$.
\end{theorem}

\begin{proof}
    Vamos a suponer que $\Omega$ está acotado. Fijemos $\zeta \in \partial \mathbb{D}$. Primero vamos a probar que $\varphi$ tiene una extensión continua en $\zeta$. Sea $0 < \delta < 1$,
    \begin{equation*}
        D(\zeta, \delta) = \{z: \abs{z - \zeta} < \delta \}
    \end{equation*}

    y tomemos $\gamma_{\delta} = \mathbb{D} \cap \partial D(\zeta, \delta)$. Entonces $\varphi (\gamma_{\delta})$ es una curva de Jordan de longitud
    \begin{equation*}
        L(\delta) = \int_{\gamma_{\delta}} \abs{\varphi ' (z)} ds
    \end{equation*}
    \\
    Por la desigualdad de Cauchy-Schwartz, tenemos
    \begin{equation*}
        L^2(\delta) \leq \pi \delta \int_{\gamma_{\delta}} \abs{\varphi ' (z)}^2 ds
    \end{equation*}

    entonces para $\rho < 1$

    \begin{equation*}
        \int_{0}^{\rho} \frac{L^2(\delta)}{\delta} d\delta \leq \pi \int \int_{\mathbb{D} \cap D(\zeta, \rho)} \abs{\varphi ' (z)}^2 dxdy = \pi \text{Area}(\varphi(\mathbb{D} \cap D(\zeta, \rho))) < \infty
    \end{equation*}
    \\
    %Figura
    Entonces, existe una sucesión $\{ \delta_n\} \downarrow 0$ tal que $L(\delta_n) \rightarrow 0$. Cuando $L(\delta_n) < \infty$, la curva $\varphi(\gamma_{\delta_n})$ tiene extremos $\alpha_n, \beta_n \in \xbar{\Omega}$ y ambos puntos deben estar en $\Gamma = \partial \Omega$. De hecho, si $\alpha_n \in \Omega$, entonces algún punto cerca de $\alpha_n$ tiene dos preimágenes distintas en $\mathbb{D}$ y esto es imposible pues $\varphi$ es inyectiva. Además,
    \begin{equation}\label{res}
        \abs{\alpha_n - \beta_n} \leq L(\delta_n) \rightarrow 0
    \end{equation}
    \\
    Sea $\sigma_n$ el subarco cerrado de $\Gamma$ que tiene extremos $\alpha_n$ y $\beta_n$ y con un diámetro menor. Entonces \ref{res} implica que $\text{diam}(\sigma_n) \rightarrow 0$ porque $\Gamma$ es homeomorfa al círculo. Por el teorema de la curva de Jordan, $\sigma_n \cup \varphi(\gamma_{\delta_n})$ divide al plano en dos regiones, y una de ellas, llamémosla $U_n$ es acotada. Entonces $U_n \subset \Omega$ ya que $\mathbb{C}^* \setminus \xbar{\Omega}$ es conexo por arcos. Como
    \begin{equation}
        \label{res2}
        \text{diam}(\partial U_n) = \text{diam}(\sigma_n \cup \varphi(\gamma_{\delta_n})) \rightarrow 0,
        \text{ concluimos que }
        \text{diam}(U_n) \rightarrow 0.
    \end{equation}

    Tomamos $D_n = \mathbb{D} \cup \{ z: \abs{z - \zeta} < \delta_n \}$. Sabemos que para $n$ suficientemente grande, $\varphi(D_n) = U_n$. Si no, por conexión tendríamos que $\varphi(\mathbb{D} \setminus \xbar{D_n}) = U_n$ y
    \begin{equation*}
        \text{diam} (U_n) \geq \text{diam} (\varphi(B(0, 1/2))) > 0
    \end{equation*}

    que contracide con \ref{res2}. Entonces $\text{diam}(\varphi(D_n)) \rightarrow 0$ y $\bigcap \xbar{\varphi(D_n)}$ es un solo punto pues $\varphi(D_{n+1}) \subset \varphi(D_n)$. Esto significa que $\varphi$ tiene una extensión continua en $\mathbb{D} \cap \{ \zeta \}$. La extensión a todos estos puntos define una aplicación continua en $\overline{\mathbb{D}}$.

    Denotemos ahora por $\varphi$ a la extensión $\varphi : \overline{\mathbb{D}} \rightarrow \xbar{\Omega}$. Como $\varphi(\mathbb{D}) = \Omega$, $\varphi$ lleva  $\overline{\mathbb{D}}$ en $\xbar{\Omega}$. Para probar que $\varphi$ es inyectiva, supongamos que $\varphi(\zeta_1) = \varphi(\zeta_2), \zeta_1 \not = \zeta_2$. El argumento utilizado para mostrar que $\alpha_n \in \Gamma$, también prueba que $\varphi (\partial \mathbb{D}) = \Gamma$, así que podemos suponer que $\zeta_j \in \partial \mathbb{D}, j=1,2$. La curva de Jordan
    \begin{equation*}
        \{\varphi (r \zeta_1) : 0 \leq r \leq 1\} \cup \{\varphi (r \zeta_2) : 0 \leq r \leq 1\}
    \end{equation*}

    acota al dominio $W \subset \Omega$, luego $\varphi ^{-1} (W)$ es una de las dos componentes de
    \begin{equation*}
        \mathbb{D} \setminus ( \{ r \zeta_1 : 0 \leq r \leq 1\} \cup \{ r \zeta_2 : 0 \leq r \leq 1\})
    \end{equation*}
    \\
    Pero como $\varphi(\partial \mathbb{D}) \subset \Gamma$,
    \begin{equation*}
        \varphi(\partial \mathbb{D} \cap \partial \varphi ^{-1} (W)) \subset \partial W \cap \partial \Omega = \{ \varphi (\zeta_1)\}
    \end{equation*}

    y $\varphi$ es constante en un arco de $\partial \mathbb{D}$, se tiene que $\varphi$ es constante y esta contradicción prueba que $\varphi(\zeta_1) \not = \varphi(\zeta_2)$.
\end{proof}

\begin{theorem}
    Sea $C$ un camino simple, cerrado y continuamente diferenciable con interior $D$. Sea $f \in \mathcal{H}(C \cup D)$ una aplicación inyectiva en $C$. Entonces $f$ es holomorfa e inyectiva en $D$.
\end{theorem}

\begin{proof}
    La aplicación $w = f(z)$ lleva $C$ en un camino simple, cerrado y continuamente diferenciable $C'$. Sea $w_0$ un punto arbitrario que no esté en $C'$. Entonces,
    \begin{equation*}
        n = \dfrac{1}{2 \pi i} \int_{C_+} \dfrac{f'(z)}{f(z) - w_0} dz =  \dfrac{1}{2 \pi i} \int_{C'} \dfrac{dw}{w - w_0}.
    \end{equation*}

    Ahora la última integral es cero si $w_0$ está fuera de $C'$ y es $\pm 1$ si $w_0$ está dentro de $C'$. Sin embargo, $n$ no puede ser negativo pues la primera integral nos da el número de ceros de $f(z) - w_0$ dentro de $C$. Entonces, $n=1$ si $w_0$ está dentro de $C'$.

    Esto prueba que $f(z) = w_0$ tiene una sola solución si $w_0$ está dentro de $C'$, que $f(z)$ es holomorfa e inyectiva en $D$ y lleva $D$ en $D'$ (el interior de $C'$) y que la dirección positiva de $C'$ se corresponde con la dirección positiva de $C$.
\end{proof}

\begin{comment}
:Title: Homotopy of paths
:Tags: Coordinate calculations, Decorations, Diagrams, Geometry, Mathematics
:Author: Alain Matthes
:Slug: homotopy
http://texblog.net/latex-archive/maths/jpgfdraw-example/ rewritten in TikZ

Following an illustration in Singer/Thorpe: Lecture Notes in Elementary Topology
and Geometry, the example has been drawn by Stefan Kottwitz using jpgfdraw,
and programmed by Alain Matthes on http://tex.stackexchange.com/q/1238/ .

\begin{tikzpicture}
  \node at (0,0) {$F : I \times I \rightarrow X$};
  \node[label=below:$x_1$]  (x1) at (6,0)  {$\bullet$};
  \node[label=above:$x_0$]  (x0) at (9,4)  {$\bullet$};
  \node  at (9.5,2)  {$\subset X$};
  \draw (x1.center) to [out=5,in=-90]++(2.8,1.8) to[out=90,in=-95](x0.center);
  \draw (x1.center) to [out=10,in=-110]++(2.6,2) to[out=70,in=-103](x0.center);
  \draw (x1.center) to [out=15,in=-105](x0.center);
  \draw (x1.center) to [out=30,in=-150](x0.center);
  \draw (x1.center) to [out=45,in=-170](x0.center);
  \draw (x1.center) to [out=50,in=-105]++(1.2,3)to [out=75,in=-172](x0.center);
  \draw (x1.center) to [out=55,in=-100]++(1.0,3) to[out=80,in=-175](x0.center);
  \draw (x1.center) to [out=60,in=-90]++(0.8,3) to[out=90,in=-180] (x0.center);
  \begin{scope}[every node/.style={draw, anchor=text, rectangle split,
    rectangle split parts=7,minimum width=2cm}]
    \node (R) at (2,4){ \nodepart{two} \nodepart{three}
    \nodepart{four}$I\times I$\nodepart{five}\nodepart{six}\nodepart{seven}};
  \end{scope}
  \draw[decorate,decoration={brace,mirror,raise=6pt,amplitude=10pt}, thick]
    (R.north west)--(R.south west) ;
  \draw[decorate,decoration={brace,raise=6pt,amplitude=10pt}, thick]
    (R.north east)--(R.south east);
  \draw[->] ($(R.west)+(-20pt,0)$) to[out=-180,in=240] ++(0,2)
    to [out=60,in=120]node[above,midway]{$F(0,t_2)$}(x0) ;
  \draw[->] ($(R.north)+(0,10pt)$) to [out=60,in=120]
    node[above,midway]{$\beta \simeq \alpha$} ++(4.5,-1) ;
  \draw[->] ($(R.east)+(20pt,0)$)  to [out=0,in=140]
    node[right,midway]{$F(1,t_2)$}(x1) ;
  \draw[->] ($(R.south)+(0,-20pt)$)  to [out=-85,in=-30]
    node[below,midway]{$\alpha$}++(7,0) ;
\end{tikzpicture}

\begin{tikzpicture}
\draw (2,2) circle (3cm);

\draw (3,0) arc (0:75:3cm);

\draw (3,0) .. controls (0,-0.3) and (1,-1) .. (0,1);
\end{tikzpicture}

\begin{tikzpicture}[use Hobby shortcut,closed=true]
      \filldraw [gray] (-3.5,0.5) circle (2pt)
                   (-3,2.5) circle (2pt)
                   (-1,3.5) circle (2pt)
                   (1.5,3) circle (2pt)
                   (4,3.5) circle (2pt)
                   (5,2.5) circle (2pt)
                   (5,0.5) circle (2pt)
                   (2.5,-2) circle (2pt)
                   (0,-0.5) circle (2pt)
                   (-3,-2) circle (2pt)
                   (-3.5,0.5) circle (2pt);

    \draw (-3.5,0.5) .. (-3,2.5) .. (-1,3.5).. (1.5,3).. (4,3.5).. (5,2.5).. (5,0.5) .. (2.5,-2).. (0,-0.5).. (-3,-2).. (-3.5,0.5);
\end{tikzpicture}

\begin{tikzpicture}[use Hobby shortcut,closed=true]
         \filldraw [gray] (1,0) circle (2pt)
                   (0.2,1) circle (2pt)
                   (1.3,2) circle (2pt)
                   (1.7,3) circle (2pt)
                   (2,4.2) circle (2pt)
                   (3,5.2) circle (2pt)
                   (4.2,4) circle (2pt)
                   (4,2.2) circle (2pt)
                   (3.2,1) circle (2pt)
                   (2.2,-0.2) circle (2pt);

    \draw (1,0) .. (0.2,1) .. (1.3,2) .. (1.7,3) .. (2,4.2) .. (3,5.2) ..
            (4.2,4) .. (4,2.2) .. (3.2,1) .. (2.2, -0.2);
\end{tikzpicture}
\end{comment}
